% !TeX TS-program = xelatex
% This file is part of Mahāvīrī LaTeX Source Code
% Mahāvīrī LaTeX Source Code is free software: you can redistribute it and/or modify it under the terms of the GNU General Public License as published by the Free Software Foundation, either version 3 of the License, or (at your option) any later version.
% Mahāvīrī LaTeX Source Code is distributed in the hope that it will be useful, but WITHOUT ANY WARRANTY; without even the implied warranty of MERCHANTABILITY or FITNESS FOR A PARTICULAR PURPOSE. See the GNU General Public License for more details.
% You should have received a copy of the GNU General Public License along with Mahāvīrī LaTeX Source Code. If not, see <https://www.gnu.org/licenses/>.
% This code was compiled on 
\documentclass[12pt]{article}
\usepackage{calc}
\usepackage{ragged2e}
\newlength{\bleed}
\newlength{\mylength}
\newlength{\width}
\newlength{\height}
\newlength{\pdfwidth}
\newlength{\pdfheight}
\setlength{\bleed}{3mm}
\setlength{\width}{138mm}
\setlength{\height}{216mm}
\setlength{\pdfwidth}{\width*\real{1.0} + \bleed*\real{2.0}}
\setlength{\pdfheight}{\height + \bleed*\real{2.0}}
\usepackage[dvips=false,pdftex=false,vtex=false,margin=0in,paperwidth=\width,paperheight=\height]{geometry}
\usepackage[noaxes,noinfo,cam,dvips,pdftex,center,width=\the\pdfwidth,height=\the\pdfheight]{crop}
\usepackage[dvipsnames,prologue,table]{pstricks}
\usepackage{graphicx}
\usepackage{xcolor}
% ISBN is 978-81-931144-1-4
\usepackage[ISBN=978-81-931144-1-4]{ean13isbn}
\usepackage{polyglossia}
\setmainlanguage{hindi}
\setmainfont[Script=Devanagari, Path = ./, Extension = .ttf,UprightFont = *-Regular, BoldItalicFont = *-BoldItalic, BoldFont = *-Bold, ItalicFont = *-Italic, Mapping=devanagaridigits]{ChanakyaSanskrit}
\usepackage{hyperref}
\hypersetup{
	pdfstartview={XYZ null null 1},
	bookmarks=false
}
\setlength{\parindent}{0pt}
\begin{document}
\pagecolor{white}
\pagestyle{empty}
\psset{unit=1in}
\begin{pspicture}(\width,\height)
% Text above the image
\rput[lb](0mm,194mm){\parbox{\width}{\centering\fontsize{18}{27}\selectfont\color{red}{\textit{श्रीहनुमान्‌-चालीसा}के प्रशस्त व्याख्याकार}}}
% The image
\newsavebox\IBox
\setlength{\mylength}{\width*\real{0.5} - 45mm}
\fboxsep=0pt%padding thickness
\fboxrule=1pt%border thickness
\sbox\IBox{\fcolorbox{red}{yellow}{\includegraphics[width=90mm]{image.jpeg}}}
\rput[lb](\mylength,55mm){\usebox\IBox}
% Text below the image
\rput[lb](0mm,45mm){\parbox{\width}{\centering\fontsize{18}{27}\selectfont\color{red}{धर्मचक्रवर्ती महामहोपाध्याय वाचस्पति कविकुलरत्न महाकवि}}}
\rput[lb](0mm,37.5mm){\parbox{\width}{\centering\fontsize{18}{27}\selectfont\color{red}{प्रस्थानत्रयीभाष्यकार श्रीचित्रकूटतुलसीपीठाधीश्वर}}}
\rput[lb](0mm,30mm){\parbox{\width}{\centering\fontsize{18}{27}\selectfont\color{red}{ऋतम्भराचक्षु पद्मविभूषण-विभूषित जगद्गुरु रामानन्दाचार्य
}}}
\rput[lb](0mm,21.5mm){\parbox{\width}{\centering\fontsize{26}{39}\selectfont\color{red}{\textbf{स्वामी रामभद्राचार्य}}}}
\rput[lb](0mm,15mm){\parbox{\width}{\centering\fontsize{18}{27}\selectfont\color{red}{(१९५०–)}}}
\end{pspicture}
\end{document}