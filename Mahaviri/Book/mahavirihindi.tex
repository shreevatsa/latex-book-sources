% !TeX TS-program = lualatex
% This file is part of Mahāvīrī LaTeX Source Code
% Mahāvīrī LaTeX Source Code is free software: you can redistribute it and/or modify it under the terms of the GNU General Public License as published by the Free Software Foundation, either version 3 of the License, or (at your option) any later version.
% Mahāvīrī LaTeX Source Code is distributed in the hope that it will be useful, but WITHOUT ANY WARRANTY; without even the implied warranty of MERCHANTABILITY or FITNESS FOR A PARTICULAR PURPOSE. See the GNU General Public License for more details.
% You should have received a copy of the GNU General Public License along with Mahāvīrī LaTeX Source Code. If not, see <https://www.gnu.org/licenses/>.
\fancyhead[RE,LO]{{\textmd{\large \textit{महावीरी} व्याख्या}}}
\fancyhead[LE,RO]{{\textmd{\large मङ्गलाचरणम्}}}
\phantomsection
\addcontentsline{toc}{section}{व्याख्याकारका मङ्गलाचरण}

\centering{॥ मङ्गलाचरणम् ॥\\}

{\bfseries
\setlength{\mylenone}{0pt}
\settowidth{\mylentwo}{तापिच्छनीलं धृतदिव्यशीलं ब्रह्माद्वयं व्यापकमव्ययञ्च}
\setlength{\mylenone}{\maxof{\mylenone}{\mylentwo}}
\settowidth{\mylentwo}{राजाधिराजं विशदं विराजं सीताभिरामं प्रणमामि रामम्}
\setlength{\mylenone}{\maxof{\mylenone}{\mylentwo}}
\setlength{\mylentwo}{\baselineskip}
\setlength{\mylenone}{\mylenone + 1pt}
\setlength{\mylen}{(\textwidth - \mylenone)*\real{0.5}}
\begin{longtable}[l]{@{\hspace*{\mylen}}>{\setlength\parfillskip{0pt}}p{\mylenone}@{}@{}l@{}}
 & \\[-\the\mylentwo]
तापिच्छनीलं धृतदिव्यशीलं ब्रह्माद्वयं व्यापकमव्ययञ्च & ।\\ \nopagebreak
राजाधिराजं विशदं विराजं सीताभिरामं प्रणमामि रामम् & ॥\\
\end{longtable}
}

{\bfseries
\setlength{\mylenone}{0pt}
\settowidth{\mylentwo}{सीतावियोगानलवारिवाहः श्रीरामपादाब्जमिलिन्दवर्यः}
\setlength{\mylenone}{\maxof{\mylenone}{\mylentwo}}
\settowidth{\mylentwo}{दिव्याञ्जनाशुक्तिललामभूतः स मारुतिर्मङ्गलमातनोतु}
\setlength{\mylenone}{\maxof{\mylenone}{\mylentwo}}
\setlength{\mylentwo}{\baselineskip}
\setlength{\mylenone}{\mylenone + 1pt}
\setlength{\mylen}{(\textwidth - \mylenone)*\real{0.5}}
\begin{longtable}[l]{@{\hspace*{\mylen}}>{\setlength\parfillskip{0pt}}p{\mylenone}@{}@{}l@{}}
 & \\[-\the\mylentwo]
सीतावियोगानलवारिवाहः श्रीरामपादाब्जमिलिन्दवर्यः & ।\\
दिव्याञ्जनाशुक्तिललामभूतः स मारुतिर्मङ्गलमातनोतु & ॥\\
\end{longtable}
}

{\bfseries
\setlength{\mylenone}{0pt}
\settowidth{\mylentwo}{गुरून्नत्वा सीतापतिचरणपाथोजयुगलं}
\setlength{\mylenone}{\maxof{\mylenone}{\mylentwo}}
\settowidth{\mylentwo}{चिरञ्चित्ते ध्यात्वा पवनतनयं भक्तसुखदम्}
\setlength{\mylenone}{\maxof{\mylenone}{\mylentwo}}
\settowidth{\mylentwo}{गिरं स्वीयां दुष्टां विमलयितुमेवार्यचरितै-}
\setlength{\mylenone}{\maxof{\mylenone}{\mylentwo}}
\settowidth{\mylentwo}{र्महावीरीव्याख्यां विरचयति बालो गिरिधरः}
\setlength{\mylenone}{\maxof{\mylenone}{\mylentwo}}
\setlength{\mylentwo}{\baselineskip}
\setlength{\mylenone}{\mylenone + 1pt}
\setlength{\mylen}{(\textwidth - \mylenone)*\real{0.5}}
\begin{longtable}[l]{@{\hspace*{\mylen}}>{\setlength\parfillskip{0pt}}p{\mylenone}@{}@{}l@{}}
 & \\[-\the\mylentwo]
गुरून्नत्वा सीतापतिचरणपाथोजयुगलं & \\ \nopagebreak
चिरञ्चित्ते ध्यात्वा पवनतनयं भक्तसुखदम् & ।\\
गिरं स्वीयां दुष्टां विमलयितुमेवार्यचरितै- & \\ \nopagebreak
र्महावीरीव्याख्यां विरचयति बालो गिरिधरः & ॥\\
\end{longtable}
}

{\bfseries
\setlength{\mylenone}{0pt}
\settowidth{\mylentwo}{श्रीगुरुदेव गजानन मारुति-आरति-नाशिनि गौरि गिरीशा}
\setlength{\mylenone}{\maxof{\mylenone}{\mylentwo}}
\settowidth{\mylentwo}{जानकि-जीवन मारुतनंदन पंकज पायन नाइके शीशा}
\setlength{\mylenone}{\maxof{\mylenone}{\mylentwo}}
\settowidth{\mylentwo}{माधव शुक्ल शुभा परिवा तिथि भार्गववार प्रभातगवीशा}
\setlength{\mylenone}{\maxof{\mylenone}{\mylentwo}}
\settowidth{\mylentwo}{संबत बीस-शताधिक-चालिस व्याख्या करी हनुमान-चलीसा}
\setlength{\mylenone}{\maxof{\mylenone}{\mylentwo}}
\setlength{\mylentwo}{\baselineskip}
\setlength{\mylenone}{\mylenone + 1pt}
\setlength{\mylen}{(\textwidth - \mylenone)*\real{0.5}}
\begin{longtable}[l]{@{\hspace*{\mylen}}>{\setlength\parfillskip{0pt}}p{\mylenone}@{}@{}l@{}}
 & \\[-\the\mylentwo]
श्रीगुरुदेव गजानन मारुति-आरति-नाशिनि गौरि गिरीशा & ।\\
जानकि-जीवन मारुतनंदन पंकज पायन नाइके शीशा & ।\\
माधव शुक्ल शुभा परिवा तिथि भार्गववार प्रभातगवीशा & ।\\
संबत बीस-शताधिक-चालिस व्याख्या करी हनुमान-चलीसा & ॥\\
\end{longtable}
}
{\bfseries
\setlength{\mylenone}{0pt}
\settowidth{\mylentwo}{अतुलितबलधामं स्वर्णशैलाभदेहं}
\setlength{\mylenone}{\maxof{\mylenone}{\mylentwo}}
\settowidth{\mylentwo}{दनुजवनकृशानुं ज्ञानिनामग्रगण्यम्}
\setlength{\mylenone}{\maxof{\mylenone}{\mylentwo}}
\settowidth{\mylentwo}{सकलगुणनिधानं वानराणामधीशं}
\setlength{\mylenone}{\maxof{\mylenone}{\mylentwo}}
\settowidth{\mylentwo}{रघुपतिवरदूतं वातजातं नमामि}
\setlength{\mylenone}{\maxof{\mylenone}{\mylentwo}}
\setlength{\mylentwo}{\baselineskip}
\setlength{\mylenone}{\mylenone + 1pt}
\setlength{\mylen}{(\textwidth - \mylenone)*\real{0.5}}
\begin{longtable}[l]{@{\hspace*{\mylen}}>{\setlength\parfillskip{0pt}}p{\mylenone}@{}@{}l@{}}
 & \\[-\the\mylentwo]
अतुलितबलधामं स्वर्णशैलाभदेहं & \\ \nopagebreak
दनुजवनकृशानुं ज्ञानिनामग्रगण्यम् & ।\\
सकलगुणनिधानं वानराणामधीशं & \\ \nopagebreak
रघुपतिवरदूतं वातजातं नमामि & ॥\\ \nopagebreak
\caption*{—रा.च.मा. ५-मङ्गलाचरण श्लोक ३}
\end{longtable}
}
\paraseplotus
\pagebreak



\fancyhead[LE,RO]{{\textmd{\large दो. १: श्रीगुरु-चरन-सरोज-रज}}}
\phantomsection
\addcontentsline{toc}{section}{मङ्गलाचरण दोहा १: श्रीगुरु-चरन-सरोज-रज}
\centering{॥ श्रीराम ॥}
\begin{sloppypar}\justifying\hyphenrules{nohyphenation}
मूल (दोहा)—
\end{sloppypar}

{\bfseries\relscale{1.2}
\setlength{\mylenone}{0pt}
\settowidth{\mylentwo}{श्रीगुरु-चरन-सरोज-रज निज-मन-मुकुर सुधारि}
\setlength{\mylenone}{\maxof{\mylenone}{\mylentwo}}
\settowidth{\mylentwo}{बरनउँ रघुबर-बिमल-जस जो दायक फल चारि}
\setlength{\mylenone}{\maxof{\mylenone}{\mylentwo}}
\setlength{\mylentwo}{\baselineskip}
\setlength{\mylenone}{\mylenone + 1pt}
\setlength{\mylen}{(\textwidth - \mylenone)*\real{0.5}}
\begin{longtable}[l]{@{\hspace*{\mylen}}>{\setlength\parfillskip{0pt}}p{\mylenone}@{}@{}l@{}}
 & \\[-\the\mylentwo]
श्रीगुरु-चरन-सरोज-रज निज-मन-मुकुर सुधारि & ।\\ \nopagebreak[1mm]
बरनउँ रघुबर-बिमल-जस जो दायक फल चारि & ॥
\end{longtable}
}

\parasepone
\begin{sloppypar}\justifying\hyphenrules{nohyphenation}
\textbf{शब्दार्थ}—\textbf{\textit{मुकुर}} {\unifont{\relscale{0.7}▶}} दर्पण।
\end{sloppypar}
\begin{sloppypar}\justifying\hyphenrules{nohyphenation}
\textbf{अर्थ}—श्रीगुरुदेवजीके श्रीचरणकमलकी पराग-रूप धूलिसे अपने मन-रूप दर्पणको स्वच्छ करके रघुकुलमें श्रेष्ठ श्रीरामभद्रजूके निर्मल यशका वर्णन कर रहा हूँ, जो चारों फलोंको देने वाला है।
\end{sloppypar}
\parasepone
\index[ardha]{श्रीगुरुचरनसरोजरजनिजमनमुकुरसुधारि(म.दो.१,पूर्वार्ध)@श्रीगुरु-चरन-सरोज-रज निज-मन-मुकुर सुधारि (म.दो. १, पूर्वार्ध)}
\index[ardha]{बरनउँरघुबरबिमलजसजोदायकफलचारि(म.दो.१,उत्तरार्ध)@बरनउँ रघुबर-बिमल-जस जो दायक फल चारि (म.दो. १, उत्तरार्ध)}
\index[pada]{श्रीगुरु@श्रीगुरु}
\index[pada]{चरन@चरन}
\index[pada]{सरोज@सरोज}
\index[pada]{रज@रज}
\index[pada]{निज@निज}
\index[pada]{मन@मन}
\index[pada]{मुकुर@मुकुर}
\index[pada]{सुधारि@सुधारि}
\index[pada]{बरनउँ@बरनउँ}
\index[pada]{रघुबर@रघुबर}
\index[pada]{बिमल@बिमल}
\index[pada]{जस@जस}
\index[pada]{जो@जो}
\index[pada]{दायक@दायक}
\index[pada]{फल@फल}
\index[pada]{चारि@चारि}
\begin{sloppypar}\justifying\hyphenrules{nohyphenation}
\textbf{व्याख्या}—सनातन धर्मके अलंकारभूत परम पावन स्तोत्ररत्न \textit{श्रीहनुमान्‌-चालीसा}की रचनाका प्रारम्भ करते हुए कलिपावनावतार, निखिल-वैष्णवकुल-शेखर, सारस्वत-सार्वभौम, परम रामभक्त, प्रातःस्मरणीय, कविकुल\-तिलक, पूज्य श्रीगोस्वामी तुलसीदासजी महाराज प्रतिज्ञा-वाक्यमें सर्वप्रथम \textbf{\textit{श्री}}पदके प्रयोगसे श्रीजीका स्मरण कर रहे हैं, जो समस्त मङ्गलोंकी खान हैं।
\end{sloppypar}
{\bfseries
\setlength{\mylenone}{0pt}
\settowidth{\mylentwo}{बाम भाग शोभति अनुकूला}
\setlength{\mylenone}{\maxof{\mylenone}{\mylentwo}}
\settowidth{\mylentwo}{आदिशक्ति छबिनिधि जगमूला}
\setlength{\mylenone}{\maxof{\mylenone}{\mylentwo}}
\setlength{\mylentwo}{\baselineskip}
\setlength{\mylenone}{\mylenone + 1pt}
\setlength{\mylen}{(\textwidth - \mylenone)*\real{0.5}}
\begin{longtable}[l]{@{\hspace*{\mylen}}>{\setlength\parfillskip{0pt}}p{\mylenone}@{}@{}l@{}}
 & \\[-\the\mylentwo]
बाम भाग शोभति अनुकूला & ।\\ \nopagebreak
आदिशक्ति छबिनिधि जगमूला & ॥\\ \nopagebreak
\caption*{—रा.च.मा. १-१४८-२}
\end{longtable}
}
\begin{sloppypar}\justifying\hyphenrules{nohyphenation}
\noindent ये ही श्री जनक महाराजके यशोवर्धन हेतु श्रीमिथिला-भूमिमें प्रकट होती हैं तथा श्रीसीता रूपसे श्रीरामभद्रजूके वाम भागमें विराजमान होकर जीवके भगवत्प्रातिकूल्यको निरस्त करती हैं। \textbf{\textit{श्री}} शब्दका \textbf{\textit{गुरु}} शब्दसे दो प्रकारका समास है—
\end{sloppypar}
\begin{sloppypar}\justifying\hyphenrules{nohyphenation}
(१) \textit{मध्यमपदलोपी तृतीयातत्पुरुष समास}—\textit{श्रिया अनुगृहीतो गुरुः इति श्रीगुरुः}। अर्थात् श्रीजीके द्वारा अनुगृहीत गुरुदेव। अभिप्राय यह है कि श्रीसंप्रदायमें दीक्षित गुरुदेवकी ही चरण-धूलिसे मनकी निर्मलता संभव है। क्योंकि श्रीजीकी कृपाके बिना अविद्याकृत दोष नष्ट नहीं होते। तात्पर्य यह है कि श्रीजीको गोस्वामीजीने भगवदभिन्न होनेपर भी भक्तिरूपमें स्वीकारा है। यथा—
\end{sloppypar}
{\bfseries
\setlength{\mylenone}{0pt}
\settowidth{\mylentwo}{लसत मंजु मुनि-मंडली मध्य सीय रघुचंद}
\setlength{\mylenone}{\maxof{\mylenone}{\mylentwo}}
\settowidth{\mylentwo}{ग्यान-सभा जनु तनु धरे भगति सच्चिदानंद}
\setlength{\mylenone}{\maxof{\mylenone}{\mylentwo}}
\setlength{\mylentwo}{\baselineskip}
\setlength{\mylenone}{\mylenone + 1pt}
\setlength{\mylen}{(\textwidth - \mylenone)*\real{0.5}}
\begin{longtable}[l]{@{\hspace*{\mylen}}>{\setlength\parfillskip{0pt}}p{\mylenone}@{}@{}l@{}}
 & \\[-\the\mylentwo]
लसत मंजु मुनि-मंडली मध्य सीय रघुचंद & ।\\ \nopagebreak
ग्यान-सभा जनु तनु धरे भगति सच्चिदानंद & ॥\\ \nopagebreak
\caption*{—रा.च.मा. २-२३९}
\end{longtable}
}
\begin{sloppypar}\justifying\hyphenrules{nohyphenation}
(२) \textit{कर्मधारय समास}—\textit{श्रीरेव गुरुः इति श्रीगुरुः}। अर्थात् श्री ही गुरु हैं। श्रीसंप्रदायमें श्रीरामानुजाचार्यजी तथा श्रीरामानन्दाचार्यजीने श्रीजीको ही परम गुरु माना है। पूर्वाचार्योंने श्रीसीता भगवतीको श्रीहनुमान्‌जीकी आचार्याके रूपमें स्वीकारा है। यथा—\textbf{समस्तनिगमाचार्यं सीताशिष्यं गुरोर्गुरुम्}, अर्थात् श्रीहनुमान्‌जी श्रीसीताजीके शिष्य तथा देवगुरु बृहस्पतिजीके भी गुरु हैं। अतः श्रीहनुमान्‌जीकी संतुष्टिके लिए प्रणीत \textit{श्रीहनुमान्‌-चालीसा}के प्रारम्भमें उनकी आचार्या श्रीसीताजीका स्मरण अत्यन्त उपयोगी है, यही \textbf{\textit{श्रीगुरु}} शब्दका अभिप्राय प्रतीत होता है।
\end{sloppypar}
\begin{sloppypar}\justifying\hyphenrules{nohyphenation}
\textbf{\textit{रज}} शब्द यहाँ श्लेषके बलसे कमल-पक्षमें पराग एवं चरण-पक्षमें धूलि रूप अर्थका द्योतक है। मनको \textbf{\textit{मुकुर}} कहनेका अभिप्राय यह है कि जैसे दर्पणमें बिम्बका प्रतिबिम्बन होता है, उसी प्रकार मनमें श्रीभुवनमनोहर श्रीराघवके रूपका प्रतिबिम्बन होता है। पर वह मन विषय रूप काई (जलका मल)से मलिन हो चुका है, यथा—\textbf{काई बिषय मुकुर मन लागी} (रा.च.मा. १-११५-१)। अतः उसे श्रीगुरुदेवके चरण-कमलकी पराग जैसी मृदु धूलिसे स्वच्छ करके पुनः श्रीरामजीके यशोवर्णनकी प्रतिज्ञा करते हैं, जिससे स्वच्छ मनोदर्पणमें भली-भाँति उस यशश्चन्द्रका प्रतिबिम्बन हो सके।
\end{sloppypar}
\begin{sloppypar}\justifying\hyphenrules{nohyphenation}
\textit{श्रीहनुमान्‌-चालीसा}के प्रारम्भमें \textbf{\textit{रघुबर-बिमल-जस बरनउँ}} यह वाक्यखण्ड एक जिज्ञासाका केन्द्र बन जाता है, तथा कुछ सामान्य मस्तिष्क वालोंको असंगत प्रतीत होता है। पर विचार करने पर इसका सुगमतया समाधान हो जाता है। श्रीहनुमान्‌जी महाराज श्रीरामभद्रजूके सर्वतोभावेन समर्पित भक्तोंमें अग्रणी हैं। श्रीरघुनाथजीके अतिरिक्त वे अपना किञ्चित् भी अस्तित्व मानने को तैयार नहीं हैं। यथा—
\end{sloppypar}
{\bfseries
\setlength{\mylenone}{0pt}
\settowidth{\mylentwo}{ता पर मैं रघुबीर दोहाई}
\setlength{\mylenone}{\maxof{\mylenone}{\mylentwo}}
\settowidth{\mylentwo}{जानउँ नहिं कछु भजन उपाई}
\setlength{\mylenone}{\maxof{\mylenone}{\mylentwo}}
\setlength{\mylentwo}{\baselineskip}
\setlength{\mylenone}{\mylenone + 1pt}
\setlength{\mylen}{(\textwidth - \mylenone)*\real{0.5}}
\begin{longtable}[l]{@{\hspace*{\mylen}}>{\setlength\parfillskip{0pt}}p{\mylenone}@{}@{}l@{}}
 & \\[-\the\mylentwo]
ता पर मैं रघुबीर दोहाई & ।\\ \nopagebreak
जानउँ नहिं कछु भजन उपाई & ॥\\ \nopagebreak
\caption*{—रा.च.मा. ४-३-३}
\end{longtable}
}
\begin{sloppypar}\justifying\hyphenrules{nohyphenation}
\noindent अतः श्रीरघुवर-यशोवर्णनमें ही उनके यशका वर्णन गतार्थ हो जाता है। दूसरी बात यह भी है कि वैष्णव भक्तोंको अपनी प्रशंसा नहीं भाती। अतः रघुवर-यशोवर्णनसे ही श्रीहनुमान्‌जीकी प्रसन्नता संभव है। इसी उद्देश्यको ध्यानमें रखकर श्रीगोस्वामीजीने अभिधावृत्तिसे श्रीरामजीके यशका वर्णन कर \textit{श्रीहनुमान्‌-चालीसा}से श्रीमारुतिको प्रसन्न किया तथा रघुवर-यशोभङ्गिमासे लक्षणावृत्ति द्वारा श्रीहनुमत्-यशोगान कर इस \textit{हनुमान्‌-चालीसा} स्तोत्रको श्रीराघवकी प्रसन्नताका केन्द्र बना दिया। अतः \textbf{\textit{रघुबर-बिमल-जस बरनउँ}} से उपक्रम करके \textbf{\textit{राम लखन सीता सहित हृदय बसहु सुर-भूप}} से उपसंहार करेंगे। यह रामयश अर्थ, धर्म, काम, मोक्ष—इन चारों फलोंका प्रदाता है। भाव यह है कि इससे प्रसन्न होकर हनुमान्‌जी महाराज \textit{श्रीहनुमान्‌-चालीसा}के पाठकको पुरुषार्थ-चतुष्टय दे डालते हैं। यद्वा सालोक्य, सामीप्य, सायुज्य, सारूप्य—इन चारों मुक्तिफलोंको देते हैं। अथवा धर्म, ज्ञान, योग, जप—इन चारों फलोंको देते हैं। किंवा ज्ञानवादियोंको साधन-चतुष्टयसे संपन्न कर देते हैं।
\end{sloppypar}
\paraseplotus
\pagebreak


\fancyhead[LE,RO]{{\textmd{\large दो. २: बुद्धि-हीन तनु जानिकै}}}
\phantomsection
\addcontentsline{toc}{section}{मङ्गलाचरण दोहा २: बुद्धि-हीन तनु जानिकै}
\centering{॥ श्रीराम ॥}
\begin{sloppypar}\justifying\hyphenrules{nohyphenation}
मूल (दोहा)—
\end{sloppypar}

{\bfseries\relscale{1.2}
\setlength{\mylenone}{0pt}
\settowidth{\mylentwo}{बुद्धि-हीन तनु जानिकै सुमिरौं पवनकुमार}
\setlength{\mylenone}{\maxof{\mylenone}{\mylentwo}}
\settowidth{\mylentwo}{बल बुधि बिद्या देहु मोहिं हरहु कलेश बिकार}
\setlength{\mylenone}{\maxof{\mylenone}{\mylentwo}}
\setlength{\mylentwo}{\baselineskip}
\setlength{\mylenone}{\mylenone + 1pt}
\setlength{\mylen}{(\textwidth - \mylenone)*\real{0.5}}
\begin{longtable}[l]{@{\hspace*{\mylen}}>{\setlength\parfillskip{0pt}}p{\mylenone}@{}@{}l@{}}
 & \\[-\the\mylentwo]
बुद्धि-हीन तनु जानिकै सुमिरौं पवनकुमार & ।\\ \nopagebreak[1mm]
बल बुधि बिद्या देहु मोहिं हरहु कलेश बिकार & ॥
\end{longtable}
}

\parasepone
\index[ardha]{बुद्धिहीनतनुजानिकैसुमिरौंपवनकुमार(म.दो.२,पूर्वार्ध)@बुद्धि-हीन तनु जानिकै सुमिरौं पवनकुमार (म.दो. २, पूर्वार्ध)}
\index[ardha]{बलबुधिबिद्यादेहुमोहिंहरहुकलेशबिकार(म.दो.२,उत्तरार्ध)@बल बुधि बिद्या देहु मोहिं हरहु कलेश बिकार (म.दो. २, उत्तरार्ध)}
\index[pada]{बुद्धि@बुद्धि}
\index[pada]{हीन@हीन}
\index[pada]{तनु@तनु}
\index[pada]{जानिकै@जानिकै}
\index[pada]{सुमिरौं@सुमिरौं}
\index[pada]{पवनकुमार@पवनकुमार}
\index[pada]{बल@बल}
\index[pada]{बुधि@बुधि}
\index[pada]{बिद्या@बिद्या}
\index[pada]{देहु@देहु}
\index[pada]{मोहिं@मोहिं}
\index[pada]{हरहु@हरहु}
\index[pada]{कलेश@कलेश}
\index[pada]{बिकार@बिकार}
\begin{sloppypar}\justifying\hyphenrules{nohyphenation}
\textbf{शब्दार्थ}—\textbf{\textit{बिकार}} {\unifont{\relscale{0.7}▶}} दोष।
\end{sloppypar}
\begin{sloppypar}\justifying\hyphenrules{nohyphenation}
\textbf{अर्थ}—अपने शरीरको बुद्धिसे हीन जानकर मैं श्रीपवनपुत्र हनुमान्‌जीका स्मरण कर रहा हूँ। हे प्रभो! आप मुझे बल, बुद्धि, तथा विद्या प्रदान करें तथा क्लेश एवं विकारोंको समाप्त कर दें।
\end{sloppypar}
\parasepone
\begin{sloppypar}\justifying\hyphenrules{nohyphenation}
\textbf{व्याख्या}—यहाँ \textbf{\textit{बुद्धि}} शब्द भगवत्सेवोपयोगिनी बुद्धिका वाचक है तथा \textbf{\textit{तनु}} सूक्ष्म शरीर का, क्योंकि बुद्धिको सूक्ष्म शरीरका अवयव माना गया है। अर्थात् मेरी बुद्धि तमोगुणके आधिक्यसे भगवान्‌के श्रीचरण-कमलोंसे विमुख हो गई है, अतः पवनपुत्रका स्मरण करता हूँ। \textbf{\textit{पवन}} शब्दका अर्थ है पवित्र करने वाला। यथा—\textit{पुनाति इति पवनः}। आप उनके पुत्र अर्थात् अग्नि हैं, यथा—\textbf{वायोरग्निः} (तै.उ. २-१-१)। इसलिए अग्निवत् बुद्धिमें परम प्रकाशका आधान करके क्लेश आदि मलोंको ध्वस्त कर दें।
\end{sloppypar}
\begin{sloppypar}\justifying\hyphenrules{nohyphenation}
अब गोस्वामीजी हनुमान्‌जीसे तीन वस्तुओंकी याचना करते हैं—
\end{sloppypar}
\begin{sloppypar}\justifying\hyphenrules{nohyphenation}
(१) \textbf{\textit{बल}} शब्द यहाँ काम-राग-विवर्जित-आत्मबल-परक है। यथा—\textbf{बलं बलवतां चाहं कामरागविवर्जितम्} (भ.गी. ७-११)। यही आत्मबल भगवत्प्राप्तिमें साधन है, यथा—\textbf{नायमात्मा बलहीनेन लभ्यः} (मु.उ. ३-२-४)।
\end{sloppypar}
\begin{sloppypar}\justifying\hyphenrules{nohyphenation}
(२) \textbf{\textit{बुधि}} (संस्कृत: \textit{बुद्धि}) शब्दसे यहाँ ईश्वर\-प्रपन्न बुद्धि अभिप्रेत है, यथा—\textbf{चरन-सरोरुह नाथ जनि कबहुँ तजै मति मोरि} (रा.च.मा. ३-४)।
\end{sloppypar}
\begin{sloppypar}\justifying\hyphenrules{nohyphenation}
(३) \textbf{\textit{बिद्या}} (संस्कृत:\textit{विद्या})—यहाँ विद्या विनय\-संपन्ना अपेक्षित है, जो भगवत्संबन्धका विवेक उत्पन्न करके जीवको राघवके चरण-कमलसे जोड़ दे। यथा—\textbf{सा विद्या या विमुक्तये} (वि.पु. १-१९-४१)। अपि च—
\end{sloppypar}
{\bfseries
\setlength{\mylenone}{0pt}
\settowidth{\mylentwo}{बिद्या बिनु बिबेक उपजाए}
\setlength{\mylenone}{\maxof{\mylenone}{\mylentwo}}
\settowidth{\mylentwo}{श्रम फल पढ़े किए अरु पाए}
\setlength{\mylenone}{\maxof{\mylenone}{\mylentwo}}
\setlength{\mylentwo}{\baselineskip}
\setlength{\mylenone}{\mylenone + 1pt}
\setlength{\mylen}{(\textwidth - \mylenone)*\real{0.5}}
\begin{longtable}[l]{@{\hspace*{\mylen}}>{\setlength\parfillskip{0pt}}p{\mylenone}@{}@{}l@{}}
 & \\[-\the\mylentwo]
बिद्या बिनु बिबेक उपजाए & ।\\ \nopagebreak
श्रम फल पढ़े किए अरु पाए & ॥\\ \nopagebreak
\caption*{—रा.च.मा. ३-२१-९}
\end{longtable}
}
\begin{sloppypar}\justifying\hyphenrules{nohyphenation}
\noindent अर्थात् श्रीआञ्जनेय बल, बुद्धि, एवं अध्यात्म-विद्यासे भगवान्‌के सौन्दर्य, ऐश्वर्य, एवं माधुर्यकी अनुभूतिका सामर्थ्य दें।
\end{sloppypar}
\begin{sloppypar}\justifying\hyphenrules{nohyphenation}
क्लेश पाँच होते हैं—अविद्या, अस्मिता, राग, द्वेष, और अभिनिवेश (मरण)। यथा—\textbf{अविद्यास्मितारागद्वेषाभिनिवेशाः पञ्च क्लेशाः} (यो.सू. २-३)। विकार छः कहे जाते हैं—काम, क्रोध, लोभ, मोह, मद, एवं मात्सर्य। यथा—\textbf{षट-बिकार-जित अनघ अकामा} (रा.च.मा. ३-४७-७)। इस प्रकार पञ्च क्लेशों और षट् विकारोंका योग ग्यारह (११) हुआ और आप एकादशरुद्रमय हैं। यथा—\textbf{रुद्र-अवतार संसार-पाता} (वि.प. २५-३)। अतः मेरे इन एकादश शत्रुओंको समाप्त करें।
\end{sloppypar}
\paraseplotus
\pagebreak


\fancyhead[LE,RO]{{\textmd{\large चौ. १: जय हनुमान ज्ञान-गुण-सागर}}}
\phantomsection
\addcontentsline{toc}{section}{चौपाई १: जय हनुमान ज्ञान-गुण-सागर}
\centering{॥ श्रीराम ॥}
\begin{sloppypar}\justifying\hyphenrules{nohyphenation}
मूल (चौपाई)—
\end{sloppypar}

{\bfseries\relscale{1.2}
\setlength{\mylenone}{0pt}
\settowidth{\mylentwo}{जय हनुमान ज्ञान-गुण-सागर}
\setlength{\mylenone}{\maxof{\mylenone}{\mylentwo}}
\settowidth{\mylentwo}{जय कपीश तिहुँ लोक उजागर}
\setlength{\mylenone}{\maxof{\mylenone}{\mylentwo}}
\setlength{\mylentwo}{\baselineskip}
\setlength{\mylenone}{\mylenone + 1pt}
\setlength{\mylen}{(\textwidth - \mylenone)*\real{0.5}}
\begin{longtable}[l]{@{\hspace*{\mylen}}>{\setlength\parfillskip{0pt}}p{\mylenone}@{}@{}l@{}}
 & \\[-\the\mylentwo]
जय हनुमान ज्ञान-गुण-सागर & ।\\ \nopagebreak[1mm]
जय कपीश तिहुँ लोक उजागर & ॥ १ ॥
\end{longtable}
}

\parasepone
\index[ardha]{जयहनुमानज्ञानगुणसागर(चौ.१,पूर्वार्ध)@जय हनुमान ज्ञान-गुण-सागर (चौ. १, पूर्वार्ध)}
\index[ardha]{जयकपीशतिहुँलोकउजागर(चौ.१,उत्तरार्ध)@जय कपीश तिहुँ लोक उजागर (चौ. १, उत्तरार्ध)}
\index[pada]{जय@जय}
\index[pada]{हनुमान@हनुमान}
\index[pada]{ज्ञान@ज्ञान}
\index[pada]{गुण@गुण}
\index[pada]{सागर@सागर}
\index[pada]{जय@जय}
\index[pada]{कपीश@कपीश}
\index[pada]{तिहुँ@तिहुँ}
\index[pada]{लोक@लोक}
\index[pada]{उजागर@उजागर}
\begin{sloppypar}\justifying\hyphenrules{nohyphenation}
\textbf{शब्दार्थ}—\textbf{\textit{उजागर}} (\textit{उज्जागर}) {\unifont{\relscale{0.7}▶}} प्रसिद्ध।
\end{sloppypar}
\begin{sloppypar}\justifying\hyphenrules{nohyphenation}
\textbf{अर्थ}—समस्त शास्त्रीय ज्ञान एवं गुणोंके समुद्र श्रीहनुमान्‌जी! आपकी जय हो! हे तीनों लोकोंमें प्रसिद्ध वानरोंमें श्रेष्ठ आञ्जनेय! आपकी जय हो!
\end{sloppypar}
\parasepone
\begin{sloppypar}\justifying\hyphenrules{nohyphenation}
\textbf{व्याख्या}—इस चौपाईके पूर्वार्धमें श्रीहनुमान्‌जीके पारलौकिक उत्कर्ष तथा उत्तरार्धमें लौकिक आदर्शका वर्णन करते हैं। श्रीहनुमान्‌जी समस्त शास्त्रोंके ज्ञाता हैं। \textit{श्रीमद्वाल्मीकीय-रामायण}के किष्किन्धा\-काण्डमें भगवान् श्रीराम इनके अलौकिक ज्ञानकी प्रशंसा करते हैं—
\end{sloppypar}
{\bfseries
\setlength{\mylenone}{0pt}
\settowidth{\mylentwo}{नानृग्वेदविनीतस्य नायजुर्वेदधारिणः}
\setlength{\mylenone}{\maxof{\mylenone}{\mylentwo}}
\settowidth{\mylentwo}{नासामवेदविदुषः शक्यमेवं प्रभाषितुम्}
\setlength{\mylenone}{\maxof{\mylenone}{\mylentwo}}
\settowidth{\mylentwo}{नूनं व्याकरणं कृत्स्नमनेन विधिना श्रुतम्}
\setlength{\mylenone}{\maxof{\mylenone}{\mylentwo}}
\settowidth{\mylentwo}{बहुव्याहरताऽनेन न किञ्चिदपशब्दितम्}
\setlength{\mylenone}{\maxof{\mylenone}{\mylentwo}}
\setlength{\mylentwo}{\baselineskip}
\setlength{\mylenone}{\mylenone + 1pt}
\setlength{\mylen}{(\textwidth - \mylenone)*\real{0.5}}
\begin{longtable}[l]{@{\hspace*{\mylen}}>{\setlength\parfillskip{0pt}}p{\mylenone}@{}@{}l@{}}
 & \\[-\the\mylentwo]
नानृग्वेदविनीतस्य नायजुर्वेदधारिणः & ।\\ \nopagebreak
नासामवेदविदुषः शक्यमेवं प्रभाषितुम् & ॥\\
नूनं व्याकरणं कृत्स्नमनेन विधिना श्रुतम् & ।\\ \nopagebreak
बहुव्याहरताऽनेन न किञ्चिदपशब्दितम् & ॥\\ \nopagebreak
\caption*{—वा.रा. ४-३-२८,२९}
\end{longtable}
}
\begin{sloppypar}\justifying\hyphenrules{nohyphenation}
\noindent श्रीरामचन्द्रजी प्रशंसाके स्वरमें कहते हैं, “जिसने ऋग्वेदकी पूर्ण शिक्षा नहीं पाई तथा जिसने यजुर्वेदको अर्थतः धारण नहीं किया तथा जो सामवेदका विद्वान् नहीं, वह इस प्रकारका भाषण कभी भी नहीं कर सकता। निश्चित ही संपूर्ण व्याकरण इसने विधिवत् सुना है क्योंकि धाराप्रवाहसे बोलता हुआ यह विद्यार्थी कहीं भी एक भी अक्षर अशुद्ध नहीं बोला।”
\end{sloppypar}
\begin{sloppypar}\justifying\hyphenrules{nohyphenation}
\textbf{\textit{तिहुँ लोक उजागर}}—हनुमान्‌जी रुद्रावतार होनेसे तथा जल, थल, एवं नभमें अव्याहतगति होनेके कारण तीनों लोकोंमें प्रसिद्ध हैं।
\end{sloppypar}
\begin{sloppypar}\justifying\hyphenrules{nohyphenation}
\textbf{\textit{जय कपीश}}—हनुमान्‌जी कपियोंके ईश्वर हैं। यथा—\allowbreak\textbf{वानराणामधीशम्} (रा.च.मा. सुन्दरकाण्ड, मङ्गलाचरण श्लोक~३)। इन्होंने श्रीरामजीकी सेवाके लिए वानर-शरीर धारण किया, क्योंकि इनके स्वामी नरवेषमें अवतार लिए—\textbf{तुमहिं लागि धरिहउँ नरबेसा} (रा.च.मा. १-१८७-१)। स्वामीसे सेवकको निम्न कक्षामें होना चाहिए। यथा—
\end{sloppypar}
{\bfseries
\setlength{\mylenone}{0pt}
\settowidth{\mylentwo}{जेहि शरीर रति राम सों सोइ आदरहिं सुजान}
\setlength{\mylenone}{\maxof{\mylenone}{\mylentwo}}
\settowidth{\mylentwo}{रुद्र-देह तजि नेह-बश बानर भे हनुमान}
\setlength{\mylenone}{\maxof{\mylenone}{\mylentwo}}
\setlength{\mylentwo}{\baselineskip}
\setlength{\mylenone}{\mylenone + 1pt}
\setlength{\mylen}{(\textwidth - \mylenone)*\real{0.5}}
\begin{longtable}[l]{@{\hspace*{\mylen}}>{\setlength\parfillskip{0pt}}p{\mylenone}@{}@{}l@{}}
 & \\[-\the\mylentwo]
जेहि शरीर रति राम सों सोइ आदरहिं सुजान & ।\\ \nopagebreak
रुद्र-देह तजि नेह-बश बानर भे हनुमान & ॥\\ \nopagebreak
\caption*{—दो. १४२}
\end{longtable}
}
\begin{sloppypar}\justifying\hyphenrules{nohyphenation}
\noindent यद्यपि और देवोंको ब्रह्माजीने वानर रूपमें अवतार लेनेका आदेश दिया था। यथा—
\end{sloppypar}
{\bfseries
\setlength{\mylenone}{0pt}
\settowidth{\mylentwo}{निज लोकहिं बिरंचि गे देवन इहइ सिखाइ}
\setlength{\mylenone}{\maxof{\mylenone}{\mylentwo}}
\settowidth{\mylentwo}{बानर-तनु धरि धरनि महँ हरि-पद सेवहु जाइ}
\setlength{\mylenone}{\maxof{\mylenone}{\mylentwo}}
\setlength{\mylentwo}{\baselineskip}
\setlength{\mylenone}{\mylenone + 1pt}
\setlength{\mylen}{(\textwidth - \mylenone)*\real{0.5}}
\begin{longtable}[l]{@{\hspace*{\mylen}}>{\setlength\parfillskip{0pt}}p{\mylenone}@{}@{}l@{}}
 & \\[-\the\mylentwo]
निज लोकहिं बिरंचि गे देवन इहइ सिखाइ & ।\\ \nopagebreak
बानर-तनु धरि धरनि महँ हरि-पद सेवहु जाइ & ॥\\ \nopagebreak
\caption*{—रा.च.मा. १-१८७}
\end{longtable}
}
\begin{sloppypar}\justifying\hyphenrules{nohyphenation}
\noindent पर महादेवको आदेश नहीं दिया था, अतः \textbf{देवन इहइ सिखाइ} कहा, न तु \textbf{महादेवहि सिखाइ}। वानर-शरीर धारण करनेमें दूसरा हेतु यह भी है कि वानर शुद्ध शाकाहारी होता है; अर्थात् वन्य फल, मूल, पत्तोंसे ही अपनी जीविका चलाता है, जो श्रीराघवको बहुत प्रिय हैं। यथा—\textbf{फलमूलाशिनौ दान्तौ} (रा.र.स्तो. १८), \textbf{शाकप्रियः पार्थिवः शाकपार्थिवः} (का.वा. २-१-६९), और \textbf{न मांसं राघवो भुङ्क्ते} (वा.रा. ५-३६-४१), अर्थात् राघव मांस कभी नहीं खाते हैं। अतः प्रभुकी वृत्तिके अनुसार श्रीहनुमान्‌जीने विशुद्ध शाकाहारी वानर-शरीर धारण किया। अतः हनुमान्‌जीके उपासक भावुक भक्तोंको कभी भी मांस, मत्स्य, तथा मद्यका सेवन नहीं करना चाहिए। मांसाहारी उपासक निश्चित ही हनुमान्‌जीके कोपका भाजन बनता है।
\end{sloppypar}

\paraseplotus
\pagebreak


\fancyhead[LE,RO]{{\textmd{\large चौ. २: राम-दूत अतुलित-बल-धामा}}}
\phantomsection
\addcontentsline{toc}{section}{चौपाई २: राम-दूत अतुलित-बल-धामा}
\centering{॥ श्रीराम ॥}
\begin{sloppypar}\justifying\hyphenrules{nohyphenation}
मूल (चौपाई)—
\end{sloppypar}

{\bfseries\relscale{1.2}
\setlength{\mylenone}{0pt}
\settowidth{\mylentwo}{राम-दूत अतुलित-बल-धामा}
\setlength{\mylenone}{\maxof{\mylenone}{\mylentwo}}
\settowidth{\mylentwo}{अंजनिपुत्र - पवनसुत - नामा}
\setlength{\mylenone}{\maxof{\mylenone}{\mylentwo}}
\setlength{\mylentwo}{\baselineskip}
\setlength{\mylenone}{\mylenone + 1pt}
\setlength{\mylen}{(\textwidth - \mylenone)*\real{0.5}}
\begin{longtable}[l]{@{\hspace*{\mylen}}>{\setlength\parfillskip{0pt}}p{\mylenone}@{}@{}l@{}}
 & \\[-\the\mylentwo]
राम-दूत अतुलित-बल-धामा & ।\\ \nopagebreak[1mm]
अंजनिपुत्र - पवनसुत - नामा & ॥ २ ॥
\end{longtable}
}

\parasepone
\index[ardha]{रामदूतअतुलितबलधामा(चौ.२,पूर्वार्ध)@राम-दूत अतुलित-बल-धामा (चौ. २, पूर्वार्ध)}
\index[ardha]{अंजनिपुत्रपवनसुतनामा(चौ.२,उत्तरार्ध)@अंजनिपुत्र-पवनसुत-नामा (चौ. २, उत्तरार्ध)}
\index[pada]{राम@राम}
\index[pada]{दूत@दूत}
\index[pada]{अतुलित@अतुलित}
\index[pada]{बल@बल}
\index[pada]{धामा@धामा}
\index[pada]{अंजनिपुत्र@अंजनिपुत्र}
\index[pada]{पवनसुत@पवनसुत}
\index[pada]{नामा@नामा}
\begin{sloppypar}\justifying\hyphenrules{nohyphenation}
\textbf{शब्दार्थ}—\textbf{\textit{अतुलित-बल-धामा}} {\unifont{\relscale{0.7}▶}} अतुलनीय बलके आश्रय।
\end{sloppypar}
\begin{sloppypar}\justifying\hyphenrules{nohyphenation}
\textbf{अर्थ}—आप श्रीरामजीके विश्वस्त दूत तथा अतुलनीय बलके आश्रय हैं तथा आप अञ्जनीपुत्र एवं पवनपुत्र नामसे प्रसिद्ध हैं।
\end{sloppypar}
\parasepone
\begin{sloppypar}\justifying\hyphenrules{nohyphenation}
\textbf{व्याख्या}—श्रीहनुमान्‌जी श्रीराघवके अन्तरङ्गतम दूत हैं। अतः श्रीसीताजीके प्रति गोपनीय संदेशवाहकका कार्य इन्हींको सौंपा गया। यथा—
\end{sloppypar}
{\bfseries
\setlength{\mylenone}{0pt}
\settowidth{\mylentwo}{बहु प्रकार सीतहिं समुझाएहु}
\setlength{\mylenone}{\maxof{\mylenone}{\mylentwo}}
\settowidth{\mylentwo}{कहि बल बीर बेगि तुम आएहु}
\setlength{\mylenone}{\maxof{\mylenone}{\mylentwo}}
\setlength{\mylentwo}{\baselineskip}
\setlength{\mylenone}{\mylenone + 1pt}
\setlength{\mylen}{(\textwidth - \mylenone)*\real{0.5}}
\begin{longtable}[l]{@{\hspace*{\mylen}}>{\setlength\parfillskip{0pt}}p{\mylenone}@{}@{}l@{}}
 & \\[-\the\mylentwo]
बहु प्रकार सीतहिं समुझाएहु & ।\\ \nopagebreak
कहि बल बीर बेगि तुम आएहु & ॥\\ \nopagebreak
\caption*{—रा.च.मा. ४-२३-११}
\end{longtable}
}
\begin{sloppypar}\justifying\hyphenrules{nohyphenation}
\noindent सुन्दरकाण्डमें ये श्रीसीताजीके समक्ष स्वयं कहते हैं—\textbf{रामदूत मैं मातु जानकी} (रा.च.मा. ५-१३-९), और प्रामाणिकताके लिए करुणा\-निधानकी शपथ करते हैं—\textbf{सत्य शपथ करुणा\-निधान~की} (रा.च.मा. ५-१३-९)। भाव यह है कि मुझमें दूत बननेकी कोई पात्रता न होनेपर भी करुणा\-निधानकी करुणाने यह महत्त्वपूर्ण पद दे दिया।
\end{sloppypar}
\begin{sloppypar}\justifying\hyphenrules{nohyphenation}
\textbf{\textit{अतुलित-बल-धामा}}—हनुमान्‌जी अतुलनीय बलके आश्रय हैं ही, यथा—\textbf{तेरे बल बानर जिताये रन रावन सों} (ह.बा. ३३)। अथवा अतुलित बलशाली भगवान् श्रीराम हैं, यथा—\textbf{अतुलित बल अतुलित प्रभुताई} (रा.च.मा. ३-२-१२)। उनके भी आश्रय हैं श्रीहनुमान्‌जी। यथा—\textbf{चलेउ हरषि हिय धरि रघुनाथा} (रा.च.मा. ५-१-४)।
\end{sloppypar}
\begin{sloppypar}\justifying\hyphenrules{nohyphenation}
\textbf{\textit{अंजनिपुत्र-पवनसुत-नामा}}—ये दोनों संबोधन हनुमान्‌जीकी मातृमत्ता एवं पितृमत्ताको सूचित करते हैं। अञ्जना—जो पहले पुञ्जिकस्थला नामक अप्सरा थीं—उन्हें अगस्त्यजीके शापसे वानर-शरीर प्राप्त हुआ। कामरूप-धारणका सामर्थ्य होनेसे कदाचित् अञ्जनाको दिव्यरूप-संपन्न देखकर वायुदेवने उनका मानसिक स्पर्श कर हनुमान्‌जी जैसे महापराक्रमी महापुरुषको उनके गर्भमन्दिरमें प्रतिष्ठित किया। वायुके रूपरहित होनेसे केसरी-पत्नीका न तो सतीत्व भङ्ग हुआ और न ही कोई साङ्कर्य दोष आया, क्योंकि वायु संपूर्ण प्राणियोंके अन्तःस्थ हैं तथा प्रत्येक वस्तुके त्याग एवं स्वीकारमें वे ही कारण हैं। वायुके बिना कभी भी न तो गर्भाधान संभव है और न ही बालकका जन्म—\textbf{प्रसव पवन प्रेरेउ अपराधी} (वि.प. १३६.५)। यहाँ यह भी ध्यान रहे कि रूपवान्‌का स्पर्श ही पदार्थमें विकृति लाता है किन्तु वायु रूपरहित स्पर्श वाला है, यथा—\textbf{रूपरहितस्पर्शवान् वायुः} (त.सं. १३)। वायु सामान्यतः प्रत्येक प्राणीके प्रत्येक अङ्गका स्पर्श करता है, अतः वायवीय स्पर्शमें दोष नहीं। वायु सबसे पवित्र है, यथा—\textbf{पवनः पवतामस्मि} (भ.गी. १०-३१)। अतः परम पवित्र पितासे जन्म होनेके कारण हनुमान्‌जी परम-पवित्रतम-व्यक्तित्व-संपन्न हुए।
\end{sloppypar}
\paraseplotus
\pagebreak


\fancyhead[LE,RO]{{\textmd{\large चौ. ३: महाबीर बिक्रम बजरंगी}}}
\phantomsection
\addcontentsline{toc}{section}{चौपाई ३: महाबीर बिक्रम बजरंगी}
\centering{॥ श्रीराम ॥}
\begin{sloppypar}\justifying\hyphenrules{nohyphenation}
मूल (चौपाई)—
\end{sloppypar}

{\bfseries\relscale{1.2}
\setlength{\mylenone}{0pt}
\settowidth{\mylentwo}{महाबीर बिक्रम बजरंगी}
\setlength{\mylenone}{\maxof{\mylenone}{\mylentwo}}
\settowidth{\mylentwo}{कुमति-निवार सुमति के संगी}
\setlength{\mylenone}{\maxof{\mylenone}{\mylentwo}}
\setlength{\mylentwo}{\baselineskip}
\setlength{\mylenone}{\mylenone + 1pt}
\setlength{\mylen}{(\textwidth - \mylenone)*\real{0.5}}
\begin{longtable}[l]{@{\hspace*{\mylen}}>{\setlength\parfillskip{0pt}}p{\mylenone}@{}@{}l@{}}
 & \\[-\the\mylentwo]
महाबीर बिक्रम बजरंगी & ।\\ \nopagebreak[1mm]
कुमति-निवार सुमति के संगी & ॥ ३ ॥
\end{longtable}
}

\parasepone
\index[ardha]{महाबीरबिक्रमबजरंगी(चौ.३,पूर्वार्ध)@महाबीर बिक्रम बजरंगी (चौ. ३, पूर्वार्ध)}
\index[ardha]{कुमतिनिवारसुमतिकेसंगी(चौ.३,उत्तरार्ध)@कुमति-निवार सुमति के संगी (चौ. ३, उत्तरार्ध)}
\index[pada]{महाबीर@महाबीर}
\index[pada]{बिक्रम@बिक्रम}
\index[pada]{बजरंगी@बजरंगी}
\index[pada]{कुमति@कुमति}
\index[pada]{निवार@निवार}
\index[pada]{सुमति@सुमति}
\index[pada]{के@के}
\index[pada]{संगी@संगी}
\begin{sloppypar}\justifying\hyphenrules{nohyphenation}
\textbf{शब्दार्थ}—\textbf{\textit{बिक्रम}} {\unifont{\relscale{0.7}▶}} विशिष्ट-क्रम-संपन्न या विशेष प्रकारकी लङ्घनक्रियासे संपन्न (\textbf{क्रमुँ पादविक्षेपे}, धा.पा. ४७३)।
\end{sloppypar}
\begin{sloppypar}\justifying\hyphenrules{nohyphenation}
\textbf{अर्थ}—आप महावीर तथा विशेष साधना-क्रमसे संपन्न किंवा समुद्रके लाँघनेकी विशिष्ट क्रियासे युक्त हैं। आपका शरीर वज्रमय है। आप कुबुद्धिको नष्ट करनेवाले एवं भगवद्भक्तिपूर्ण बुद्धिसे युक्त व्यक्तिका साथ देने वाले उचित संगी हैं।
\end{sloppypar}
\parasepone
\begin{sloppypar}\justifying\hyphenrules{nohyphenation}
\textbf{व्याख्या}—महावीर केवल बाह्य शत्रुओंपर विजय प्राप्त करता है, पर श्रीमारुति बाह्य एवं आन्तरिक (बाहरी तथा भीतरी) उभय प्रकारके शत्रुओंका दमन करते हैं। इसीलिए इनके विषयमें एक सूक्ति है—
\end{sloppypar}
{\bfseries
\setlength{\mylenone}{0pt}
\settowidth{\mylentwo}{ऋते भीष्माद्धि गाङ्गेयादृते वीराद्धनूमतः}
\setlength{\mylenone}{\maxof{\mylenone}{\mylentwo}}
\settowidth{\mylentwo}{हरिणीखुरमात्रेण चर्मणा मोहितं जगत्}
\setlength{\mylenone}{\maxof{\mylenone}{\mylentwo}}
\setlength{\mylentwo}{\baselineskip}
\setlength{\mylenone}{\mylenone + 1pt}
\setlength{\mylen}{(\textwidth - \mylenone)*\real{0.5}}
\begin{longtable}[l]{@{\hspace*{\mylen}}>{\setlength\parfillskip{0pt}}p{\mylenone}@{}@{}l@{}}
 & \\[-\the\mylentwo]
ऋते भीष्माद्धि गाङ्गेयादृते वीराद्धनूमतः & ।\\ \nopagebreak
हरिणीखुरमात्रेण चर्मणा मोहितं जगत् & ॥\\ \nopagebreak
\caption*{—मा.सु.सं. १७४९}
\end{longtable}
}
\begin{sloppypar}\justifying\hyphenrules{nohyphenation}
\noindent अर्थात्‌ \textbf{को जग काम नचाव न जेही} (रा.च.मा. ७-७०-७)। समस्त प्राणी काम-किङ्कर होकर उसके समक्ष नाचते हैं, पर आञ्जनेयजी रघुपति-किङ्कर होकर उन्हींके श्रीचरणोंमें नृत्य करते हैं। यथा—\textbf{जयति सिंहासनासीन-सीतारमण निरखि निर्भर-हरष नृत्यकारी} (वि.प. २७-५)। अतः मानसमें भी इन्हें \textit{महाबीर} कहा गया—
\end{sloppypar}
{\bfseries
\setlength{\mylenone}{0pt}
\settowidth{\mylentwo}{महाबीर बिनवऊँ हनुमाना}
\setlength{\mylenone}{\maxof{\mylenone}{\mylentwo}}
\settowidth{\mylentwo}{राम जासु जस आपु बखाना}
\setlength{\mylenone}{\maxof{\mylenone}{\mylentwo}}
\setlength{\mylentwo}{\baselineskip}
\setlength{\mylenone}{\mylenone + 1pt}
\setlength{\mylen}{(\textwidth - \mylenone)*\real{0.5}}
\begin{longtable}[l]{@{\hspace*{\mylen}}>{\setlength\parfillskip{0pt}}p{\mylenone}@{}@{}l@{}}
 & \\[-\the\mylentwo]
महाबीर बिनवऊँ हनुमाना & ।\\ \nopagebreak
राम जासु जस आपु बखाना & ॥\\ \nopagebreak
\caption*{—रा.च.मा. १-१७-१०}
\end{longtable}
}
\begin{sloppypar}\justifying\hyphenrules{nohyphenation}
\textbf{\textit{बिक्रम}}—इनका साधना-क्रम विशिष्ट है। इसलिए ये भगवान् श्रीरामको पीठ तथा हृदयपर आसीन करते हैं। यथा—\textbf{लिए दुऔ जन पीठ चढ़ाई} (रा.च.मा. ४-४-५) और \textbf{चलेउ हृदय धरि कृपानिधाना} (रा.च.मा. ५-२३-१२)। यद्वा, इनके समुद्र-लङ्घनकी क्रिया भी बहुत विशिष्ट है। स्वयं कहते हैं—\textbf{लीलहिं नाँघउ जलधि अपारा} (रा.च.मा. ४-३०-८)। मैनाक, सुरसा, एवं सिंहिका जैसे विघ्नोंके प्रस्तुत होनेपर भी इनका वेग विहत नहीं हुआ। यथा—
\end{sloppypar}
{\bfseries
\setlength{\mylenone}{0pt}
\settowidth{\mylentwo}{विषण्णा हरयः सर्वे हनूमन् किमुपेक्षसे}
\setlength{\mylenone}{\maxof{\mylenone}{\mylentwo}}
\settowidth{\mylentwo}{विक्रमस्व महावेग विष्णुस्त्रीन् विक्रमानिव}
\setlength{\mylenone}{\maxof{\mylenone}{\mylentwo}}
\setlength{\mylentwo}{\baselineskip}
\setlength{\mylenone}{\mylenone + 1pt}
\setlength{\mylen}{(\textwidth - \mylenone)*\real{0.5}}
\begin{longtable}[l]{@{\hspace*{\mylen}}>{\setlength\parfillskip{0pt}}p{\mylenone}@{}@{}l@{}}
 & \\[-\the\mylentwo]
विषण्णा हरयः सर्वे हनूमन् किमुपेक्षसे & ।\\ \nopagebreak
विक्रमस्व महावेग विष्णुस्त्रीन् विक्रमानिव & ॥\\ \nopagebreak
\caption*{—वा.रा. ४-६६-३७}
\end{longtable}
}
\begin{sloppypar}\justifying\hyphenrules{nohyphenation}
\noindent अर्थात् “हे हनुमान्! संपूर्ण वानर बहुत दुःखी हैं, उनकी उपेक्षा क्यों करते हो? जिस प्रकार भगवान् विष्णुने तीन बार चरणका विक्षेप करके समस्त लोकोंको नाप लिया था, उसी प्रकार एक बार समुद्र लाँघनेके लिए अपने चरणका विक्षेप करो।” अतः यहाँ \textbf{\textit{बिक्रम}}का अर्थ है विशेष प्रकारके चरण-विक्षेपकी प्रक्रिया, जिसकी एक छलाँगसे भी कम हो गया भूमण्डलसे सुदूर नभोमण्डलका परिमाण! यथा—\textbf{बानर सुभाय बालकेलि भूमि भानु लागि फलँगु फलाँगहूँ ते घाटि नभतल भो} (ह.बा. ५)।
\end{sloppypar}
\begin{sloppypar}\justifying\hyphenrules{nohyphenation}
\textbf{\textit{बजरंगी}}—यह शब्द \textit{वज्राङ्गी}का तद्भव है। जन्म लेनेके एक दिनके ही पश्चात्, अर्थात् कार्त्तिक कृष्ण अमावस्याको, प्रातःकाल क्षुधासे पीड़ित हनुमान्‌जी महाराजने लाल फलकी भ्रान्तिसे सूर्यनारायणके ऊपर आक्रमण कर दिया। उसी समय राहु सूर्यनारायणको ग्रसने हेतु वहाँ उपस्थित था। हनुमान्‌जीने सूर्यको छोड़कर राहुपर ही आक्रमण कर दिया। अनन्तर राहुसे सूचना पाकर वज्रपाणि इन्द्र ऐरावतपर आरूढ हो वहाँ उपस्थित हुए। अब श्वेत फलकी भ्रान्तिसे ऐरावतपर आक्रमण करते देखकर इन्द्रने उनकी बाँई ठोड़ीपर वज्रका प्रहार करके उन्हें धराशायी कर दिया। पश्चात् क्रुद्ध होकर पवनने अपने संचारको समाप्त कर प्रत्येक प्राणीके प्राणका अवरोध कर दिया। पश्चात् समस्त देवताओंने श्रीहनुमान्‌जीको विविध वरदान देकर पवनदेवको प्रसन्न किया। उसी समय इन्द्रने श्रीहनुमान्‌जीके शरीरको वज्रसे अभेद्यताका वरदान देकर उनका \textit{हनुमान्} नामकरण कर दिया। यह कथा पुराणोंमें विस्तारसे वर्णित है। अपि च—\textbf{उर बिशाल भुजदंड चंड नख बज्र बज्रतन} (ह.बा. २)।
\end{sloppypar}
\begin{sloppypar}\justifying\hyphenrules{nohyphenation}
\textbf{\textit{कुमति-निवार सुमति के संगी}}—हनुमान्‌जीके स्मरणसे कुमतिका निवारण होता है अथवा कुमतिपूर्ण खलका वे संहार करते हैं तथा सुमतिमान् सज्जनकी सहायता। जैसे कुमति रावणके विनाशमें वे मुख्य भूमिका निभाते हैं। यथा—
\end{sloppypar}
{\bfseries
\setlength{\mylenone}{0pt}
\settowidth{\mylentwo}{तव उर कुमति बसी बिपरीता}
\setlength{\mylenone}{\maxof{\mylenone}{\mylentwo}}
\settowidth{\mylentwo}{हित अनहित मानहु रिपु प्रीता}
\setlength{\mylenone}{\maxof{\mylenone}{\mylentwo}}
\setlength{\mylentwo}{\baselineskip}
\setlength{\mylenone}{\mylenone + 1pt}
\setlength{\mylen}{(\textwidth - \mylenone)*\real{0.5}}
\begin{longtable}[l]{@{\hspace*{\mylen}}>{\setlength\parfillskip{0pt}}p{\mylenone}@{}@{}l@{}}
 & \\[-\the\mylentwo]
तव उर कुमति बसी बिपरीता & ।\\ \nopagebreak
हित अनहित मानहु रिपु प्रीता & ॥\\ \nopagebreak
\caption*{—रा.च.मा. ५-४०-७}
\end{longtable}
}
\begin{sloppypar}\justifying\hyphenrules{nohyphenation}
\noindent और उसके विनाशके लिए—\textbf{दशमुख-दुसह-दरिद्र दरिबेको भयो प्रकट तिलोक ओक तुलसी-निधान सो} (ह.बा. ८)। और सुमति विभीषणकी सहायता कर हनुमान्‌जीने उन्हें अविचल राज्य दिला दिया। यथा—\textbf{जयति भुवनैकभूषण विभीषणवरद} (वि.प. २६-६)।
\end{sloppypar}
\paraseplotus
\pagebreak


\fancyhead[LE,RO]{{\textmd{\large चौ. ४: कंचन-बरन बिराज सुबेसा}}}
\phantomsection
\addcontentsline{toc}{section}{चौपाई ४: कंचन-बरन बिराज सुबेसा}
\centering{॥ श्रीराम ॥}
\begin{sloppypar}\justifying\hyphenrules{nohyphenation}
मूल (चौपाई)—
\end{sloppypar}

{\bfseries\relscale{1.2}
\setlength{\mylenone}{0pt}
\settowidth{\mylentwo}{कंचन-बरन बिराज सुबेसा}
\setlength{\mylenone}{\maxof{\mylenone}{\mylentwo}}
\settowidth{\mylentwo}{कानन कुंडल कुंचित केसा}
\setlength{\mylenone}{\maxof{\mylenone}{\mylentwo}}
\setlength{\mylentwo}{\baselineskip}
\setlength{\mylenone}{\mylenone + 1pt}
\setlength{\mylen}{(\textwidth - \mylenone)*\real{0.5}}
\begin{longtable}[l]{@{\hspace*{\mylen}}>{\setlength\parfillskip{0pt}}p{\mylenone}@{}@{}l@{}}
 & \\[-\the\mylentwo]
कंचन-बरन बिराज सुबेसा & ।\\ \nopagebreak[1mm]
कानन कुंडल कुंचित केसा & ॥ ४ ॥
\end{longtable}
}

\parasepone
\index[ardha]{कंचनबरनबिराजसुबेसा(चौ.४,पूर्वार्ध)@कंचन-बरन बिराज सुबेसा (चौ. ४, पूर्वार्ध)}
\index[ardha]{काननकुंडलकुंचितकेसा(चौ.४,उत्तरार्ध)@कानन कुंडल कुंचित केसा (चौ. ४, उत्तरार्ध)}
\index[pada]{कंचन@कंचन}
\index[pada]{बरन@बरन}
\index[pada]{बिराज@बिराज}
\index[pada]{सुबेसा@सुबेसा}
\index[pada]{कानन@कानन}
\index[pada]{कुंडल@कुंडल}
\index[pada]{कुंचित@कुंचित}
\index[pada]{केसा@केसा}
\begin{sloppypar}\justifying\hyphenrules{nohyphenation}
\textbf{शब्दार्थ}—\textbf{\textit{कंचन}} {\unifont{\relscale{0.7}▶}} स्वर्ण। \textbf{\textit{कुंचित}} {\unifont{\relscale{0.7}▶}} घुँघराले।
\end{sloppypar}
\begin{sloppypar}\justifying\hyphenrules{nohyphenation}
\textbf{अर्थ}—हे कपिश्रेष्ठ! आपका वर्ण तप्त स्वर्णके समान तेजसे पूर्ण है तथा आप अत्यन्त सुन्दरवेषमें विराज रहे हैं। आपके श्रवणोंमें कुण्डल चमक रहे हैं तथा आपके केश घुँघराले हैं।
\end{sloppypar}
\parasepone
\begin{sloppypar}\justifying\hyphenrules{nohyphenation}
\textbf{व्याख्या}—श्रीहनुमान्‌जीने निष्किञ्चना सेवाके लिए अपनेको वानर-शरीरमें परिणत किया, जिसमें सुन्दर वेष, श्रवणोंमें कुण्डल, एवं केशोंकी सजावट संगत नहीं हो पाती। इन्होंने अपनेको सर्वविधिहीन चञ्चल कपि भी कहा। यथा—
\end{sloppypar}
{\bfseries
\setlength{\mylenone}{0pt}
\settowidth{\mylentwo}{कहुहु कवन मैं परम कुलीना}
\setlength{\mylenone}{\maxof{\mylenone}{\mylentwo}}
\settowidth{\mylentwo}{कपि चंचल सबहीं बिधि हीना}
\setlength{\mylenone}{\maxof{\mylenone}{\mylentwo}}
\setlength{\mylentwo}{\baselineskip}
\setlength{\mylenone}{\mylenone + 1pt}
\setlength{\mylen}{(\textwidth - \mylenone)*\real{0.5}}
\begin{longtable}[l]{@{\hspace*{\mylen}}>{\setlength\parfillskip{0pt}}p{\mylenone}@{}@{}l@{}}
 & \\[-\the\mylentwo]
कहुहु कवन मैं परम कुलीना & ।\\ \nopagebreak
कपि चंचल सबहीं बिधि हीना & ॥\\ \nopagebreak
\caption*{—रा.च.मा. ५-७-७}
\end{longtable}
}
\begin{sloppypar}\justifying\hyphenrules{nohyphenation}
\noindent श्रीगोस्वामीजीने भी साधुमें वेष-प्राधान्यका खण्डन करते हुए जाम्बवान् एवं हनुमान्‌जीको ही कुवेषधारी होनेपर साधुओंमें शिरमौर एवं सम्मानार्ह माना। यथा—
\end{sloppypar}
{\bfseries
\setlength{\mylenone}{0pt}
\settowidth{\mylentwo}{किएहुँ कुबेष साधु सनमानू}
\setlength{\mylenone}{\maxof{\mylenone}{\mylentwo}}
\settowidth{\mylentwo}{जिमि जग जामवंत हनुमानू}
\setlength{\mylenone}{\maxof{\mylenone}{\mylentwo}}
\setlength{\mylentwo}{\baselineskip}
\setlength{\mylenone}{\mylenone + 1pt}
\setlength{\mylen}{(\textwidth - \mylenone)*\real{0.5}}
\begin{longtable}[l]{@{\hspace*{\mylen}}>{\setlength\parfillskip{0pt}}p{\mylenone}@{}@{}l@{}}
 & \\[-\the\mylentwo]
किएहुँ कुबेष साधु सनमानू & ।\\ \nopagebreak
जिमि जग जामवंत हनुमानू & ॥\\ \nopagebreak
\caption*{—रा.च.मा. १-७-७}
\end{longtable}
}
\begin{sloppypar}\justifying\hyphenrules{nohyphenation}
\noindent अतः यहाँ ब्राह्मण-वेषधारी हनुमान्‌जीकी झाँकीका वर्णन संगत लगता है। श्रीमानसजीमें भी श्रीराम, विभीषण, एवं श्रीभरतजीके समक्ष ब्राह्मण वेषमें हनुमान्‌जीका आगमन प्रसिद्ध ही है। यथा—
\end{sloppypar}
\begin{sloppypar}\justifying\hyphenrules{nohyphenation}
(१) \textbf{बिप्र रूप धरि कपि तहँ गयऊ} (रा.च.मा. ४-१-६)।
\end{sloppypar}
\begin{sloppypar}\justifying\hyphenrules{nohyphenation}
(२) \textbf{बिप्र रूप धरि बचन सुनाए} (रा.च.मा. ५-६-५)।
\end{sloppypar}
\begin{sloppypar}\justifying\hyphenrules{nohyphenation}
(३) \textbf{बिप्र रूप धरि पवनसुत आइ गयउ जनु पोत} (रा.च.मा. ७-१क)।
\end{sloppypar}
\begin{sloppypar}\justifying\hyphenrules{nohyphenation}
\textbf{\textit{कंचन-बरन}}—गौर शरीरकी उपमा काञ्चन वर्णसे ही दी जाती है। तथा ब्राह्मणका गौरवर्ण उसकी कुलीनताका द्योतक होता है—\textbf{गौरो ब्राह्मणः कुलीनः} (लोकोक्ति)। बहुवेषधारी आञ्जनेयका कुलीन भद्रवेष एवं केशोंका कुञ्चित होना उनके रूपानुरूप ही है। यही ध्यान अगली चौपाईमें भी समझना चाहिए।
\end{sloppypar}
\paraseplotus
\pagebreak


\fancyhead[LE,RO]{{\textmd{\large चौ. ५: हाथ बज्र अरु ध्वजा बिराजै}}}
\phantomsection
\addcontentsline{toc}{section}{चौपाई ५: हाथ बज्र अरु ध्वजा बिराजै}
\centering{॥ श्रीराम ॥}
\begin{sloppypar}\justifying\hyphenrules{nohyphenation}
मूल (चौपाई)—
\end{sloppypar}

{\bfseries\relscale{1.2}
\setlength{\mylenone}{0pt}
\settowidth{\mylentwo}{हाथ बज्र अरु ध्वजा बिराजै}
\setlength{\mylenone}{\maxof{\mylenone}{\mylentwo}}
\settowidth{\mylentwo}{काँधे मूँज-जनेऊ छाजै}
\setlength{\mylenone}{\maxof{\mylenone}{\mylentwo}}
\setlength{\mylentwo}{\baselineskip}
\setlength{\mylenone}{\mylenone + 1pt}
\setlength{\mylen}{(\textwidth - \mylenone)*\real{0.5}}
\begin{longtable}[l]{@{\hspace*{\mylen}}>{\setlength\parfillskip{0pt}}p{\mylenone}@{}@{}l@{}}
 & \\[-\the\mylentwo]
हाथ बज्र अरु ध्वजा बिराजै & ।\\ \nopagebreak[1mm]
काँधे मूँज-जनेऊ छाजै & ॥ ५ ॥
\end{longtable}
}

\parasepone
\index[ardha]{हाथबज्रअरुध्वजाबिराजै(चौ.५,पूर्वार्ध)@हाथ बज्र अरु ध्वजा बिराजै (चौ. ५, पूर्वार्ध)}
\index[ardha]{काँधेमूँजजनेऊछाजै(चौ.५,उत्तरार्ध)@काँधे मूँज-जनेऊ छाजै (चौ. ५, उत्तरार्ध)}
\index[pada]{हाथ@हाथ}
\index[pada]{बज्र@बज्र}
\index[pada]{अरु@अरु}
\index[pada]{ध्वजा@ध्वजा}
\index[pada]{बिराजै@बिराजै}
\index[pada]{काँधे@काँधे}
\index[pada]{मूँज@मूँज}
\index[pada]{जनेऊ@जनेऊ}
\index[pada]{छाजै@छाजै}
\begin{sloppypar}\justifying\hyphenrules{nohyphenation}
\textbf{शब्दार्थ}—\textbf{\textit{छाजै}} {\unifont{\relscale{0.7}▶}} शोभित हो रहा है।
\end{sloppypar}
\begin{sloppypar}\justifying\hyphenrules{nohyphenation}
\textbf{अर्थ}—हे आञ्जनेय! आपके वज्रवत् सुदृढ हस्तमें श्रीरामकी विजय-ध्वजा विराज रही है एवं आपके स्कन्धपर मूँजका यज्ञोपवीत शोभित हो रहा है।
\end{sloppypar}
\parasepone
\begin{sloppypar}\justifying\hyphenrules{nohyphenation}
\textbf{व्याख्या}—यह झाँकी श्रीभरतजीके समक्ष ब्राह्मण-वेषमें पधारे हुए श्रीआञ्जनेयकी है। इनके हाथमें श्रीराघवजीकी विजय-वैजयन्ती फहरा रही है। यथा—\textbf{भानुकुल-भानु-कीरति-पताका} (वि.प. २६-६)। काँधेपर मूँजका यज्ञोपवीत आञ्जनेयके अखण्ड ब्रह्मचर्यको सूचित करता है।
\end{sloppypar}
\begin{sloppypar}\justifying\hyphenrules{nohyphenation}
अथवा, इनके हाथमें वज्रके समान शत्रुदल-नाशिनी गदा एवं वैष्णव विजय-ध्वजा विराजमान हैं। इस व्याख्यासे व्यञ्जित हनुमान्‌जीका गदाधारण ध्वनित होता है।
\end{sloppypar}
\paraseplotus
\pagebreak


\fancyhead[LE,RO]{{\textmd{\large चौ. ६: शंकर स्वयं केसरीनंदन}}}
\phantomsection
\addcontentsline{toc}{section}{चौपाई ६: शंकर स्वयं केसरीनंदन}
\centering{॥ श्रीराम ॥}
\begin{sloppypar}\justifying\hyphenrules{nohyphenation}
मूल (चौपाई)—
\end{sloppypar}

{\bfseries\relscale{1.2}
\setlength{\mylenone}{0pt}
\settowidth{\mylentwo}{शंकर स्वयं केसरीनंदन}
\setlength{\mylenone}{\maxof{\mylenone}{\mylentwo}}
\settowidth{\mylentwo}{तेज प्रताप महा जग-बंदन}
\setlength{\mylenone}{\maxof{\mylenone}{\mylentwo}}
\setlength{\mylentwo}{\baselineskip}
\setlength{\mylenone}{\mylenone + 1pt}
\setlength{\mylen}{(\textwidth - \mylenone)*\real{0.5}}
\begin{longtable}[l]{@{\hspace*{\mylen}}>{\setlength\parfillskip{0pt}}p{\mylenone}@{}@{}l@{}}
 & \\[-\the\mylentwo]
शंकर स्वयं केसरीनंदन & ।\\ \nopagebreak[1mm]
तेज प्रताप महा जग-बंदन & ॥ ६ ॥
\end{longtable}
}

\parasepone
\index[ardha]{शंकरस्वयंकेसरीनंदन(चौ.६,पूर्वार्ध)@शंकर स्वयं केसरीनंदन (चौ. ६, पूर्वार्ध)}
\index[ardha]{तेजप्रतापमहाजगबंदन(चौ.६,उत्तरार्ध)@तेज प्रताप महा जग-बंदन (चौ. ६, उत्तरार्ध)}
\index[pada]{शंकर@शंकर}
\index[pada]{स्वयं@स्वयं}
\index[pada]{केसरीनंदन@केसरीनंदन}
\index[pada]{तेज@तेज}
\index[pada]{प्रताप@प्रताप}
\index[pada]{महा@महा}
\index[pada]{जग@जग}
\index[pada]{बंदन@बंदन}
\begin{sloppypar}\justifying\hyphenrules{nohyphenation}
\textbf{शब्दार्थ}—\textbf{\textit{स्वयं}} {\unifont{\relscale{0.7}▶}} साक्षात्।
\end{sloppypar}
\begin{sloppypar}\justifying\hyphenrules{nohyphenation}
\textbf{अर्थ}—हे प्रभो! आप साक्षात् श्रीशंकर भगवान् अर्थात् उनके अभिन्न अंश तथा केसरीके नन्दन (क्षेत्रज पुत्र) हैं। आपका तेज एवं प्रताप महान् है तथा आप संपूर्ण  जगत्‌के द्वारा वन्दित हैं।
\end{sloppypar}
\parasepone
\begin{sloppypar}\justifying\hyphenrules{nohyphenation}
\textbf{व्याख्या}—अर्थात् स्वयं शिव ही केसरीनन्दनके रूपमें पधारे। यथा—\textbf{रुद्र-अवतार संसार-पाता} (वि.प. २५-३)। \textbf{\textit{शंकर स्वयं केसरीनंदन}} यह प्रसंग अद्यावधि बहुतसे जिज्ञासुओंकी जिज्ञासा एवं अनेक शङ्कालुओंकी शङ्काका केन्द्र-बिन्दु बना रहा है क्योंकि एक ही हनुमान्‌जी महाराजको \textbf{\textit{पवनसुत}} एवं \textbf{\textit{केसरीनंदन}} शब्दसे अभिहित किया गया है। एक पुत्रके दो पिता कैसे संभव हैं? पर इसका समाधान श्रीराघवकी कृपासे अत्यन्त सरल एवं सुबोध है। सौभाग्यका विषय है कि श्रीहनुमान्‌जीके न केवल दो अपितु तीन पिताओंका प्रमाण \textit{हनुमान्‌-चालीसा}में ही प्राप्त हो जाता है—
\end{sloppypar}
\begin{sloppypar}\justifying\hyphenrules{nohyphenation}
(१) \textbf{पवनसुत-नामा} (ह.चा. २)
\end{sloppypar}
\begin{sloppypar}\justifying\hyphenrules{nohyphenation}
(२) \textbf{केसरीनंदन} (ह.चा. ६)
\end{sloppypar}
\begin{sloppypar}\justifying\hyphenrules{nohyphenation}
(३) \textbf{राम-दुलारे} (ह.चा. ३०)
\end{sloppypar}
\begin{sloppypar}\justifying\hyphenrules{nohyphenation}
\noindent इसका उत्तर यह है कि श्रीहनुमान्‌जी श्रीपवनके औरस पुत्र हैं। यथा—\textbf{मारुतस्यौरसः पुत्रः} (वा.रा. ४-६६-७)। क्योंकि वायुदेवने ही साक्षात् शिवके तेजको अञ्जनाके गर्भमें समाहित किया था, क्योंकि उनके बिना इस पवित्र तेजको कोई भी अञ्जना तक पहुँचा नहीं सकता था। चूँकि ये केसरीजीकी पत्नीमें प्रकट हुए, अतः ये केसरी नामक वानरके क्षेत्रज पुत्र कहलाए। यथा—\textbf{सकैं न बिलोकि बेष केसरी-कुमार को} (क. ५-१२)। श्रीरामजीने इन्हें वात्सल्य प्रदान किया, अतः ये उनके मानस पुत्र हुए। यथा—
\end{sloppypar}
{\bfseries
\setlength{\mylenone}{0pt}
\settowidth{\mylentwo}{सुनु सुत तोहि उरिन मैं नाहीं}
\setlength{\mylenone}{\maxof{\mylenone}{\mylentwo}}
\settowidth{\mylentwo}{देखेउँ करि बिचार मन माहीं}
\setlength{\mylenone}{\maxof{\mylenone}{\mylentwo}}
\setlength{\mylentwo}{\baselineskip}
\setlength{\mylenone}{\mylenone + 1pt}
\setlength{\mylen}{(\textwidth - \mylenone)*\real{0.5}}
\begin{longtable}[l]{@{\hspace*{\mylen}}>{\setlength\parfillskip{0pt}}p{\mylenone}@{}@{}l@{}}
 & \\[-\the\mylentwo]
सुनु सुत तोहि उरिन मैं नाहीं & ।\\ \nopagebreak
देखेउँ करि बिचार मन माहीं & ॥\\ \nopagebreak
\caption*{—रा.च.मा. ५-३२-७}
\end{longtable}
}
\begin{sloppypar}\justifying\hyphenrules{nohyphenation}
पौराणिक गाथाओंसे स्पष्ट है कि अञ्जनाकी तपस्यासे प्रसन्न होकर शिवजीने उनके यहाँ पुत्ररूपमें आनेका वरदान दिया। चूँकि शिवजी आञ्जनेयके रूपमें अवतीर्ण हुए, अतः \textbf{\textit{स्वयं}} शब्द पूर्णावतारके ही अर्थमें व्यवहृत हुआ।
\end{sloppypar}
\begin{sloppypar}\justifying\hyphenrules{nohyphenation}
इस प्रकार उपासनाकी दृष्टिसे श्रीआञ्जनेय वायुके औरस पुत्र, ज्ञानकी दृष्टिसे शिवजीके अभिन्नावतार, कर्मकाण्डकी दृष्टिसे कपिकुलतिलक केसरीके क्षेत्रज पुत्र, एवं शरणागतिकी दृष्टिसे श्रीराघवके मानस पुत्र हैं।
\end{sloppypar}
\begin{sloppypar}\justifying\hyphenrules{nohyphenation}
\textbf{\textit{तेज प्रताप महा जग-बंदन}}—इनका तेज एवं प्रताप महान् है। यथा—
\end{sloppypar}
\begin{sloppypar}\justifying\hyphenrules{nohyphenation}
(१) \textbf{तेज को निधान मानो कोटिक कृसानु भानु} (क. ५-४)।
\end{sloppypar}
\begin{sloppypar}\justifying\hyphenrules{nohyphenation}
(२) \textbf{बेग जीत्यो मारुत प्रताप मारतंड कोटि} (क. ५-९)।
\end{sloppypar}
\begin{sloppypar}\justifying\hyphenrules{nohyphenation}
प्रचलित प्रतिमें \textit{शंकर-सुवन} पाठ मिलता है, जो अशुद्ध और अनुचित है।\footnote{\ देखें: {\englishfont{\relscale{0.75} \foreignlanguage{english}{\textit{Mahāvīrī: Hanumān-Cālīsā Demystified} (2018). Translated, expanded, and annotated by Nityānanda Miśra. New Delhi: Bloomsbury India, ISBN 978-93-87471-59-7, pp.~68, 169–170}}}—संपादक।}
\end{sloppypar}
\paraseplotus
\vspace{2\baselineskip}
\pagebreak


\fancyhead[LE,RO]{{\textmd{\large चौ. ७: बिद्यावान गुणी अति चातुर}}}
\phantomsection
\addcontentsline{toc}{section}{चौपाई ७: बिद्यावान गुणी अति चातुर}
\centering{॥ श्रीराम ॥}
\begin{sloppypar}\justifying\hyphenrules{nohyphenation}
मूल (चौपाई)—
\end{sloppypar}

{\bfseries\relscale{1.2}
\setlength{\mylenone}{0pt}
\settowidth{\mylentwo}{बिद्यावान गुणी अति चातुर}
\setlength{\mylenone}{\maxof{\mylenone}{\mylentwo}}
\settowidth{\mylentwo}{राम-काज करिबे को आतुर}
\setlength{\mylenone}{\maxof{\mylenone}{\mylentwo}}
\setlength{\mylentwo}{\baselineskip}
\setlength{\mylenone}{\mylenone + 1pt}
\setlength{\mylen}{(\textwidth - \mylenone)*\real{0.5}}
\begin{longtable}[l]{@{\hspace*{\mylen}}>{\setlength\parfillskip{0pt}}p{\mylenone}@{}@{}l@{}}
 & \\[-\the\mylentwo]
बिद्यावान गुणी अति चातुर & ।\\ \nopagebreak[1mm]
राम-काज करिबे को आतुर & ॥ ७ ॥
\end{longtable}
}

\parasepone
\index[ardha]{बिद्यावानगुणीअतिचातुर(चौ.७,पूर्वार्ध)@बिद्यावान गुणी अति चातुर (चौ. ७, पूर्वार्ध)}
\index[ardha]{रामकाजकरिबेकोआतुर(चौ.७,उत्तरार्ध)@राम-काज करिबे को आतुर (चौ. ७, उत्तरार्ध)}
\index[pada]{बिद्यावान@बिद्यावान}
\index[pada]{गुणी@गुणी}
\index[pada]{अति@अति}
\index[pada]{चातुर@चातुर}
\index[pada]{राम@राम}
\index[pada]{काज@काज}
\index[pada]{करिबे@करिबे}
\index[pada]{को@को}
\index[pada]{आतुर@आतुर}
\begin{sloppypar}\justifying\hyphenrules{nohyphenation}
\textbf{शब्दार्थ}—\textbf{\textit{आतुर}} {\unifont{\relscale{0.7}▶}} उत्सुक।
\end{sloppypar}
\begin{sloppypar}\justifying\hyphenrules{nohyphenation}
\textbf{अर्थ}—हे आञ्जनेय! आप समस्त विद्याओंके प्रशस्त भण्डार हैं एवं समस्त गुण आपमें निवास करते हैं। एवं आप अत्यन्त चतुर हैं तथा श्रीरामचन्द्रजूके कार्यको करनेके लिए उत्सुक रहा करते हैं।
\end{sloppypar}
\parasepone
\begin{sloppypar}\justifying\hyphenrules{nohyphenation}
\textbf{व्याख्या}—\textbf{\textit{बिद्या}} शब्द यहाँ अष्टादश विद्याओंका संकेत करता है। जैसा कि ऊपर कहा जा चुका है कि श्रीहनुमान्‌जीने श्रीमानसके धरातलपर तीन बार ब्राह्मण-वेष बनाया। प्रथम बार श्रीराम-लक्ष्मणके समक्ष। श्रीराम विद्यानिधि हैं। यथा—\textbf{बिद्या-बिनय-निपुन गुणशीला} (रा.च.मा. १-२०४-६) और \textbf{बिद्यानिधि कहुँ बिद्या दीन्ही} (रा.च.मा. १-२०९-७)। अतः विद्यानिधिके समक्ष हनुमान्‌जीने ब्रह्मविषयक प्रश्न करके अपनी विद्या-प्रखरताका परिचय दिया। द्वितीय बार विभीषणके समक्ष लङ्का में। यहाँ विभीषणके आकुलत्व रूप गुणको देखकर उन्हें राम-परत्वका उपदेश कर अपनी गुणज्ञताका परिचय दिया। क्योंकि—
\end{sloppypar}
{\bfseries
\setlength{\mylenone}{0pt}
\settowidth{\mylentwo}{सोइ सर्बग्य गुणी सोइ ग्याता}
\setlength{\mylenone}{\maxof{\mylenone}{\mylentwo}}
\settowidth{\mylentwo}{सोइ महि मंडित पंडित दाता}
\setlength{\mylenone}{\maxof{\mylenone}{\mylentwo}}
\settowidth{\mylentwo}{धर्म-परायन सोइ कुल त्राता}
\setlength{\mylenone}{\maxof{\mylenone}{\mylentwo}}
\settowidth{\mylentwo}{राम-चरन जा कर मन राता}
\setlength{\mylenone}{\maxof{\mylenone}{\mylentwo}}
\setlength{\mylentwo}{\baselineskip}
\setlength{\mylenone}{\mylenone + 1pt}
\setlength{\mylen}{(\textwidth - \mylenone)*\real{0.5}}
\begin{longtable}[l]{@{\hspace*{\mylen}}>{\setlength\parfillskip{0pt}}p{\mylenone}@{}@{}l@{}}
 & \\[-\the\mylentwo]
सोइ सर्बग्य गुणी सोइ ग्याता & ।\\ \nopagebreak
सोइ महि मंडित पंडित दाता & ॥\\
धर्म-परायन सोइ कुल त्राता & ।\\ \nopagebreak
राम-चरन जा कर मन राता & ॥\\ \nopagebreak
\caption*{—रा.च.मा. ७-१२७-१,२}
\end{longtable}
}
\begin{sloppypar}\justifying\hyphenrules{nohyphenation}
\noindent आञ्जनेय विभीषणसे कह पड़ते हैं—
\end{sloppypar}
{\bfseries
\setlength{\mylenone}{0pt}
\settowidth{\mylentwo}{अस मैं अधम सखा सुनु मोहू पर रघुबीर}
\setlength{\mylenone}{\maxof{\mylenone}{\mylentwo}}
\settowidth{\mylentwo}{कीन्ही कृपा सुमिरि गुण भरे बिलोचन नीर}
\setlength{\mylenone}{\maxof{\mylenone}{\mylentwo}}
\setlength{\mylentwo}{\baselineskip}
\setlength{\mylenone}{\mylenone + 1pt}
\setlength{\mylen}{(\textwidth - \mylenone)*\real{0.5}}
\begin{longtable}[l]{@{\hspace*{\mylen}}>{\setlength\parfillskip{0pt}}p{\mylenone}@{}@{}l@{}}
 & \\[-\the\mylentwo]
अस मैं अधम सखा सुनु मोहू पर रघुबीर & ।\\ \nopagebreak
कीन्ही कृपा सुमिरि गुण भरे बिलोचन नीर & ॥\\ \nopagebreak
\caption*{—रा.च.मा. ५-७}
\end{longtable}
}
\begin{sloppypar}\justifying\hyphenrules{nohyphenation}
\noindent तृतीय बार नन्दिग्राममें भरतजीके समक्ष। यहाँ आञ्जनेयका लोकोत्तर चातुर्य दृष्टिगोचर होता है। श्रीभरतभद्र अविराम अश्रुधारासे श्रीरामभद्रके भावनामय पादपद्मका अभिषेक कर रहे हैं—\textbf{राम राम रघुपति जपत स्रवत नयन-जलजात} (रा.च.मा. ७-१ख)। अतः श्रीहनुमान्‌जी सामने नहीं आते। क्योंकि अश्रुपात-कालमें ठीक-ठीक न दीख पड़नेसे पूर्वकी भाँति फिर कोई अन्यथा अनुमान न हो जाए, इसलिए—\textbf{बोलेउ श्रवन सुधा सम बानी} (रा.च.मा. ७-२-२)। उनके कानमें जाकर बोले। तीन चौपाइयोंमें श्रीराघवके आगमनका समाचार सुनाकर श्रीभरतको उन्होंने विरह, भय, तथा विषादसे मुक्त किया। और भरतभद्रने \textbf{नाहिन तात उरिन मैं तोही} (रा.च.मा. ७-२-१४) कहकर कृतज्ञता-ज्ञापन किया। ग्रन्थ-गौरवके भयसे सूत्ररूपमें दिग्दर्शन कराया गया। विज्ञ पाठक तीनों प्रसंगोंकी स्वयं संगति लगा लेंगे।
\end{sloppypar}
\begin{sloppypar}\justifying\hyphenrules{nohyphenation}
\textbf{\textit{राम-काज करिबे~को आतुर}}—श्रीराम\-कार्य करनेके लिए ये इतने आतुर हैं कि क्षण-भर भी विश्राम उचित नहीं मानते, यथा—\textbf{राम-काज कीन्हे बिनु मोहि कहाँ बिश्राम} (रा.च.मा. ५-१); और नागपाशमें बँधकर भी लज्जाका अनुभव नहीं करते—
\end{sloppypar}
{\bfseries
\setlength{\mylenone}{0pt}
\settowidth{\mylentwo}{मोहि न कछु बाँधे कइ लाजा}
\setlength{\mylenone}{\maxof{\mylenone}{\mylentwo}}
\settowidth{\mylentwo}{कीन्ह चहउँ निज प्रभु कर काजा}
\setlength{\mylenone}{\maxof{\mylenone}{\mylentwo}}
\setlength{\mylentwo}{\baselineskip}
\setlength{\mylenone}{\mylenone + 1pt}
\setlength{\mylen}{(\textwidth - \mylenone)*\real{0.5}}
\begin{longtable}[l]{@{\hspace*{\mylen}}>{\setlength\parfillskip{0pt}}p{\mylenone}@{}@{}l@{}}
 & \\[-\the\mylentwo]
मोहि न कछु बाँधे कइ लाजा & ।\\ \nopagebreak
कीन्ह चहउँ निज प्रभु कर काजा & ॥\\ \nopagebreak
\caption*{—रा.च.मा. ५-२२-६}
\end{longtable}
}
\paraseplotus
\pagebreak


\fancyhead[LE,RO]{{\textmd{\large चौ. ८: प्रभु-चरित्र सुनिबे को रसिया}}}
\phantomsection
\addcontentsline{toc}{section}{चौपाई ८: प्रभु-चरित्र सुनिबे को रसिया}
\centering{॥ श्रीराम ॥}
\begin{sloppypar}\justifying\hyphenrules{nohyphenation}
मूल (चौपाई)—
\end{sloppypar}

{\bfseries\relscale{1.2}
\setlength{\mylenone}{0pt}
\settowidth{\mylentwo}{प्रभु-चरित्र सुनिबे को रसिया}
\setlength{\mylenone}{\maxof{\mylenone}{\mylentwo}}
\settowidth{\mylentwo}{राम-लखन-सीता-मन-बसिया}
\setlength{\mylenone}{\maxof{\mylenone}{\mylentwo}}
\setlength{\mylentwo}{\baselineskip}
\setlength{\mylenone}{\mylenone + 1pt}
\setlength{\mylen}{(\textwidth - \mylenone)*\real{0.5}}
\begin{longtable}[l]{@{\hspace*{\mylen}}>{\setlength\parfillskip{0pt}}p{\mylenone}@{}@{}l@{}}
 & \\[-\the\mylentwo]
प्रभु-चरित्र सुनिबे को रसिया & ।\\ \nopagebreak[1mm]
राम-लखन-सीता-मन-बसिया & ॥ ८ ॥
\end{longtable}
}

\parasepone
\index[ardha]{प्रभुचरित्रसुनिबेकोरसिया(चौ.८,पूर्वार्ध)@प्रभु-चरित्र सुनिबे को रसिया (चौ. ८, पूर्वार्ध)}
\index[ardha]{रामलखनसीतामनबसिया(चौ.८,उत्तरार्ध)@राम-लखन-सीता-मन-बसिया (चौ. ८, उत्तरार्ध)}
\index[pada]{प्रभु@प्रभु}
\index[pada]{चरित्र@चरित्र}
\index[pada]{सुनिबे@सुनिबे}
\index[pada]{को@को}
\index[pada]{रसिया@रसिया}
\index[pada]{राम@राम}
\index[pada]{लखन@लखन}
\index[pada]{सीता@सीता}
\index[pada]{मन@मन}
\index[pada]{बसिया@बसिया}
\begin{sloppypar}\justifying\hyphenrules{nohyphenation}
\textbf{शब्दार्थ}—\textbf{\textit{रसिया}} {\unifont{\relscale{0.7}▶}} रसिक। \textbf{\textit{बसिया}} {\unifont{\relscale{0.7}▶}} निवास करने वाले।
\end{sloppypar}
\begin{sloppypar}\justifying\hyphenrules{nohyphenation}
\textbf{अर्थ}—हे मारुते! आप श्रीराघवजूके चरितामृतको सुननेके अद्वितीय रसिक हैं एवं आपके मनोमन्दिरमें श्रीराम, श्रीलक्ष्मण, एवं श्रीसीताका निवास है। यद्वा, आप ही वात्सल्यातिशय होनेसे श्रीराम-लक्ष्मण एवं श्रीसीताजीके मनमें निवास करते हैं।
\end{sloppypar}
\parasepone
\begin{sloppypar}\justifying\hyphenrules{nohyphenation}
\textbf{व्याख्या}—प्रभु-चरित्र-श्रवणके श्रीहनुमान्‌जी इतने रसिक हैं कि कथाके लोभमें इन्होंने प्रभुका सान्निध्य ठुकरा दिया। राज्याभिषेकके पश्चात् राजाधिराज श्रीरामने संपूर्ण वानर-भालुओंको विविध दान एवं प्रीतिदान देकर विसर्जित किया एवं श्रीआञ्जनेयसे अपने परम धाममें रहनेके लिए मूक इच्छा व्यक्त की। तब श्रीमारुतिने मूक भाषामें अन्तर-प्रश्न किया कि क्या आप अपने परम धाममें मेरे रामायण-कथा-श्रवणकी व्यवस्था करेंगे? श्रीरामभद्रजूको निरुत्तर देखकर हनुमान्‌जी महाराजने पुनः कहा, “हे वीर! जब तक आपकी श्रीरामायण-कथा इस भूमण्डलपर चलती रहेगी, तब तक आपकी आज्ञासे मेरे शरीरमें प्राण विद्यमान रहेंगे।” श्रीरामने भी इस वरदानको स्वीकारते हुए कहा, “जब तक इस लोकमें मेरी कथा चलेगी, तब तक तुम्हारी कीर्ति अचल रहेगी एवं तुम्हारे शरीरमें प्राण तब तक वर्तमान रहेंगे।” राम-कथाके लोभमें कालनेमिका षड्यन्त्र भी इन्हें क्षणमात्र तक स्वीकार्य-सा हो गया। स्वयं श्रीराम-कथा-श्रवणमात्रसे हनुमान्‌जीके नेत्र-कमल सजल हो जाते हैं और वाणी शिथिल हो जाती है। यथा—\textbf{जयति रामायण-श्रवण-संजात-रोमांच-लोचन-सजल-शिथिल-वाणी} (वि.प. २९-५)। श्रीराम-कथाके निमित्त ही जिन्होंने साकेतके सुखको ठुकराकर धराधामपर विचरण करते हुए राम-कथा-श्रवणार्थ अपना जीवन रक्खा, ऐसे श्रीआञ्जनेयके समान श्रीराम-कथाका और कौन रसिक हो सकता है?
\end{sloppypar}
\paraseplotus
\pagebreak


\fancyhead[LE,RO]{{\textmd{\large चौ. ९: सूक्ष्म रूप धरि सियहिं दिखावा}}}
\phantomsection
\addcontentsline{toc}{section}{चौपाई ९: सूक्ष्म रूप धरि सियहिं दिखावा}
\centering{॥ श्रीराम ॥}
\begin{sloppypar}\justifying\hyphenrules{nohyphenation}
मूल (चौपाई)—
\end{sloppypar}

{\bfseries\relscale{1.2}
\setlength{\mylenone}{0pt}
\settowidth{\mylentwo}{सूक्ष्म रूप धरि सियहिं दिखावा}
\setlength{\mylenone}{\maxof{\mylenone}{\mylentwo}}
\settowidth{\mylentwo}{बिकट रूप धरि लंक जरावा}
\setlength{\mylenone}{\maxof{\mylenone}{\mylentwo}}
\setlength{\mylentwo}{\baselineskip}
\setlength{\mylenone}{\mylenone + 1pt}
\setlength{\mylen}{(\textwidth - \mylenone)*\real{0.5}}
\begin{longtable}[l]{@{\hspace*{\mylen}}>{\setlength\parfillskip{0pt}}p{\mylenone}@{}@{}l@{}}
 & \\[-\the\mylentwo]
सूक्ष्म रूप धरि सियहिं दिखावा & ।\\ \nopagebreak[1mm]
बिकट रूप धरि लंक जरावा & ॥ ९ ॥
\end{longtable}
}

\parasepone
\index[ardha]{सूक्ष्मरूपधरिसियहिंदिखावा(चौ.९,पूर्वार्ध)@सूक्ष्म रूप धरि सियहिं दिखावा (चौ. ९, पूर्वार्ध)}
\index[ardha]{बिकटरूपधरिलंकजरावा(चौ.९,उत्तरार्ध)@बिकट रूप धरि लंक जरावा (चौ. ९, उत्तरार्ध)}
\index[pada]{सूक्ष्म@सूक्ष्म}
\index[pada]{रूप@रूप}
\index[pada]{धरि@धरि}
\index[pada]{सियहिं@सियहिं}
\index[pada]{दिखावा@दिखावा}
\index[pada]{बिकट@बिकट}
\index[pada]{रूप@रूप}
\index[pada]{धरि@धरि}
\index[pada]{लंक@लंक}
\index[pada]{जरावा@जरावा}
\begin{sloppypar}\justifying\hyphenrules{nohyphenation}
\textbf{शब्दार्थ}—\textbf{\textit{बिकट}} {\unifont{\relscale{0.7}▶}} भयंकर।
\end{sloppypar}
\begin{sloppypar}\justifying\hyphenrules{nohyphenation}
\textbf{अर्थ}—हे मारुते!  आपने सूक्ष्म अर्थात् अत्यन्त लघु बन्दरका रूप धारण करके माता सीताको दिखाया और भयंकर रूप धारण करके रावणकी नगरी लङ्काको भस्मसात् कर दिया।
\end{sloppypar}
\parasepone
\begin{sloppypar}\justifying\hyphenrules{nohyphenation}
\textbf{व्याख्या}—अशोक\-वाटिकामें हनुमान्‌जीका इतना लघुतम रूप हुआ कि उन्होंने अपनेको वृक्षके पल्लवोंमें छिपा रक्खा—
\end{sloppypar}
{\bfseries
\setlength{\mylenone}{0pt}
\settowidth{\mylentwo}{तरु-पल्लव महँ रहा लुकाई}
\setlength{\mylenone}{\maxof{\mylenone}{\mylentwo}}
\settowidth{\mylentwo}{करइ बिचार करौं का भाई}
\setlength{\mylenone}{\maxof{\mylenone}{\mylentwo}}
\setlength{\mylentwo}{\baselineskip}
\setlength{\mylenone}{\mylenone + 1pt}
\setlength{\mylen}{(\textwidth - \mylenone)*\real{0.5}}
\begin{longtable}[l]{@{\hspace*{\mylen}}>{\setlength\parfillskip{0pt}}p{\mylenone}@{}@{}l@{}}
 & \\[-\the\mylentwo]
तरु-पल्लव महँ रहा लुकाई & ।\\ \nopagebreak
करइ बिचार करौं का भाई & ॥\\ \nopagebreak
\caption*{—रा.च.मा. ५-९-१}
\end{longtable}
}
\begin{sloppypar}\justifying\hyphenrules{nohyphenation}
\textit{अध्यात्म-रामायण}में इन्हें \textbf{कलविङ्कसमाकारः} (अ.रा. ५-३-२०) कहा गया है, अर्थात् छोटी गौरैयाके समान इनका आकार था। यहाँ यह शङ्का करना निरर्थक है कि मुद्रिका कहाँ रही होगी। मुद्रिका भगवान्‌के श्रीविग्रहका आभूषण होनेसे चिन्मय है। अतः परिस्थितिके अनुसार लघुता एवं गौरव उसके लिए सहज है। \textit{गीतावली}जीमें सीता-मुद्रिका-संवाद भी वर्णित है। मुद्रिका स्पष्ट कहती है—\textbf{नींद भूख न देवरहि परिहरे~को पछिताउ} (गी. ५-४-२)। जो मुद्रिका श्रीसीताजीको श्रीराघवके समाचार सुना सकती है, उसकी लघुता एवं गुरुताके विषयमें संदेहको स्थान ही कहाँ? हनुमान्‌जीके लघु रूपको देखकर श्रीसीताजीको लीलापक्षमें संदेह हो गया। पुनः उन्होंने भीम रूपकी झाँकीसे मैथिलीके संदेहको दूर किया। यथा—
\end{sloppypar}
{\bfseries
\setlength{\mylenone}{0pt}
\settowidth{\mylentwo}{मोरे हृदय परम संदेहा}
\setlength{\mylenone}{\maxof{\mylenone}{\mylentwo}}
\settowidth{\mylentwo}{सुनि कपि प्रगट कीन्ह निज देहा}
\setlength{\mylenone}{\maxof{\mylenone}{\mylentwo}}
\settowidth{\mylentwo}{कनक-भूधराकार शरीरा}
\setlength{\mylenone}{\maxof{\mylenone}{\mylentwo}}
\settowidth{\mylentwo}{समर-भयंकर अतिबल बीरा}
\setlength{\mylenone}{\maxof{\mylenone}{\mylentwo}}
\settowidth{\mylentwo}{सीता मन भरोस तब भयऊ}
\setlength{\mylenone}{\maxof{\mylenone}{\mylentwo}}
\settowidth{\mylentwo}{पुनि लघु रूप पवनसुत लयऊ}
\setlength{\mylenone}{\maxof{\mylenone}{\mylentwo}}
\setlength{\mylentwo}{\baselineskip}
\setlength{\mylenone}{\mylenone + 1pt}
\setlength{\mylen}{(\textwidth - \mylenone)*\real{0.5}}
\begin{longtable}[l]{@{\hspace*{\mylen}}>{\setlength\parfillskip{0pt}}p{\mylenone}@{}@{}l@{}}
 & \\[-\the\mylentwo]
मोरे हृदय परम संदेहा & ।\\ \nopagebreak
सुनि कपि प्रगट कीन्ह निज देहा & ॥\\
कनक-भूधराकार शरीरा & ।\\ \nopagebreak
समर-भयंकर अतिबल बीरा & ॥\\
सीता मन भरोस तब भयऊ & ।\\ \nopagebreak
पुनि लघु रूप पवनसुत लयऊ & ॥\\ \nopagebreak
\caption*{—रा.च.मा. ५-१६-७,८,९}
\end{longtable}
}
\begin{sloppypar}\justifying\hyphenrules{nohyphenation}
\textbf{\textit{बिकट रूप धरि लंक जरावा}}—लङ्कादहनमें तो इनका विकट रूप प्रसिद्ध ही है। विशेष \textit{कवितावली}का सुन्दरकाण्ड द्रष्टव्य है। इसी अवसरपर रावणको अपना रुद्ररूप प्रदर्शित करने हेतु हनुमान्‌जीने अपना पञ्चमुखी रूप प्रदर्शित किया।
\end{sloppypar}
\paraseplotus
\pagebreak


\fancyhead[LE,RO]{{\textmd{\large चौ. १०: भीम रूप धरि असुर सँहारे}}}
\phantomsection
\addcontentsline{toc}{section}{चौपाई १०: भीम रूप धरि असुर सँहारे}
\centering{॥ श्रीराम ॥}
\begin{sloppypar}\justifying\hyphenrules{nohyphenation}
मूल (चौपाई)—
\end{sloppypar}

{\bfseries\relscale{1.2}
\setlength{\mylenone}{0pt}
\settowidth{\mylentwo}{भीम रूप धरि असुर सँहारे}
\setlength{\mylenone}{\maxof{\mylenone}{\mylentwo}}
\settowidth{\mylentwo}{रामचंद्र के काज सँवारे}
\setlength{\mylenone}{\maxof{\mylenone}{\mylentwo}}
\setlength{\mylentwo}{\baselineskip}
\setlength{\mylenone}{\mylenone + 1pt}
\setlength{\mylen}{(\textwidth - \mylenone)*\real{0.5}}
\begin{longtable}[l]{@{\hspace*{\mylen}}>{\setlength\parfillskip{0pt}}p{\mylenone}@{}@{}l@{}}
 & \\[-\the\mylentwo]
भीम रूप धरि असुर सँहारे & ।\\ \nopagebreak[1mm]
रामचंद्र के काज सँवारे & ॥ १० ॥
\end{longtable}
}

\parasepone
\index[ardha]{भीमरूपधरिअसुरसँहारे(चौ.१०,पूर्वार्ध)@भीम रूप धरि असुर सँहारे (चौ. १०, पूर्वार्ध)}
\index[ardha]{रामचंद्रकेकाजसँवारे(चौ.१०,उत्तरार्ध)@रामचंद्र के काज सँवारे (चौ. १०, उत्तरार्ध)}
\index[pada]{भीम@भीम}
\index[pada]{रूप@रूप}
\index[pada]{धरि@धरि}
\index[pada]{असुर@असुर}
\index[pada]{सँहारे@सँहारे}
\index[pada]{रामचंद्र@रामचंद्र}
\index[pada]{के@के}
\index[pada]{काज@काज}
\index[pada]{सँवारे@सँवारे}
\begin{sloppypar}\justifying\hyphenrules{nohyphenation}
\textbf{शब्दार्थ}—\textbf{\textit{भीम}} {\unifont{\relscale{0.7}▶}} बीभत्स तथा धीरोंको भी त्रास देने वाला।
\end{sloppypar}
\begin{sloppypar}\justifying\hyphenrules{nohyphenation}
\textbf{अर्थ}—आपने महाकालको भी भयभीत करने वाले भीम रूपको धारण कर रावणपक्षीय असुरोंका संहार किया एवं भगवान् श्रीरामचन्द्रजीके समस्त कार्योंको सँवारा।
\end{sloppypar}
\parasepone
\begin{sloppypar}\justifying\hyphenrules{nohyphenation}
\textbf{व्याख्या}—इनकी भीमता देखकर महाभारत-कालके धुरन्धर वीर भीमने भी अपनी आँखें बन्द कर लीं थीं। एक बार द्रौपदीकी रुचिरञ्जनके लिए स्वर्णकिञ्जल्क\-युक्त कमल लेने भीमसेन श्रीहनुमान्‌जीके निवासस्थान गन्धमादनके निकट कदलीवन पधारे। भीमको उद्धत देखकर हनुमान्‌जी वृद्ध बन्दरका रूप धारण कर मार्गमें लेट गए। भीमने मार्ग छोड़नेका अनुरोध किया। श्रीआञ्जनेयने सहजतासे कहा, “मैं वृद्ध हूँ, अतः मेरी पूँछ उठाकर मुझे इस स्थानसे हटा दो।” भीमकी सभी चेष्टाएँ असफल हुईं, पर हनुमान्‌जीकी पूँछ टस-से-मस न हुई। अपनेको श्रीहत देखकर भीमने उस वृद्ध वानरेन्द्रको प्रणाम करके उनका परिचय पूछा। अनन्तर अपना परिचय देकर श्रीमारुतिने भीमको संक्षेपमें रामायण-कथा सुनाई। पश्चात् अपने कथा-श्रवणमें व्यतिक्रम जानकर श्रीमारुति शीघ्र जानेके लिए भीमको आदिष्ट करते हुए बोले—
\end{sloppypar}
{\bfseries
\setlength{\mylenone}{0pt}
\settowidth{\mylentwo}{तदिहाप्सरसस्तात गन्धर्वाश्च तथाऽनघ}
\setlength{\mylenone}{\maxof{\mylenone}{\mylentwo}}
\settowidth{\mylentwo}{तस्य वीरस्य चरितं गायन्तो रमयन्ति माम्}
\setlength{\mylenone}{\maxof{\mylenone}{\mylentwo}}
\setlength{\mylentwo}{\baselineskip}
\setlength{\mylenone}{\mylenone + 1pt}
\setlength{\mylen}{(\textwidth - \mylenone)*\real{0.5}}
\begin{longtable}[l]{@{\hspace*{\mylen}}>{\setlength\parfillskip{0pt}}p{\mylenone}@{}@{}l@{}}
 & \\[-\the\mylentwo]
तदिहाप्सरसस्तात गन्धर्वाश्च तथाऽनघ & ।\\ \nopagebreak
तस्य वीरस्य चरितं गायन्तो रमयन्ति माम् & ॥\\ \nopagebreak
\caption*{—म.भा. ३-१४८-२०}
\end{longtable}
}
\begin{sloppypar}\justifying\hyphenrules{nohyphenation}
\noindent अर्थात् “हे भीम! स्वर्गकी श्रेष्ठ अप्सराएँ एवं तुम्बुरु आदि कुशलगायक गन्धर्व भगवान् श्रीराघवके चारु चरितको गाते हुए मुझे परमानन्द-सुधा-सागरमें मग्न किए रहते हैं।” जानेसे पहले भीमने उनके मौलिक रूपको देखनेकी इच्छा प्रकट की। तब श्रीहनुमान्‌जीने अपना स्वर्णशैल-संकाश शरीर प्रस्तुत किया, जिसे देखकर भीमने आँखें बन्द कर लीं। यह कथा \textit{महाभारत}के वनपर्वमें स्पष्ट है। इसीका उद्धरण गोस्वामीजीने \textit{कवितावली}में प्रस्तुत किया है—\textbf{कौनके तेज बलसीम भट भीम-से भीमता निरखि कर नयन ढाँके} (क. ६-४५)।
\end{sloppypar}
\begin{sloppypar}\justifying\hyphenrules{nohyphenation}
\textbf{\textit{असुर सँहारे}}—असुर-संहारका क्या कहना! आञ्जनेयके संग्रामकी प्रशंसा श्रीरामचन्द्र स्वयं करते हैं—
\end{sloppypar}
{\bfseries
\setlength{\mylenone}{0pt}
\settowidth{\mylentwo}{हाथिन सों हाथी मारे घोरे सों सँघारे घोरे}
\setlength{\mylenone}{\maxof{\mylenone}{\mylentwo}}
\settowidth{\mylentwo}{रथन सों रथ बिदरनि बलवान की}
\setlength{\mylenone}{\maxof{\mylenone}{\mylentwo}}
\settowidth{\mylentwo}{चंचल चपेट चोट-चरन चकोट चाहे}
\setlength{\mylenone}{\maxof{\mylenone}{\mylentwo}}
\settowidth{\mylentwo}{हहरानि फौजें भहरानी जातुधान की}
\setlength{\mylenone}{\maxof{\mylenone}{\mylentwo}}
\settowidth{\mylentwo}{बार बार सेवक-सराहना करत राम}
\setlength{\mylenone}{\maxof{\mylenone}{\mylentwo}}
\settowidth{\mylentwo}{तुलसी सराहै रीति साहिब सुजान की}
\setlength{\mylenone}{\maxof{\mylenone}{\mylentwo}}
\settowidth{\mylentwo}{लाँबी लूम लसत लपेटि पटकत भट}
\setlength{\mylenone}{\maxof{\mylenone}{\mylentwo}}
\settowidth{\mylentwo}{देखौ देखौ लखन लरनि हनुमान की}
\setlength{\mylenone}{\maxof{\mylenone}{\mylentwo}}
\setlength{\mylentwo}{\baselineskip}
\setlength{\mylenone}{\mylenone + 1pt}
\setlength{\mylen}{(\textwidth - \mylenone)*\real{0.5}}
\begin{longtable}[l]{@{\hspace*{\mylen}}>{\setlength\parfillskip{0pt}}p{\mylenone}@{}@{}l@{}}
 & \\[-\the\mylentwo]
हाथिन सों हाथी मारे घोरे सों सँघारे घोरे & \\ \nopagebreak
रथन सों रथ बिदरनि बलवान की & ।\\
चंचल चपेट चोट-चरन चकोट चाहे & \\ \nopagebreak
हहरानि फौजें भहरानी जातुधान की & ।\\
बार बार सेवक-सराहना करत राम & \\ \nopagebreak
तुलसी सराहै रीति साहिब सुजान की & ।\\
लाँबी लूम लसत लपेटि पटकत भट & \\ \nopagebreak
देखौ देखौ लखन लरनि हनुमान की & ॥\\ \nopagebreak
\caption*{—क. ६-४०}
\end{longtable}
}
\begin{sloppypar}\justifying\hyphenrules{nohyphenation}
\textbf{\textit{रामचंद्र के काज सँवारे}}—श्रीरामजीके समस्त कार्योंको हनुमान्‌जीने सजा दिया। भाव यह है कि रावणादिका वध राघवकी इच्छासे हो सकता था, पर विभीषणादिका उद्धार आञ्जनेयके बिना कथमपि संभव नहीं था। कारण कि जब तक जीव रघुनाथजीके सम्मुख नहीं होता, तब तक उसके पाप नष्ट नहीं होते। यथा—
\end{sloppypar}
{\bfseries
\setlength{\mylenone}{0pt}
\settowidth{\mylentwo}{सनमुख होइ जीव मोहि जबहीं}
\setlength{\mylenone}{\maxof{\mylenone}{\mylentwo}}
\settowidth{\mylentwo}{जन्म कोटि अघ नासहिं तबहीं}
\setlength{\mylenone}{\maxof{\mylenone}{\mylentwo}}
\setlength{\mylentwo}{\baselineskip}
\setlength{\mylenone}{\mylenone + 1pt}
\setlength{\mylen}{(\textwidth - \mylenone)*\real{0.5}}
\begin{longtable}[l]{@{\hspace*{\mylen}}>{\setlength\parfillskip{0pt}}p{\mylenone}@{}@{}l@{}}
 & \\[-\the\mylentwo]
सनमुख होइ जीव मोहि जबहीं & ।\\ \nopagebreak
जन्म कोटि अघ नासहिं तबहीं & ॥\\ \nopagebreak
\caption*{—रा.च.मा. ५-४४-२}
\end{longtable}
}
\begin{sloppypar}\justifying\hyphenrules{nohyphenation}
\noindent श्रीरघुनाथजी समस्त प्राणिमात्रके सम्मुख होकर भी, यथा—\textbf{सन्मुख सब~की ओर} (रा.च.मा. ३-१२), जीवकी सम्मुखताके बिना उसका कल्याण नहीं कर सकते। अतः गोस्वामीजी यह अनुरोध करते हैं कि अनादि कालसे भगवत्पाद-पद्म-विमुख इस जड़ जीवको श्रीमन्मारुतिके बिना कौन श्रीराम-सम्मुख कर सकता है? यथा—
\end{sloppypar}
{\bfseries
\setlength{\mylenone}{0pt}
\settowidth{\mylentwo}{आते आञ्जनेय न जो व्याकुल धरा पै आज}
\setlength{\mylenone}{\maxof{\mylenone}{\mylentwo}}
\settowidth{\mylentwo}{क्षुधित जनों को भक्ति-अमिय पिलाता कौन}
\setlength{\mylenone}{\maxof{\mylenone}{\mylentwo}}
\settowidth{\mylentwo}{कौन दरशाता रामधाम का पवित्र पंथ}
\setlength{\mylenone}{\maxof{\mylenone}{\mylentwo}}
\settowidth{\mylentwo}{राम-नाम-मञ्जु-मणिदीपक जलाता कौन}
\setlength{\mylenone}{\maxof{\mylenone}{\mylentwo}}
\settowidth{\mylentwo}{कौन सरसाता उर-भाव-सरसीरुह को}
\setlength{\mylenone}{\maxof{\mylenone}{\mylentwo}}
\settowidth{\mylentwo}{राम-प्रेम मधुर सुमोदक खिलाता कौन}
\setlength{\mylenone}{\maxof{\mylenone}{\mylentwo}}
\settowidth{\mylentwo}{राम-गुण गायक बनाता कौन गिरिधर को}
\setlength{\mylenone}{\maxof{\mylenone}{\mylentwo}}
\settowidth{\mylentwo}{मुझ-से पतित को पथ सुमति दिलाता कौन}
\setlength{\mylenone}{\maxof{\mylenone}{\mylentwo}}
\settowidth{\mylentwo}{आते आञ्जनेय जो न अमल अवनि पै आज}
\setlength{\mylenone}{\maxof{\mylenone}{\mylentwo}}
\settowidth{\mylentwo}{वैष्णवों की विजय-वैजयन्ती फहराता कौन}
\setlength{\mylenone}{\maxof{\mylenone}{\mylentwo}}
\settowidth{\mylentwo}{कौन लाँघ जाता शतयोजन पयोनिधि को}
\setlength{\mylenone}{\maxof{\mylenone}{\mylentwo}}
\settowidth{\mylentwo}{मैथिली का विरह-दवानल बुझाता कौन}
\setlength{\mylenone}{\maxof{\mylenone}{\mylentwo}}
\settowidth{\mylentwo}{कौन लिपटाता रघुबीर-पद-पङ्कज में}
\setlength{\mylenone}{\maxof{\mylenone}{\mylentwo}}
\settowidth{\mylentwo}{राजीव-नयन के नयन-नीर से नहाता कौन}
\setlength{\mylenone}{\maxof{\mylenone}{\mylentwo}}
\settowidth{\mylentwo}{साधन-विहीन दृगहीन मूढ़ गिरिधर को}
\setlength{\mylenone}{\maxof{\mylenone}{\mylentwo}}
\settowidth{\mylentwo}{मानस-मन्दाकिनी में मज्जन कराता कौन}
\setlength{\mylenone}{\maxof{\mylenone}{\mylentwo}}
\setlength{\mylentwo}{\baselineskip}
\setlength{\mylenone}{\mylenone + 1pt}
\setlength{\mylen}{(\textwidth - \mylenone)*\real{0.5}}
\begin{longtable}[l]{@{\hspace*{\mylen}}>{\setlength\parfillskip{0pt}}p{\mylenone}@{}@{}l@{}}
 & \\[-\the\mylentwo]
आते आञ्जनेय न जो व्याकुल धरा पै आज & \\ \nopagebreak
क्षुधित जनों को भक्ति-अमिय पिलाता कौन & ।\\
कौन दरशाता रामधाम का पवित्र पंथ & \\ \nopagebreak
राम-नाम-मञ्जु-मणिदीपक जलाता कौन & ।\\
कौन सरसाता उर-भाव-सरसीरुह को & \\ \nopagebreak
राम-प्रेम मधुर सुमोदक खिलाता कौन & ।\\
राम-गुण गायक बनाता कौन गिरिधर को & \\ \nopagebreak
मुझ-से पतित को पथ सुमति दिलाता कौन & ॥\\
आते आञ्जनेय जो न अमल अवनि पै आज & \\ \nopagebreak
वैष्णवों की विजय-वैजयन्ती फहराता कौन & ।\\
कौन लाँघ जाता शतयोजन पयोनिधि को & \\ \nopagebreak
मैथिली का विरह-दवानल बुझाता कौन & ।\\
कौन लिपटाता रघुबीर-पद-पङ्कज में & \\ \nopagebreak
राजीव-नयन के नयन-नीर से नहाता कौन & ।\\
साधन-विहीन दृगहीन मूढ़ गिरिधर को & \\ \nopagebreak
मानस-मन्दाकिनी में मज्जन कराता कौन & ॥\\
\end{longtable}
}
\pagebreak

\begin{sloppypar}\justifying\hyphenrules{nohyphenation}
\noindent इस प्रकार स्पष्ट है कि आञ्जनेयका चरित्र श्रीराघवजीकी लीलाका शृङ्गार है और यही \textbf{\textit{सँवारे}} पदका स्वारस्य है। यथा—\textbf{काज महाराज~के समाज सब साजे हैं} (ह.बा. १५) और \textbf{सकल समाज-साज साजे रघुबर~के} (ह.बा. ३३)।
\end{sloppypar}
\paraseplotus
\pagebreak


\fancyhead[LE,RO]{{\textmd{\large चौ. ११: लाय सँजीवनि लखन जियाये}}}
\phantomsection
\addcontentsline{toc}{section}{चौपाई ११: लाय सँजीवनि लखन जियाये}
\centering{॥ श्रीराम ॥}
\begin{sloppypar}\justifying\hyphenrules{nohyphenation}
मूल (चौपाई)—
\end{sloppypar}

{\bfseries\relscale{1.2}
\setlength{\mylenone}{0pt}
\settowidth{\mylentwo}{लाय सँजीवनि लखन जियाये}
\setlength{\mylenone}{\maxof{\mylenone}{\mylentwo}}
\settowidth{\mylentwo}{श्रीरघुबीर हरषि उर लाये}
\setlength{\mylenone}{\maxof{\mylenone}{\mylentwo}}
\setlength{\mylentwo}{\baselineskip}
\setlength{\mylenone}{\mylenone + 1pt}
\setlength{\mylen}{(\textwidth - \mylenone)*\real{0.5}}
\begin{longtable}[l]{@{\hspace*{\mylen}}>{\setlength\parfillskip{0pt}}p{\mylenone}@{}@{}l@{}}
 & \\[-\the\mylentwo]
लाय सँजीवनि लखन जियाये & ।\\ \nopagebreak[1mm]
श्रीरघुबीर हरषि उर लाये & ॥ ११ ॥
\end{longtable}
}

\parasepone
\index[ardha]{लायसँजीवनिलखनजियाये(चौ.११,पूर्वार्ध)@लाय सँजीवनि लखन जियाये (चौ. ११, पूर्वार्ध)}
\index[ardha]{श्रीरघुबीरहरषिउरलाये(चौ.११,उत्तरार्ध)@श्रीरघुबीर हरषि उर लाये (चौ. ११, उत्तरार्ध)}
\index[pada]{लाय@लाय}
\index[pada]{सँजीवनि@सँजीवनि}
\index[pada]{लखन@लखन}
\index[pada]{जियाये@जियाये}
\index[pada]{श्रीरघुबीर@श्रीरघुबीर}
\index[pada]{हरषि@हरषि}
\index[pada]{उर@उर}
\index[pada]{लाये@लाये}
\begin{sloppypar}\justifying\hyphenrules{nohyphenation}
\textbf{शब्दार्थ}—\textbf{\textit{सँजीवनि}} {\unifont{\relscale{0.7}▶}} द्रोणाचलसे लाई हुई मृतसंजीवनी।
\end{sloppypar}
\begin{sloppypar}\justifying\hyphenrules{nohyphenation}
\textbf{अर्थ}—हे पवननन्दन! आपने द्रोणाचलसे मृतसंजीवनी ले आकर श्रीलक्ष्मणको जिलाया तथा रघुवीर रामभद्रजूने प्रसन्नतासे आपको अपने हृदयसे लगा लिया।
\end{sloppypar}
\parasepone
\begin{sloppypar}\justifying\hyphenrules{nohyphenation}
\textbf{व्याख्या}—कालके विजेता मेघनादने कालस्वरूप श्रीलक्ष्मणको वीरघातिनी शक्तिसे मूर्च्छित किया। अनन्तर \textbf{करालं महाकाल\-कालं कृपालं} (रा.च.मा. ७-१०८-२) श्रीहनुमान्‌जी मूर्च्छित कालको कालातीत श्रीराघवके पास ले आए तथा उन्होंने सुषेणके निर्देशानुसार संजीवनी ले आकर श्रीलक्ष्मणको प्राणदान दिया एवं श्रीराघवके मङ्गलमय परिष्वङ्गका अनुभव किया। यथा—
\end{sloppypar}
{\bfseries
\setlength{\mylenone}{0pt}
\settowidth{\mylentwo}{हरषि राम भेंटेउ हनुमाना}
\setlength{\mylenone}{\maxof{\mylenone}{\mylentwo}}
\settowidth{\mylentwo}{अति कृतज्ञ प्रभु परम सुजाना}
\setlength{\mylenone}{\maxof{\mylenone}{\mylentwo}}
\setlength{\mylentwo}{\baselineskip}
\setlength{\mylenone}{\mylenone + 1pt}
\setlength{\mylen}{(\textwidth - \mylenone)*\real{0.5}}
\begin{longtable}[l]{@{\hspace*{\mylen}}>{\setlength\parfillskip{0pt}}p{\mylenone}@{}@{}l@{}}
 & \\[-\the\mylentwo]
हरषि राम भेंटेउ हनुमाना & ।\\ \nopagebreak
अति कृतज्ञ प्रभु परम सुजाना & ॥\\ \nopagebreak
\caption*{—रा.च.मा. ६-६२-१}
\end{longtable}
}
\paraseplotus
\pagebreak


\fancyhead[LE,RO]{{\textmd{\large चौ. १२: रघुपति कीन्ही बहुत बड़ाई}}}
\phantomsection
\addcontentsline{toc}{section}{चौपाई १२: रघुपति कीन्ही बहुत बड़ाई}
\centering{॥ श्रीराम ॥}
\begin{sloppypar}\justifying\hyphenrules{nohyphenation}
मूल (चौपाई)—
\end{sloppypar}

{\bfseries\relscale{1.2}
\setlength{\mylenone}{0pt}
\settowidth{\mylentwo}{रघुपति कीन्ही बहुत बड़ाई}
\setlength{\mylenone}{\maxof{\mylenone}{\mylentwo}}
\settowidth{\mylentwo}{तुम मम प्रिय भरतहिं सम भाई}
\setlength{\mylenone}{\maxof{\mylenone}{\mylentwo}}
\setlength{\mylentwo}{\baselineskip}
\setlength{\mylenone}{\mylenone + 1pt}
\setlength{\mylen}{(\textwidth - \mylenone)*\real{0.5}}
\begin{longtable}[l]{@{\hspace*{\mylen}}>{\setlength\parfillskip{0pt}}p{\mylenone}@{}@{}l@{}}
 & \\[-\the\mylentwo]
रघुपति कीन्ही बहुत बड़ाई & ।\\ \nopagebreak[1mm]
तुम मम प्रिय भरतहिं सम भाई & ॥ १२ ॥
\end{longtable}
}

\parasepone
\index[ardha]{रघुपतिकीन्हीबहुतबड़ाई(चौ.१२,पूर्वार्ध)@रघुपति कीन्ही बहुत बड़ाई (चौ. १२, पूर्वार्ध)}
\index[ardha]{तुमममप्रियभरतहिंसमभाई(चौ.१२,उत्तरार्ध)@तुम मम प्रिय भरतहिं सम भाई (चौ. १२, उत्तरार्ध)}
\index[pada]{रघुपति@रघुपति}
\index[pada]{कीन्ही@कीन्ही}
\index[pada]{बहुत@बहुत}
\index[pada]{बडाई@बड़ाई}
\index[pada]{तुम@तुम}
\index[pada]{मम@मम}
\index[pada]{प्रिय@प्रिय}
\index[pada]{भरतहिं@भरतहिं}
\index[pada]{सम@सम}
\index[pada]{भाई@भाई}
\begin{sloppypar}\justifying\hyphenrules{nohyphenation}
\textbf{शब्दार्थ}—\textbf{\textit{रघुपति}} {\unifont{\relscale{0.7}▶}} रघुवंशके स्वामी, अथवा \textit{रघु} अर्थात् जीवमात्रके स्वामी श्रीराम। \textit{लङ्घन्ते पापपुण्यानि ये ते रघवो जीवास्तेषां पतिः इति रघुपतिः}।
\end{sloppypar}
\begin{sloppypar}\justifying\hyphenrules{nohyphenation}
\textbf{अर्थ}—रघुकुलके स्वामी तथा समस्त प्राणिमात्रके ईश्वर श्रीरामचन्द्रजीने आपकी बड़ी प्रशंसा की और कहा कि तुम मुझे भाई भरतके समान प्रिय हो।
\end{sloppypar}
\parasepone
\begin{sloppypar}\justifying\hyphenrules{nohyphenation}
\textbf{व्याख्या}—\textbf{\textit{भाई}} शब्दका अन्वय \textbf{\textit{भरतहि}} शब्दके साथ ही उचित होगा, अर्थात् ‘भाई भरतके समान तुम मुझे प्रिय हो’। हनुमान्‌जीके साथ \textbf{\textit{भाई}}~शब्दका अन्वय करनेसे \textbf{सुनु सुत तोहि उरिन मैं नाहीं} (रा.च.मा. ५-३२-७)—इस अर्धालीकी एकवाक्यता नहीं संगत होगी क्योंकि यहाँ श्रीराघवने आञ्जनेयको पुत्र कहा। ध्यान रहे कि श्रीलक्ष्मणसे श्रीहनुमान्‌जीका इतना वैशिष्ट्य अवश्य है कि श्रीरघुनाथजी श्रीलक्ष्मणजीको भाई तथा पुत्र दोनों मानते हैं। भाई, यथा—\textbf{अस जिय जानि सुनहु सिख भाई} (रा.च.मा. २-७१-१)। पुत्र, यथा—\textbf{अब अपलोक शोक सुत तोरा} (रा.च.मा. ६-६१-१३)। पर हनुमान्‌जीको केवल पुत्र-रूपमें ही स्वीकारते हैं, यथा—\textbf{सिय-सुखदायक दुलारो रघुनायक को} (ह.बा. १०)।
\end{sloppypar}
\paraseplotus
\pagebreak


\fancyhead[LE,RO]{{\textmd{\large चौ. १३: सहसबदन तुम्हरो जस गावैं}}}
\phantomsection
\addcontentsline{toc}{section}{चौपाई १३: सहसबदन तुम्हरो जस गावैं}
\centering{॥ श्रीराम ॥}
\begin{sloppypar}\justifying\hyphenrules{nohyphenation}
मूल (चौपाई)—
\end{sloppypar}

{\bfseries\relscale{1.2}
\setlength{\mylenone}{0pt}
\settowidth{\mylentwo}{सहसबदन तुम्हरो जस गावैं}
\setlength{\mylenone}{\maxof{\mylenone}{\mylentwo}}
\settowidth{\mylentwo}{अस कहि श्रीपति कंठ लगावैं}
\setlength{\mylenone}{\maxof{\mylenone}{\mylentwo}}
\setlength{\mylentwo}{\baselineskip}
\setlength{\mylenone}{\mylenone + 1pt}
\setlength{\mylen}{(\textwidth - \mylenone)*\real{0.5}}
\begin{longtable}[l]{@{\hspace*{\mylen}}>{\setlength\parfillskip{0pt}}p{\mylenone}@{}@{}l@{}}
 & \\[-\the\mylentwo]
सहसबदन तुम्हरो जस गावैं & ।\\ \nopagebreak[1mm]
अस कहि श्रीपति कंठ लगावैं & ॥ १३ ॥
\end{longtable}
}

\parasepone
\index[ardha]{सहसबदनतुम्हरोजसगावैं(चौ.१३,पूर्वार्ध)@सहसबदन तुम्हरो जस गावैं (चौ. १३, पूर्वार्ध)}
\index[ardha]{असकहिश्रीपतिकंठलगावैं(चौ.१३,उत्तरार्ध)@अस कहि श्रीपति कंठ लगावैं (चौ. १३, उत्तरार्ध)}
\index[pada]{सहसबदन@सहसबदन}
\index[pada]{तुम्हरो@तुम्हरो}
\index[pada]{जस@जस}
\index[pada]{गावैं@गावैं}
\index[pada]{अस@अस}
\index[pada]{कहि@कहि}
\index[pada]{श्रीपति@श्रीपति}
\index[pada]{कंठ@कंठ}
\index[pada]{लगावैं@लगावैं}
\begin{sloppypar}\justifying\hyphenrules{nohyphenation}
\textbf{शब्दार्थ}—\textbf{\textit{सहसबदन}} {\unifont{\relscale{0.7}▶}} शेष। \textbf{\textit{श्रीपति}} {\unifont{\relscale{0.7}▶}} सीतापति श्रीराम।
\end{sloppypar}
\begin{sloppypar}\justifying\hyphenrules{nohyphenation}
\textbf{अर्थ}—सहस्र मुख वाले शेष तुम्हारा यश गाते हैं तथा गाते रहेंगे। ऐसा कहकर श्रीसीताके पति श्रीराम हनुमान्‌जीको बार-बार गलेसे लगा रहे हैं।
\end{sloppypar}
\parasepone
\begin{sloppypar}\justifying\hyphenrules{nohyphenation}
\textbf{व्याख्या}—श्रीहनुमान्‌जीकी यह ध्यान-झाँकी श्रीलक्ष्मण-मूर्च्छा-समाप्तिके पश्चात्‌-कालकी है। श्रीलक्ष्मणजीको मूर्च्छामुक्त देखकर उन्मुक्त कण्ठसे प्रशंसा कर श्रीराघवजीने हनुमान्‌जीको गलेसे लगा लिया। \textbf{\textit{सहसबदन}} यहाँ श्रीलक्ष्मणजीके लिए अभिप्रेत है। यथा—
\end{sloppypar}
{\bfseries
\setlength{\mylenone}{0pt}
\settowidth{\mylentwo}{शेष सहस्र शीष जग कारन}
\setlength{\mylenone}{\maxof{\mylenone}{\mylentwo}}
\settowidth{\mylentwo}{जो अवतरेउ भूमि भय दारन}
\setlength{\mylenone}{\maxof{\mylenone}{\mylentwo}}
\setlength{\mylentwo}{\baselineskip}
\setlength{\mylenone}{\mylenone + 1pt}
\setlength{\mylen}{(\textwidth - \mylenone)*\real{0.5}}
\begin{longtable}[l]{@{\hspace*{\mylen}}>{\setlength\parfillskip{0pt}}p{\mylenone}@{}@{}l@{}}
 & \\[-\the\mylentwo]
शेष सहस्र शीष जग कारन & ।\\ \nopagebreak
जो अवतरेउ भूमि भय दारन & ॥\\ \nopagebreak
\caption*{—रा.च.मा. १-१७-७}
\end{longtable}
}
\begin{sloppypar}\justifying\hyphenrules{nohyphenation}
\noindent भाव यह है कि आञ्जनेय! तुम्हारे इस परम पावन यशको सहस्रमुख शेषावतार श्रीलक्ष्मण भी गाते रहेंगे, क्योंकि रणशय्यापर शान्त हुए अनन्तको भी तुमने जीवनदान दिया।
\end{sloppypar}
\paraseplotus
\pagebreak


\fancyhead[LE,RO]{{\textmd{\large चौ. १४: सनकादिक ब्रह्मादि मुनीशा}}}
\phantomsection
\addcontentsline{toc}{section}{चौपाई १४: सनकादिक ब्रह्मादि मुनीशा}
\centering{॥ श्रीराम ॥}
\begin{sloppypar}\justifying\hyphenrules{nohyphenation}
मूल (चौपाई)—
\end{sloppypar}

{\bfseries\relscale{1.2}
\setlength{\mylenone}{0pt}
\settowidth{\mylentwo}{सनकादिक ब्रह्मादि मुनीशा}
\setlength{\mylenone}{\maxof{\mylenone}{\mylentwo}}
\settowidth{\mylentwo}{नारद सारद सहित अहीशा}
\setlength{\mylenone}{\maxof{\mylenone}{\mylentwo}}
\setlength{\mylentwo}{\baselineskip}
\setlength{\mylenone}{\mylenone + 1pt}
\setlength{\mylen}{(\textwidth - \mylenone)*\real{0.5}}
\begin{longtable}[l]{@{\hspace*{\mylen}}>{\setlength\parfillskip{0pt}}p{\mylenone}@{}@{}l@{}}
 & \\[-\the\mylentwo]
सनकादिक ब्रह्मादि मुनीशा & ।\\ \nopagebreak[1mm]
नारद सारद सहित अहीशा & ॥ १४ ॥
\end{longtable}
}

\parasepone
\index[ardha]{सनकादिकब्रह्मादिमुनीशा(चौ.१४,पूर्वार्ध)@सनकादिक ब्रह्मादि मुनीशा (चौ. १४, पूर्वार्ध)}
\index[ardha]{नारदसारदसहितअहीशा(चौ.१४,उत्तरार्ध)@नारद सारद सहित अहीशा (चौ. १४, उत्तरार्ध)}
\index[pada]{सनकादिक@सनकादिक}
\index[pada]{ब्रह्मादि@ब्रह्मादि}
\index[pada]{मुनीशा@मुनीशा}
\index[pada]{नारद@नारद}
\index[pada]{सारद@सारद}
\index[pada]{सहित@सहित}
\index[pada]{अहीशा@अहीशा}
\begin{sloppypar}\justifying\hyphenrules{nohyphenation}
\textbf{शब्दार्थ}—\textbf{\textit{सनकादिक}} {\unifont{\relscale{0.7}▶}} सनक, सनन्दन, सनातन, एवं सनत्कुमार (ये चार ब्रह्माजीके प्रथम ऊर्ध्वरेता पुत्र हैं)।
\end{sloppypar}
\begin{sloppypar}\justifying\hyphenrules{nohyphenation}
\textbf{अर्थ}—श्रीराघव प्रशंसाके शब्दोंमें कह रहे हैं, “हे वत्स! तुम्हारे इस परम पावन यशको न केवल शेष अपितु सनकादिक ऊर्ध्वरेता ऋषि, ब्रह्मादि देवगण, मुनियोंमें श्रेष्ठ भगवान् नारद, और सरस्वतीके सहित अहीश्वर (विष्णु व शंकर) भी गाते रहेंगे।”
\end{sloppypar}
\parasepone
\begin{sloppypar}\justifying\hyphenrules{nohyphenation}
\textbf{व्याख्या}—इस चौपाईमें भी पूर्व क्रिया \textbf{\textit{गावैं}}का अन्वय होगा। भाव यह है कि तुम्हारा लक्ष्मण-जीवनदान-रूप परम पावन यश त्रिलोकविदित हो जाएगा। अतः पातालमें शेष, मर्त्यलोकमें सनकादि एवं नारद, तथा स्वर्गलोकमें ब्रह्मादि, शारदा, एवं विष्णु तथा शंकर भी गाएँगे। पूर्वमें शेषके लिए \textbf{\textit{सहसबदन}} शब्दका प्रयोग हो चुका है। अतः यहाँ \textbf{\textit{अहीशा}} पद अहितल्पवासी विष्णु यद्वा अहिकौपीनधारी श्रीशिवका बोधक है। यथा विष्णु—\textbf{जौ अहि सेज शयन हरि करहीं} (रा.च.मा. १-६९-५)। शिव—\textbf{जटा मुकुट अहि मौर सँवारा} (रा.च.मा. १-९२-१), \textbf{कुंडल कंकन पहिरे ब्याला} (रा.च.मा. १-९२-२), \textbf{भुजग भूति भूषन त्रिपुरारी} (रा.च.मा. १-१०६-८) आदि।
\end{sloppypar}
\paraseplotus
\pagebreak


\fancyhead[LE,RO]{{\textmd{\large चौ. १५: जम कुबेर दिगपाल जहाँ ते}}}
\phantomsection
\addcontentsline{toc}{section}{चौपाई १५: जम कुबेर दिगपाल जहाँ ते}
\centering{॥ श्रीराम ॥}
\begin{sloppypar}\justifying\hyphenrules{nohyphenation}
मूल (चौपाई)—
\end{sloppypar}

{\bfseries\relscale{1.2}
\setlength{\mylenone}{0pt}
\settowidth{\mylentwo}{जम कुबेर दिगपाल जहाँ ते}
\setlength{\mylenone}{\maxof{\mylenone}{\mylentwo}}
\settowidth{\mylentwo}{कबि कोबिद कहि सकैं कहाँ ते}
\setlength{\mylenone}{\maxof{\mylenone}{\mylentwo}}
\setlength{\mylentwo}{\baselineskip}
\setlength{\mylenone}{\mylenone + 1pt}
\setlength{\mylen}{(\textwidth - \mylenone)*\real{0.5}}
\begin{longtable}[l]{@{\hspace*{\mylen}}>{\setlength\parfillskip{0pt}}p{\mylenone}@{}@{}l@{}}
 & \\[-\the\mylentwo]
जम कुबेर दिगपाल जहाँ ते & ।\\ \nopagebreak[1mm]
कबि कोबिद कहि सकैं कहाँ ते & ॥ १५ ॥
\end{longtable}
}

\parasepone
\index[ardha]{जमकुबेरदिगपालजहाँते(चौ.१५,पूर्वार्ध)@जम कुबेर दिगपाल जहाँ ते (चौ. १५, पूर्वार्ध)}
\index[ardha]{कबिकोबिदकहिसकैंकहाँते(चौ.१५,उत्तरार्ध)@कबि कोबिद कहि सकैं कहाँ ते (चौ. १५, उत्तरार्ध)}
\index[pada]{जम@जम}
\index[pada]{कुबेर@कुबेर}
\index[pada]{दिगपाल@दिगपाल}
\index[pada]{जहाँ@जहाँ}
\index[pada]{ते@ते}
\index[pada]{कबि@कबि}
\index[pada]{कोबिद@कोबिद}
\index[pada]{कहि@कहि}
\index[pada]{सकैं@सकैं}
\index[pada]{कहाँ@कहाँ}
\index[pada]{ते@ते}
\begin{sloppypar}\justifying\hyphenrules{nohyphenation}
\textbf{शब्दार्थ}—\textbf{\textit{कोबिद}} (\textit{कोविद}) {\unifont{\relscale{0.7}▶}} वेदज्ञ। \textit{कोर्वेदस्य विदो वेत्ता कोविदः परिकीर्तितः}। \textit{को} अर्थात् वेदका \textit{विद} अर्थात् जाननेवाला। इस प्रकार \textit{कोविद}{\unifont{\relscale{0.7}=}}वेदज्ञ।
\end{sloppypar}
\begin{sloppypar}\justifying\hyphenrules{nohyphenation}
\textbf{अर्थ}—यम, कुबेर आदि यावन्मात्र दिक्पाल हैं, वे भी तुम्हारा यह यश गाते रहेंगे। इस अनन्त यशको सामान्य कवि एवं वेदज्ञ विद्वान् कहाँसे कह सकते हैं?
\end{sloppypar}
\parasepone
\begin{sloppypar}\justifying\hyphenrules{nohyphenation}
\textbf{व्याख्या}—यहाँ \textbf{\textit{जहाँ ते}} शब्दके साथ पूर्व क्रिया \textbf{\textit{गावैं}}का अन्वय होगा। \textit{श्रीहनुमान्‌-चालीसा}की प्रथम चौपाईसे लेकर दसवीं चौपाई तक गोस्वामीजीने श्रीरामजीके वात्सल्य-भाजन श्रीआञ्जनेयके मङ्गलमय स्वरूप तथा गुणका वर्णन किया। अनन्तर ग्यारहवीं चौपाईसे बीसवीं चौपाई तक श्रीहनुमान्‌जीके श्रीराघव-यशोभूमिका रूप चारु चरित्रका वर्णन कर रहे हैं। जिसमें ग्यारहवीं चौपाईसे पन्द्रहवीं चौपाई तक लक्ष्मणमूर्च्छा-प्रसंगमें प्रस्तुतकी हुई हनुमान्‌जीकी महत्तम भूमिकाका वर्णन है। मानो यही पाँच चौपाइयाँ पञ्चाक्षर महामन्त्रके तात्पर्यके रूपमें कही गई हैं।
\end{sloppypar}
\begin{sloppypar}\justifying\hyphenrules{nohyphenation}
लक्ष्मणमूर्च्छा-प्रसंगका दार्शनिक तात्पर्य बड़ा मनोरञ्जक तथा साभिप्राय है। जैसे मेघनादकी शक्तिसे मूर्च्छित श्रीलक्ष्मणको आञ्जनेयजीने द्रोणाचलसे संजीवनी लाकर जीवनदान दिया, उसी प्रकार आसक्ति-रूप वीरघातिनीसे मूर्च्छित हम जीवोंको वैराग्यवान् सन्त रामनाम-रूप हनुमान्‌जी सद्गुरु सुषेण वैद्यकी अनुमतिसे वेद-पुराण-रूप द्रोणाचलमें वर्तमान, रामभक्ति-रूप संजीवनी ले आकर श्रीरामतत्त्व-रूप जीवनदान देते रहें, इसी उद्देश्यसे \textit{श्रीहनुमान्‌-चालीसा}में यह प्रसंग निबद्ध किया गया।
\end{sloppypar}
\begin{sloppypar}\justifying\hyphenrules{nohyphenation}
दिक्पाल आठ हैं (इन्द्र, ईशान, कुबेर, अग्नि, वरुण, वायु, यम, और नैर्ऋत्य)।
\end{sloppypar}
\paraseplotus
\pagebreak


\fancyhead[LE,RO]{{\textmd{\large चौ. १६: तुम उपकार सुग्रीवहिं कीन्हा}}}
\phantomsection
\addcontentsline{toc}{section}{चौपाई १६: तुम उपकार सुग्रीवहिं कीन्हा}
\centering{॥ श्रीराम ॥}
\begin{sloppypar}\justifying\hyphenrules{nohyphenation}
मूल (चौपाई)—
\end{sloppypar}

{\bfseries\relscale{1.2}
\setlength{\mylenone}{0pt}
\settowidth{\mylentwo}{तुम उपकार सुग्रीवहिं कीन्हा}
\setlength{\mylenone}{\maxof{\mylenone}{\mylentwo}}
\settowidth{\mylentwo}{राम मिलाय राज-पद दीन्हा}
\setlength{\mylenone}{\maxof{\mylenone}{\mylentwo}}
\setlength{\mylentwo}{\baselineskip}
\setlength{\mylenone}{\mylenone + 1pt}
\setlength{\mylen}{(\textwidth - \mylenone)*\real{0.5}}
\begin{longtable}[l]{@{\hspace*{\mylen}}>{\setlength\parfillskip{0pt}}p{\mylenone}@{}@{}l@{}}
 & \\[-\the\mylentwo]
तुम उपकार सुग्रीवहिं कीन्हा & ।\\ \nopagebreak[1mm]
राम मिलाय राज-पद दीन्हा & ॥ १६ ॥
\end{longtable}
}

\parasepone
\index[ardha]{तुमउपकारसुग्रीवहिंकीन्हा(चौ.१६,पूर्वार्ध)@तुम उपकार सुग्रीवहिं कीन्हा (चौ. १६, पूर्वार्ध)}
\index[ardha]{राममिलायराजपददीन्हा(चौ.१६,उत्तरार्ध)@राम मिलाय राज-पद दीन्हा (चौ. १६, उत्तरार्ध)}
\index[pada]{तुम@तुम}
\index[pada]{उपकार@उपकार}
\index[pada]{सुग्रीवहिं@सुग्रीवहिं}
\index[pada]{कीन्हा@कीन्हा}
\index[pada]{राम@राम}
\index[pada]{मिलाय@मिलाय}
\index[pada]{राज@राज}
\index[pada]{पद@पद}
\index[pada]{दीन्हा@दीन्हा}
\begin{sloppypar}\justifying\hyphenrules{nohyphenation}
\textbf{शब्दार्थ}—\textbf{\textit{उपकार}} {\unifont{\relscale{0.7}▶}} भलाई।
\end{sloppypar}
\begin{sloppypar}\justifying\hyphenrules{nohyphenation}
\textbf{अर्थ}—आपने सुग्रीवका महान् उपकार किया तथा उन्हें श्रीरामजीका दर्शन कराकर किष्किन्धाका साम्राज्य दे दिया।
\end{sloppypar}
\parasepone
\begin{sloppypar}\justifying\hyphenrules{nohyphenation}
\textbf{व्याख्या}—भाव यह है कि आपके बिना सुग्रीव कुछ भी नहीं कर पाते। पहले उन्हें श्रीरामजीको देखकर भय हुआ, पर उन्होंने जब आपकी पीठपर बैठे हुए श्रीरामजीको देखा तभी सम्यक् दर्शन हुआ। क्योंकि परमात्माका सम्यक् दर्शन सन्त-दृष्टिके बिना संभव नहीं है। यथा—
\end{sloppypar}
{\bfseries
\setlength{\mylenone}{0pt}
\settowidth{\mylentwo}{तहँ रह सचिव सहित सुग्रीवा}
\setlength{\mylenone}{\maxof{\mylenone}{\mylentwo}}
\settowidth{\mylentwo}{आवत देखि अतुल-बल-सींवा}
\setlength{\mylenone}{\maxof{\mylenone}{\mylentwo}}
\settowidth{\mylentwo}{अति सभीत कह सुनु हनुमाना}
\setlength{\mylenone}{\maxof{\mylenone}{\mylentwo}}
\settowidth{\mylentwo}{पुरुष जुगल बल-रूप-निधाना}
\setlength{\mylenone}{\maxof{\mylenone}{\mylentwo}}
\setlength{\mylentwo}{\baselineskip}
\setlength{\mylenone}{\mylenone + 1pt}
\setlength{\mylen}{(\textwidth - \mylenone)*\real{0.5}}
\begin{longtable}[l]{@{\hspace*{\mylen}}>{\setlength\parfillskip{0pt}}p{\mylenone}@{}@{}l@{}}
 & \\[-\the\mylentwo]
तहँ रह सचिव सहित सुग्रीवा & ।\\ \nopagebreak
आवत देखि अतुल-बल-सींवा & ॥\\
अति सभीत कह सुनु हनुमाना & ।\\ \nopagebreak
पुरुष जुगल बल-रूप-निधाना & ॥\\ \nopagebreak
\caption*{—रा.च.मा. ४-१-२,३}
\end{longtable}
}
\begin{sloppypar}\justifying\hyphenrules{nohyphenation}
\noindent पश्चात् सम्यक् दर्शन—
\end{sloppypar}
{\bfseries
\setlength{\mylenone}{0pt}
\settowidth{\mylentwo}{जब सुग्रीव राम कहँ देखा}
\setlength{\mylenone}{\maxof{\mylenone}{\mylentwo}}
\settowidth{\mylentwo}{अतिशय जन्म धन्य करि लेखा}
\setlength{\mylenone}{\maxof{\mylenone}{\mylentwo}}
\setlength{\mylentwo}{\baselineskip}
\setlength{\mylenone}{\mylenone + 1pt}
\setlength{\mylen}{(\textwidth - \mylenone)*\real{0.5}}
\begin{longtable}[l]{@{\hspace*{\mylen}}>{\setlength\parfillskip{0pt}}p{\mylenone}@{}@{}l@{}}
 & \\[-\the\mylentwo]
जब सुग्रीव राम कहँ देखा & ।\\ \nopagebreak
अतिशय जन्म धन्य करि लेखा & ॥\\ \nopagebreak
\caption*{—रा.च.मा. ४-४-६}
\end{longtable}
}
\begin{sloppypar}\justifying\hyphenrules{nohyphenation}
\textbf{\textit{राम मिलाय}}—श्रीरामजीसे मिलनेकी सुग्रीवमें कोई योग्यता नहीं थी, पर आपने अपनी विशेष कृपाके आधारपर श्रीरामजीको सुग्रीवके पास ले जाकर उन्हें कृतकृत्य किया। अतः आञ्जनेयको \textbf{सुग्रीवदुःखैकबन्धु} (वि.प. २७-२) कहा गया है।
\end{sloppypar}
\begin{sloppypar}\justifying\hyphenrules{nohyphenation}
\textbf{\textit{राज-पद दीन्हा}}—भाव यह है कि आपने सुग्रीवको राजपद एवं रामपद देकर मुक्ति तथा भुक्ति दोनोंका अधिकारी बना दिया है। वैसी ही कृपा हम असमर्थ जीवोंपर भी करें।
\end{sloppypar}
\paraseplotus
\pagebreak


\fancyhead[LE,RO]{{\textmd{\large चौ. १७: तुम्हरो मंत्र बिभीषन माना}}}
\phantomsection
\addcontentsline{toc}{section}{चौपाई १७: तुम्हरो मंत्र बिभीषन माना}
\centering{॥ श्रीराम ॥}
\begin{sloppypar}\justifying\hyphenrules{nohyphenation}
मूल (चौपाई)—
\end{sloppypar}

{\bfseries\relscale{1.2}
\setlength{\mylenone}{0pt}
\settowidth{\mylentwo}{तुम्हरो मंत्र बिभीषन माना}
\setlength{\mylenone}{\maxof{\mylenone}{\mylentwo}}
\settowidth{\mylentwo}{लंकेश्वर भए सब जग जाना}
\setlength{\mylenone}{\maxof{\mylenone}{\mylentwo}}
\setlength{\mylentwo}{\baselineskip}
\setlength{\mylenone}{\mylenone + 1pt}
\setlength{\mylen}{(\textwidth - \mylenone)*\real{0.5}}
\begin{longtable}[l]{@{\hspace*{\mylen}}>{\setlength\parfillskip{0pt}}p{\mylenone}@{}@{}l@{}}
 & \\[-\the\mylentwo]
तुम्हरो मंत्र बिभीषन माना & ।\\ \nopagebreak[1mm]
लंकेश्वर भए सब जग जाना & ॥ १७ ॥
\end{longtable}
}

\parasepone
\index[ardha]{तुम्हरोमंत्रबिभीषनमाना(चौ.१७,पूर्वार्ध)@तुम्हरो मंत्र बिभीषन माना (चौ. १७, पूर्वार्ध)}
\index[ardha]{लंकेश्वरभएसबजगजाना(चौ.१७,उत्तरार्ध)@लंकेश्वर भए सब जग जाना (चौ. १७, उत्तरार्ध)}
\index[pada]{तुम्हरो@तुम्हरो}
\index[pada]{मंत्र@मंत्र}
\index[pada]{बिभीषन@बिभीषन}
\index[pada]{माना@माना}
\index[pada]{लंकेश्वर@लंकेश्वर}
\index[pada]{भए@भए}
\index[pada]{सब@सब}
\index[pada]{जग@जग}
\index[pada]{जाना@जाना}
\begin{sloppypar}\justifying\hyphenrules{nohyphenation}
\textbf{शब्दार्थ}—\textbf{\textit{मंत्र}} {\unifont{\relscale{0.7}▶}} भगवत्प्रपत्ति\-सिद्धान्त।
\end{sloppypar}
\begin{sloppypar}\justifying\hyphenrules{nohyphenation}
\textbf{अर्थ}—हे आञ्जनेय! विभीषणने आपके रामप्रेम-रूप मूलमन्त्रको स्वीकारा। उसके परिणाम\-स्वरूप वे लङ्काके कल्पान्त शासक स्वामी बन गए। यह सारा संसार जानता है।
\end{sloppypar}
\parasepone
\begin{sloppypar}\justifying\hyphenrules{nohyphenation}
\textbf{व्याख्या}—विभीषणने लङ्कामें अपर रात्रिकालके प्रथम साक्षात्कारके अनन्तर निराशा भरे स्वरमें आञ्जनेयसे कहा—
\end{sloppypar}
{\bfseries
\setlength{\mylenone}{0pt}
\settowidth{\mylentwo}{तात कबहुँ मोहि जानि अनाथा}
\setlength{\mylenone}{\maxof{\mylenone}{\mylentwo}}
\settowidth{\mylentwo}{करिहैं कृपा भानुकुल-नाथा}
\setlength{\mylenone}{\maxof{\mylenone}{\mylentwo}}
\setlength{\mylentwo}{\baselineskip}
\setlength{\mylenone}{\mylenone + 1pt}
\setlength{\mylen}{(\textwidth - \mylenone)*\real{0.5}}
\begin{longtable}[l]{@{\hspace*{\mylen}}>{\setlength\parfillskip{0pt}}p{\mylenone}@{}@{}l@{}}
 & \\[-\the\mylentwo]
तात कबहुँ मोहि जानि अनाथा & ।\\ \nopagebreak
करिहैं कृपा भानुकुल-नाथा & ॥\\ \nopagebreak
\caption*{—रा.च.मा. ५-७-२}
\end{longtable}
}
\begin{sloppypar}\justifying\hyphenrules{nohyphenation}
\noindent अर्थात् भानुकुलनाथ श्रीरामने भानुपुत्र सुग्रीवपर कृपाकी क्योंकि वह उनके कुलप्रवर्तकका पुत्र है, पर मुझमें कोई पात्रता नहीं है। श्रीआञ्जनेयने कहा, “विभीषण! भगवत्स्मरण ही उनकी कृपाका एकमात्र असाधारण साधन है।” यथा—
\end{sloppypar}
{\bfseries
\setlength{\mylenone}{0pt}
\settowidth{\mylentwo}{जानतहूँ अस स्वामि बिसारी}
\setlength{\mylenone}{\maxof{\mylenone}{\mylentwo}}
\settowidth{\mylentwo}{फिरहिं ते काहे न होहिं दुखारी}
\setlength{\mylenone}{\maxof{\mylenone}{\mylentwo}}
\setlength{\mylentwo}{\baselineskip}
\setlength{\mylenone}{\mylenone + 1pt}
\setlength{\mylen}{(\textwidth - \mylenone)*\real{0.5}}
\begin{longtable}[l]{@{\hspace*{\mylen}}>{\setlength\parfillskip{0pt}}p{\mylenone}@{}@{}l@{}}
 & \\[-\the\mylentwo]
जानतहूँ अस स्वामि बिसारी & ।\\ \nopagebreak
फिरहिं ते काहे न होहिं दुखारी & ॥\\ \nopagebreak
\caption*{—रा.च.मा. ५-८-१}
\end{longtable}
}
\begin{sloppypar}\justifying\hyphenrules{nohyphenation}
\noindent इसी मन्त्रने विभीषणको लङ्केश्वर बना दिया। अथवा, श्रीहनुमान्‌जीने विभीषणजीसे कहा, “विभीषण! पिताके बिना तुम अधूरे हो और माताके बिना मैं अधूरा हूँ। तुम मुझे सीता माताके दर्शन करा दो और मैं तुम्हें पिता श्रीरामके दर्शन करा दूँगा।” यथा—
\end{sloppypar}
{\bfseries
\setlength{\mylenone}{0pt}
\settowidth{\mylentwo}{तब हनुमंत कहा सुनु भ्राता}
\setlength{\mylenone}{\maxof{\mylenone}{\mylentwo}}
\settowidth{\mylentwo}{देखी चहउँ जानकी माता}
\setlength{\mylenone}{\maxof{\mylenone}{\mylentwo}}
\setlength{\mylentwo}{\baselineskip}
\setlength{\mylenone}{\mylenone + 1pt}
\setlength{\mylen}{(\textwidth - \mylenone)*\real{0.5}}
\begin{longtable}[l]{@{\hspace*{\mylen}}>{\setlength\parfillskip{0pt}}p{\mylenone}@{}@{}l@{}}
 & \\[-\the\mylentwo]
तब हनुमंत कहा सुनु भ्राता & ।\\ \nopagebreak
देखी चहउँ जानकी माता & ॥\\ \nopagebreak
\caption*{—रा.च.मा. ५-८-४}
\end{longtable}
}
\begin{sloppypar}\justifying\hyphenrules{nohyphenation}
\noindent तब विभीषणने एक युक्ति बताई कि आप मेरा अर्थात् विभीषणका रूप धारण करके अशोक\-वाटिकामें प्रवेश करें। आपको कोई भी राक्षस पहचान नहीं सकेगा क्योंकि अशोक\-वाटिकामें मैं अर्थात् विभीषण तथा रावण ये दो ही पुरुष जा सकते हैं। हनुमान्‌जीने वैसा ही किया—
\end{sloppypar}
{\bfseries
\setlength{\mylenone}{0pt}
\settowidth{\mylentwo}{जुगुति बिभीषन सकल सुनाई}
\setlength{\mylenone}{\maxof{\mylenone}{\mylentwo}}
\settowidth{\mylentwo}{चलेउ पवनसुत बिदा कराई}
\setlength{\mylenone}{\maxof{\mylenone}{\mylentwo}}
\settowidth{\mylentwo}{धरि सोइ रूप गयउ पुनि तहवाँ}
\setlength{\mylenone}{\maxof{\mylenone}{\mylentwo}}
\settowidth{\mylentwo}{बन अशोक सीता रह जहवाँ}
\setlength{\mylenone}{\maxof{\mylenone}{\mylentwo}}
\setlength{\mylentwo}{\baselineskip}
\setlength{\mylenone}{\mylenone + 1pt}
\setlength{\mylen}{(\textwidth - \mylenone)*\real{0.5}}
\begin{longtable}[l]{@{\hspace*{\mylen}}>{\setlength\parfillskip{0pt}}p{\mylenone}@{}@{}l@{}}
 & \\[-\the\mylentwo]
जुगुति बिभीषन सकल सुनाई & ।\\ \nopagebreak
चलेउ पवनसुत बिदा कराई & ॥\\
धरि सोइ रूप गयउ पुनि तहवाँ & ।\\ \nopagebreak
बन अशोक सीता रह जहवाँ & ॥\\ \nopagebreak
\caption*{—रा.च.मा. ५-८-५,६}
\end{longtable}
}
\begin{sloppypar}\justifying\hyphenrules{nohyphenation}
\noindent विभीषणके इसी मन्त्रने (जिसे हनुमान्‌जीने माना) हनुमान्‌जीको सीताजीके दर्शन करा दिये। उसके बदले हनुमान्‌जीने विभीषणको श्रीरामजीके दर्शन कराकर उन्हें लङ्काधिपति बना दिया। इसीलिए गोस्वामीजीने कहा—\textbf{जयति भुवनैकभूषण विभीषणवरद} (वि.प. २६-६)।
\end{sloppypar}
\begin{sloppypar}\justifying\hyphenrules{nohyphenation}
हनुमान्‌जीने सुग्रीव तथा विभीषण इन दो महानुभावोंको भगवान्‌से जोड़ा। एकके यहाँ प्रभुको ले गए और एकको प्रभुके पास ले आए। सुग्रीवको प्रभुका प्रभाव सुनाकर एवं विभीषणको प्रभुका स्वभाव समझाकर। उसी प्रकार हम विषयी साधकोंको भी प्रभु-कृपाका अनुभव कराएँ।
\end{sloppypar}
\paraseplotus
\pagebreak


\fancyhead[LE,RO]{{\textmd{\large चौ. १८: जुग सहस्र जोजन पर भानू}}}
\phantomsection
\addcontentsline{toc}{section}{चौपाई १८: जुग सहस्र जोजन पर भानू}
\centering{॥ श्रीराम ॥}
\begin{sloppypar}\justifying\hyphenrules{nohyphenation}
मूल (चौपाई)—
\end{sloppypar}

{\bfseries\relscale{1.2}
\setlength{\mylenone}{0pt}
\settowidth{\mylentwo}{जुग सहस्र जोजन पर भानू}
\setlength{\mylenone}{\maxof{\mylenone}{\mylentwo}}
\settowidth{\mylentwo}{लील्यो ताहि मधुर फल जानू}
\setlength{\mylenone}{\maxof{\mylenone}{\mylentwo}}
\setlength{\mylentwo}{\baselineskip}
\setlength{\mylenone}{\mylenone + 1pt}
\setlength{\mylen}{(\textwidth - \mylenone)*\real{0.5}}
\begin{longtable}[l]{@{\hspace*{\mylen}}>{\setlength\parfillskip{0pt}}p{\mylenone}@{}@{}l@{}}
 & \\[-\the\mylentwo]
जुग सहस्र जोजन पर भानू & ।\\ \nopagebreak[1mm]
लील्यो ताहि मधुर फल जानू & ॥ १८ ॥
\end{longtable}
}

\parasepone
\index[ardha]{जुगसहस्रजोजनपरभानू(चौ.१८,पूर्वार्ध)@जुग सहस्र जोजन पर भानू (चौ. १८, पूर्वार्ध)}
\index[ardha]{लील्योताहिमधुरफलजानू(चौ.१८,उत्तरार्ध)@लील्यो ताहि मधुर फल जानू (चौ. १८, उत्तरार्ध)}
\index[pada]{जुग@जुग}
\index[pada]{सहस्र@सहस्र}
\index[pada]{जोजन@जोजन}
\index[pada]{पर@पर}
\index[pada]{भानू@भानू}
\index[pada]{लील्यो@लील्यो}
\index[pada]{ताहि@ताहि}
\index[pada]{मधुर@मधुर}
\index[pada]{फल@फल}
\index[pada]{जानू@जानू}
\begin{sloppypar}\justifying\hyphenrules{nohyphenation}
\textbf{शब्दार्थ}—\textbf{\textit{जुग सहस्र जोजन}} {\unifont{\relscale{0.7}▶}} \textit{जुग} अर्थात् अनेक हजार योजन। \textbf{\textit{भानू}} {\unifont{\relscale{0.7}▶}} सूर्य।
\end{sloppypar}
\begin{sloppypar}\justifying\hyphenrules{nohyphenation}
\textbf{अर्थ}—हे केसरीकुमार! धरातलसे हजारों योजन किंवा अनेकों हजार योजन दूर ऊपर वर्तमान सूर्यनारायणको आपने अपने जन्मके एक दिन बाद मधुर फलकी भ्रान्तिसे निगल लिया था।
\end{sloppypar}
\parasepone
\begin{sloppypar}\justifying\hyphenrules{nohyphenation}
\textbf{व्याख्या}—\textit{युग} शब्दका \textit{युगल}-पर्याय होनेसे ‘दो-दो’ एवं ‘एकसे अनेक’ अर्थ भी होता है। वस्तुतः संस्कृतमें शतसे ऊपरकी संख्याएँ अनन्तवाचिका होती हैं। यथा—\textbf{शताधिकाः समाः संख्या गेयाश्चानन्त्यवाचिकाः}। इस दृष्टिसे \textbf{\textit{जुग सहस्र जोजन}} का अर्थ होगा ‘अगणित योजन’। यह घटना संभवतः कार्त्तिक कृष्ण अमावस्याकी है। अमावस्याको ही अपनी सन्धिमें राहु सूर्यग्रहणकी परिस्थिति प्रस्तुत करता है। श्रीहनुमान्‌जीका प्राकट्य कार्त्तिक कृष्ण चतुर्दशी मङ्गलवारको प्रभात-वेलामें मेष लग्न तथा स्वाति नक्षत्रमें श्रीअञ्जनाके गर्भसे हुआ था। यथा—
\end{sloppypar}
{\bfseries
\setlength{\mylenone}{0pt}
\settowidth{\mylentwo}{ऊर्जे कृष्णचतुर्दश्यां भौमे स्वात्यां कपीश्वरः}
\setlength{\mylenone}{\maxof{\mylenone}{\mylentwo}}
\settowidth{\mylentwo}{मेषलग्नेऽञ्जनागर्भात्प्रादुर्भूतो स्वयं शिवः}
\setlength{\mylenone}{\maxof{\mylenone}{\mylentwo}}
\setlength{\mylentwo}{\baselineskip}
\setlength{\mylenone}{\mylenone + 1pt}
\setlength{\mylen}{(\textwidth - \mylenone)*\real{0.5}}
\begin{longtable}[l]{@{\hspace*{\mylen}}>{\setlength\parfillskip{0pt}}p{\mylenone}@{}@{}l@{}}
 & \\[-\the\mylentwo]
ऊर्जे कृष्णचतुर्दश्यां भौमे स्वात्यां कपीश्वरः & ।\\ \nopagebreak
मेषलग्नेऽञ्जनागर्भात्प्रादुर्भूतो स्वयं शिवः & ॥\\ \nopagebreak
\caption*{—अ.सं.}
\end{longtable}
}
\begin{sloppypar}\justifying\hyphenrules{nohyphenation}
\noindent वाल्मीकिके अनुसार सूर्यनारायणपर आञ्जनेयजीके आक्रमण ही की चर्चा है तथा इन्द्रके द्वारा इनके वाम हनु (ठोड़ी) पर वज्रका प्रहार हुआ, पर उसमें कोई विकृति नहीं आई। इसलिए प्राशस्त्य अर्थमें \textit{मतुप्} प्रत्यय करके परम पराक्रमी इन्द्रने इनका नाम \textit{हनुमान्} रखा। कुछ लोग \textbf{वामो हनुरभज्यत} (वा.रा. ४-६५-२२) का अर्थ करते हैं कि \textit{वामो हनुर्भग्नोऽभवत्} अर्थात् हनुमान्‌जीका ‘वाम हनु किञ्चित् भङ्ग हो गया’। यह अर्थ उन्होंने \textit{भञ्ज्}~धातुसे (\textbf{भञ्जोँ आमर्दने}, धा.पा. १४५३) निष्पन्न \textit{अभज्यत} शब्दके आधारपर किया है। पर \textit{अभज्यत} रूप \textit{भज्}~धातुसे (\textbf{भजँ सेवायाम्}, धा.पा. ९९८) कर्मवाच्यमें लङ्लकारके प्रथमपुरुषके एकवचनमें भी निष्पन्न होता है। \textit{अभज्यत असेव्यत सेवितोऽभवत्}, अर्थात् इन्द्रके वज्रसे हनुमान्‌जीका वाम हनु सेवित हुआ, टूटा नहीं, इसलिए इन वानरका आजसे \textit{हनुमान्} नाम विख्यात होगा। क्योंकि यदि श्रीहनुमान्‌का हनु टूट गया होता, तब \textit{हनुमान्} शब्दमें \textit{मतुप्} प्रत्यय कैसे निष्पन्न होता? क्योंकि \textit{मतुप्} प्रत्यय सत्ता और प्रशंसा अर्थमें होता है—
\end{sloppypar}
{\bfseries
\setlength{\mylenone}{0pt}
\settowidth{\mylentwo}{भूमनिन्दाप्रशंसासु नित्ययोगेऽतिशायने}
\setlength{\mylenone}{\maxof{\mylenone}{\mylentwo}}
\settowidth{\mylentwo}{सम्बन्धेऽस्तिविवक्षायां भवन्ति मतुबादयः}
\setlength{\mylenone}{\maxof{\mylenone}{\mylentwo}}
\setlength{\mylentwo}{\baselineskip}
\setlength{\mylenone}{\mylenone + 1pt}
\setlength{\mylen}{(\textwidth - \mylenone)*\real{0.5}}
\begin{longtable}[l]{@{\hspace*{\mylen}}>{\setlength\parfillskip{0pt}}p{\mylenone}@{}@{}l@{}}
 & \\[-\the\mylentwo]
भूमनिन्दाप्रशंसासु नित्ययोगेऽतिशायने & ।\\ \nopagebreak
सम्बन्धेऽस्तिविवक्षायां भवन्ति मतुबादयः & ॥\\ \nopagebreak
\caption*{—भा.पा.सू. ५-२-९४}
\end{longtable}
}
\begin{sloppypar}\justifying\hyphenrules{nohyphenation}
\noindent भाष्यकारके उदाहरणोंके अनुसार निन्दा अर्थमें \textit{इनि} प्रत्यय होता है, तथा प्रशंसा अर्थमें \textit{मतुप्} ही होता है। जैसे किसी निर्धनको \textit{धनवान्} नहीं कहा जा सकता, उसी प्रकार टूटे हुए हनु वाले व्यक्तिको \textit{हनुमान्} कैसे कहा जाएगा? \textit{विनयपत्रिका}में भी गोस्वामीजी इसी सिद्धान्तकी पुष्टि करते हैं, यथा—\textbf{जाकी चिबुक-चोट चूरन किए रद-मद कुलिश कठोर~को} (वि.प. ३१-४)।
\end{sloppypar}
\begin{sloppypar}\justifying\hyphenrules{nohyphenation}
इस प्रकार स्पष्ट है कि \textit{वाल्मीकीय-रामायण}के अनुसार हनुमान्‌जीने सूर्यको निगला नहीं, पर \textit{हनुमान्‌-चालीसा}में \textbf{\textit{लील्यो ताहि मधुर फल जानू}} कह रहे हैं। इस पक्षकी पुष्टि गोस्वामीजी \textit{विनयपत्रिका}में करते हैं—\textbf{चंडकर-मंडल-ग्रासकर्त्ता} (वि.प. २५-२)। इस विरोधका समाधान कल्पभेदसे हो जाता है। वाल्मीकि-कथामें हनुमान्‌जीने सूर्यनारायणको नहीं ग्रसा था, \textit{विनयपत्रिका} तथा \textit{हनुमान्‌-चालीसा}के घटना-कल्पमें ग्रस लिया था। देवताओंकी विनतीपर छोड़ा। सूर्यनारायणको ग्रसना असंभव नहीं है क्योंकि कार्य कारणमें विलीन होते हैं। अतः तेजस्तत्त्व-रूप सूर्य वायुतत्त्व-रूप हनुमान्‌में विलीन हों, यह अत्यन्त उचित है।
\end{sloppypar}


\begin{sloppypar}\justifying\hyphenrules{nohyphenation}
\textbf{विशेष}—मनुष्योंका एक चतुर्युग—जिसमें संध्या और संध्यांश सहित १२,००० देववर्ष होते हैं—देवोंका एक युग कहलाता है। यथा—\textbf{एतद्‌\-द्वादशसाहस्रं देवानां युगमुच्यते} (म.स्मृ. १-७१)।\footnote{\ इस श्लोकपर अपने \textit{मनुभाष्य}में मेधातिथिने मनुष्योंके द्वादश सहस्र चतुर्युगोंको देवोंका एक युग कहा है, पर कुल्लूकभट्टने इस अर्थका सप्रमाण खण्डन करते हुए द्वादश सहस्र देववर्षों अर्थात् एक चतुर्युगको ही देवयुग सिद्ध किया है—संपादक।} तदनुसार यहाँ प्रयुक्त \textbf{\textit{जुग}} (संस्कृत: \textit{युग}) शब्द ‘देवयुग’ वैकल्पिक अर्थ ग्रहण करनेपर  १२,०००की संख्याका वाचक हुआ। \textbf{\textit{सहस्र}} शब्द १,०००का वाचक है ही और \textbf{\textit{जोजन}} (संस्कृत: \textit{योजन}) शब्द ८~मीलका परिमाण है।\footnote{{\englishfont{\relscale{0.75} \foreignlanguage{english}{Vaman Shivram Apte (1985) [1890]. \textit{The Practical Sanskrit-English Dictionary} (4th ed.). Delhi: Motilal Banarsidass,  p.~789: “A measure of distance equal to four \textit{Krosas} or eight or nine miles.”}}}—संपादक।} इस प्रकार \textbf{\textit{जुग सहस्र जोजन}}का एक और अर्थ हुआ १२,००० सहस्र योजन—अर्थात् १,२०,००,००० योजन अथवा ९,६०,००,००० मील। संभवतः यह ज्योतिषविद् गोस्वामी तुलसीदासजीके द्वारा दी हुई पृथ्वीसे सूर्यकी दूरीकी गणना है।\footnote{ सन् २०१२में पारित अन्ताराष्ट्रिय खगोलीय सङ्घके बी-२ प्रस्ताव के अनुसार सूर्यकी पृथ्वीसे औसत दूरी ९,२९,५५,८०७ मील है। इस आधुनिक वैज्ञानिक मानसे उपर्युक्त संभावित मान ३.३ प्रतिशत अधिक है—संपादक।} ऐसा भी प्रतीत होता है कि खगोलीय दूरियोंके मापनमें समयकी इकाई (\textit{युग})का सर्वप्रथम प्रयोग गोस्वामीजीने ही किया है।\footnote{\ प्रकाश-वर्ष ({\englishfont{\relscale{0.75}light-year}}) इकाईके सर्वप्रथम प्रयोगका श्रेय जर्मनीके वैज्ञानिक फ़्रीड्रिश बेसेल (१७८४–१८४६)को दिया जाता है—संपादक।}\end{sloppypar}
% \paraseplotus
% \vspace{\baselineskip}
\pagebreak


\fancyhead[LE,RO]{{\textmd{\large चौ. १९: प्रभु-मुद्रिका मेलि मुख माहीं}}}
\phantomsection
\addcontentsline{toc}{section}{चौपाई १९: प्रभु-मुद्रिका मेलि मुख माहीं}
\centering{॥ श्रीराम ॥}
\begin{sloppypar}\justifying\hyphenrules{nohyphenation}
मूल (चौपाई)—
\end{sloppypar}

{\bfseries\relscale{1.2}
\setlength{\mylenone}{0pt}
\settowidth{\mylentwo}{प्रभु-मुद्रिका मेलि मुख माहीं}
\setlength{\mylenone}{\maxof{\mylenone}{\mylentwo}}
\settowidth{\mylentwo}{जलधि लाँघि गये अचरज नाहीं}
\setlength{\mylenone}{\maxof{\mylenone}{\mylentwo}}
\setlength{\mylentwo}{\baselineskip}
\setlength{\mylenone}{\mylenone + 1pt}
\setlength{\mylen}{(\textwidth - \mylenone)*\real{0.5}}
\begin{longtable}[l]{@{\hspace*{\mylen}}>{\setlength\parfillskip{0pt}}p{\mylenone}@{}@{}l@{}}
 & \\[-\the\mylentwo]
प्रभु-मुद्रिका मेलि मुख माहीं & ।\\ \nopagebreak[1mm]
जलधि लाँघि गये अचरज नाहीं & ॥ १९ ॥
\end{longtable}
}

\parasepone
\index[ardha]{प्रभुमुद्रिकामेलिमुखमाहीं(चौ.१९,पूर्वार्ध)@प्रभु-मुद्रिका मेलि मुख माहीं (चौ. १९, पूर्वार्ध)}
\index[ardha]{जलधिलाँघिगयेअचरजनाहीं(चौ.१९,उत्तरार्ध)@जलधि लाँघि गये अचरज नाहीं (चौ. १९, उत्तरार्ध)}
\index[pada]{प्रभु@प्रभु}
\index[pada]{मुद्रिका@मुद्रिका}
\index[pada]{मेलि@मेलि}
\index[pada]{मुख@मुख}
\index[pada]{माहीं@माहीं}
\index[pada]{जलधि@जलधि}
\index[pada]{लाँघि@लाँघि}
\index[pada]{गये@गये}
\index[pada]{अचरज@अचरज}
\index[pada]{नाहीं@नाहीं}
\begin{sloppypar}\justifying\hyphenrules{nohyphenation}
\textbf{शब्दार्थ}—\textbf{\textit{मेलि}} {\unifont{\relscale{0.7}▶}} डालकर। \textbf{\textit{जलधि}} {\unifont{\relscale{0.7}▶}} समुद्र।
\end{sloppypar}
\begin{sloppypar}\justifying\hyphenrules{nohyphenation}
\textbf{अर्थ}—प्रभो! आप श्रीरामजीकी दी हुई रामनामाङ्कित मुद्रिकाको मुखमें लेकर शतयोजन-विस्तीर्ण समुद्रको लाँघ गए, इसमें कोई आश्चर्य नहीं है।
\end{sloppypar}
\parasepone
\begin{sloppypar}\justifying\hyphenrules{nohyphenation}
\textbf{व्याख्या}—दो शरणागतियोंमें प्रधान भूमिकाका निर्देश कर हनुमान्‌जीमें तारणत्व गुणका वर्णन किया। अब तरणत्वका वर्णन कर रहे हैं। अर्थात् हनुमान्‌जी सुग्रीव एवं विभीषणको सागरसे तारकर स्वयं भी तर जाते हैं। सुग्रीवके लिए नाम-सेतु तथा विभीषणके लिए कृपा-सेतुकी व्यवस्था करते हैं, एवं स्वयं प्रभु-मुद्रिकाको मुखमें लेकर राम-नामामृत चूसते हुए कौतुकमें समुद्रको पार करते हैं। यथा—
\end{sloppypar}
{\bfseries
\setlength{\mylenone}{0pt}
\settowidth{\mylentwo}{कौतुक सिंधु नाघि तव लंका}
\setlength{\mylenone}{\maxof{\mylenone}{\mylentwo}}
\settowidth{\mylentwo}{आयउ कपि-केहरी अशंका}
\setlength{\mylenone}{\maxof{\mylenone}{\mylentwo}}
\setlength{\mylentwo}{\baselineskip}
\setlength{\mylenone}{\mylenone + 1pt}
\setlength{\mylen}{(\textwidth - \mylenone)*\real{0.5}}
\begin{longtable}[l]{@{\hspace*{\mylen}}>{\setlength\parfillskip{0pt}}p{\mylenone}@{}@{}l@{}}
 & \\[-\the\mylentwo]
कौतुक सिंधु नाघि तव लंका & ।\\ \nopagebreak
आयउ कपि-केहरी अशंका & ॥\\ \nopagebreak
\caption*{—रा.च.मा. ६-३६-४}
\end{longtable}
}
\paraseplotus
\pagebreak


\fancyhead[LE,RO]{{\textmd{\large चौ. २०: दुर्गम काज जगत के जे ते}}}
\phantomsection
\addcontentsline{toc}{section}{चौपाई २०: दुर्गम काज जगत के जे ते}
\centering{॥ श्रीराम ॥}
\begin{sloppypar}\justifying\hyphenrules{nohyphenation}
मूल (चौपाई)—
\end{sloppypar}

{\bfseries\relscale{1.2}
\setlength{\mylenone}{0pt}
\settowidth{\mylentwo}{दुर्गम काज जगत के जे ते}
\setlength{\mylenone}{\maxof{\mylenone}{\mylentwo}}
\settowidth{\mylentwo}{सुगम अनुग्रह तुम्हरे ते ते}
\setlength{\mylenone}{\maxof{\mylenone}{\mylentwo}}
\setlength{\mylentwo}{\baselineskip}
\setlength{\mylenone}{\mylenone + 1pt}
\setlength{\mylen}{(\textwidth - \mylenone)*\real{0.5}}
\begin{longtable}[l]{@{\hspace*{\mylen}}>{\setlength\parfillskip{0pt}}p{\mylenone}@{}@{}l@{}}
 & \\[-\the\mylentwo]
दुर्गम काज जगत के जे ते & ।\\ \nopagebreak[1mm]
सुगम अनुग्रह तुम्हरे ते ते & ॥ २० ॥
\end{longtable}
}

\parasepone
\index[ardha]{दुर्गमकाजजगतकेजेते(चौ.२०,पूर्वार्ध)@दुर्गम काज जगत के जे ते (चौ. २०, पूर्वार्ध)}
\index[ardha]{सुगमअनुग्रहतुम्हरेतेते(चौ.२०,उत्तरार्ध)@सुगम अनुग्रह तुम्हरे ते ते (चौ. २०, उत्तरार्ध)}
\index[pada]{दुर्गम@दुर्गम}
\index[pada]{काज@काज}
\index[pada]{जगत@जगत}
\index[pada]{के@के}
\index[pada]{जे@जे}
\index[pada]{ते@ते}
\index[pada]{सुगम@सुगम}
\index[pada]{अनुग्रह@अनुग्रह}
\index[pada]{तुम्हरे@तुम्हरे}
\index[pada]{ते@ते}
\index[pada]{ते@ते}
\begin{sloppypar}\justifying\hyphenrules{nohyphenation}
\textbf{शब्दार्थ}—\textbf{\textit{दुर्गम}} {\unifont{\relscale{0.7}▶}} कठिन। \textbf{\textit{सुगम}} {\unifont{\relscale{0.7}▶}} सरल। \textbf{\textit{अनुग्रह}} {\unifont{\relscale{0.7}▶}} कृपा।
\end{sloppypar}
\begin{sloppypar}\justifying\hyphenrules{nohyphenation}
\textbf{अर्थ}—हे महावीरजी! संसारके जितने भी कठिन-से-भी-कठिन कार्य हैं, वे सब आपकी कृपासे सरल हो जाते हैं।
\end{sloppypar}
\parasepone
\begin{sloppypar}\justifying\hyphenrules{nohyphenation}
\textbf{व्याख्या}—क्योंकि आप सुग्रीव तथा विभीषणके लिए तारण एवं स्वयं तरण हैं तथा जटिल-से-जटिल कार्य आपने किया है। यथा—
\end{sloppypar}
{\bfseries
\setlength{\mylenone}{0pt}
\settowidth{\mylentwo}{मन को अगम तन सुगम किये कपीश}
\setlength{\mylenone}{\maxof{\mylenone}{\mylentwo}}
\settowidth{\mylentwo}{काज महाराज के समाज साज साजे हैं}
\setlength{\mylenone}{\maxof{\mylenone}{\mylentwo}}
\settowidth{\mylentwo}{देव बंदीछोर रनरोर केसरी-किसोर}
\setlength{\mylenone}{\maxof{\mylenone}{\mylentwo}}
\settowidth{\mylentwo}{जुग जुग जग तेरे बिरद बिराजे हैं}
\setlength{\mylenone}{\maxof{\mylenone}{\mylentwo}}
\settowidth{\mylentwo}{बीर बरजोर घटि जोर तुलसी की ओर}
\setlength{\mylenone}{\maxof{\mylenone}{\mylentwo}}
\settowidth{\mylentwo}{सुनि सकुचाने साधु खल-गन गाजे हैं}
\setlength{\mylenone}{\maxof{\mylenone}{\mylentwo}}
\settowidth{\mylentwo}{बिगरी सँवारि अँजनीकुमार कीजे मोहि}
\setlength{\mylenone}{\maxof{\mylenone}{\mylentwo}}
\settowidth{\mylentwo}{जैसे होत आये हनुमान के निवाजे हैं}
\setlength{\mylenone}{\maxof{\mylenone}{\mylentwo}}
\setlength{\mylentwo}{\baselineskip}
\setlength{\mylenone}{\mylenone + 1pt}
\setlength{\mylen}{(\textwidth - \mylenone)*\real{0.5}}
\begin{longtable}[l]{@{\hspace*{\mylen}}>{\setlength\parfillskip{0pt}}p{\mylenone}@{}@{}l@{}}
 & \\[-\the\mylentwo]
मन को अगम तन सुगम किये कपीश & \\ \nopagebreak
काज महाराज के समाज साज साजे हैं & ।\\
देव बंदीछोर रनरोर केसरी-किसोर & \\ \nopagebreak
जुग जुग जग तेरे बिरद बिराजे हैं & ।\\
बीर बरजोर घटि जोर तुलसी की ओर & \\ \nopagebreak
सुनि सकुचाने साधु खल-गन गाजे हैं & ।\\
बिगरी सँवारि अँजनीकुमार कीजे मोहि & \\ \nopagebreak
जैसे होत आये हनुमान के निवाजे हैं & ॥\\ \nopagebreak
\caption*{—ह.बा. १५}
\end{longtable}
}
\begin{sloppypar}\justifying\hyphenrules{nohyphenation}
यहाँ आञ्जनेयके चरित्र-वर्णनका उपसंहार करके अब कृपाकी आवश्यकताका उपपादन करते हैं।
\end{sloppypar}
\paraseplotus
\pagebreak


\fancyhead[LE,RO]{{\textmd{\large चौ. २१: राम-दुआरे तुम रखवारे}}}
\phantomsection
\addcontentsline{toc}{section}{चौपाई २१: राम-दुआरे तुम रखवारे}
\centering{॥ श्रीराम ॥}
\begin{sloppypar}\justifying\hyphenrules{nohyphenation}
मूल (चौपाई)—
\end{sloppypar}

{\bfseries\relscale{1.2}
\setlength{\mylenone}{0pt}
\settowidth{\mylentwo}{राम-दुआरे तुम रखवारे}
\setlength{\mylenone}{\maxof{\mylenone}{\mylentwo}}
\settowidth{\mylentwo}{होत न आज्ञा बिनु पैसारे}
\setlength{\mylenone}{\maxof{\mylenone}{\mylentwo}}
\setlength{\mylentwo}{\baselineskip}
\setlength{\mylenone}{\mylenone + 1pt}
\setlength{\mylen}{(\textwidth - \mylenone)*\real{0.5}}
\begin{longtable}[l]{@{\hspace*{\mylen}}>{\setlength\parfillskip{0pt}}p{\mylenone}@{}@{}l@{}}
 & \\[-\the\mylentwo]
राम-दुआरे तुम रखवारे & ।\\ \nopagebreak[1mm]
होत न आज्ञा बिनु पैसारे & ॥ २१ ॥
\end{longtable}
}

\parasepone
\index[ardha]{रामदुआरेतुमरखवारे(चौ.२१,पूर्वार्ध)@राम-दुआरे तुम रखवारे (चौ. २१, पूर्वार्ध)}
\index[ardha]{होतनआज्ञाबिनुपैसारे(चौ.२१,उत्तरार्ध)@होत न आज्ञा बिनु पैसारे (चौ. २१, उत्तरार्ध)}
\index[pada]{राम@राम}
\index[pada]{दुआरे@दुआरे}
\index[pada]{तुम@तुम}
\index[pada]{रखवारे@रखवारे}
\index[pada]{होत@होत}
\index[pada]{न@न}
\index[pada]{आज्ञा@आज्ञा}
\index[pada]{बिनु@बिनु}
\index[pada]{पैसारे@पैसारे}
\begin{sloppypar}\justifying\hyphenrules{nohyphenation}
\textbf{शब्दार्थ}—\textbf{\textit{पैसारे}} {\unifont{\relscale{0.7}▶}} प्रवेश।
\end{sloppypar}
\begin{sloppypar}\justifying\hyphenrules{nohyphenation}
\textbf{अर्थ}—हे अञ्जनीपुत्र! आप श्रीरामभद्रजूके राजद्वारके रक्षक, प्रतिहार, द्वारपाल हैं। आपकी आज्ञाके बिना किसीका भी श्रीरामजीके परमधाममें प्रवेश नहीं हो सकता।
\end{sloppypar}
\parasepone
\begin{sloppypar}\justifying\hyphenrules{nohyphenation}
\textbf{व्याख्या}—श्रीरामोपासनामें श्रीहनुमान्‌जीकी कृपा परम उपादेय है क्योंकि ये ही राजाधिराजके जागरूक द्वारपाल हैं। इनकी प्रतिकूलतामें जीवको श्रीराघवका आनुकूल्य नहीं प्राप्त हो सकता। अन्यत्र द्वारपाल स्वामीकी आज्ञासे आगन्तुकको भवनमें प्रविष्ट करता है, पर यहाँ तो स्वामी एवं सेवककी इतनी एकता है कि आञ्जनेयकी आज्ञा ही सर्वोपरि हो जाती है। सुग्रीव एवं विभीषणकी शरणागतिमें प्रभुसे बिना पूछे ही इन्होंने दोनोंको प्रवेशपत्र दे दिया; कारण कि वे अपने प्रभुसे इतने एकरूप हो चुके हैं कि इनके विरुद्ध कभी कोई चेष्टा करते ही नहीं और श्रीरामजी भी आञ्जनेयका अदब मानते हैं। यथा—\textbf{सेवक स्योकाई जानि जानकीश मानै कानि} (ह.बा. १२)। 
\end{sloppypar}
\begin{sloppypar}\justifying\hyphenrules{nohyphenation}
\textbf{\textit{पैसारे}} शब्द \textit{पदसार} शब्दका तद्भव है। \textit{पदेन पादेन सरणं सारः पदसारः प्रवेश इत्यर्थः}। यह शब्द प्रवेशके अर्थमें श्रीमानसजीमें भी प्रयुक्त हुआ है। यथा—\textbf{अति लघु रूप धरौं निशि नगर करौं पैसार} (रा.च.मा. ५-३)।
\end{sloppypar}
\begin{sloppypar}\justifying\hyphenrules{nohyphenation}
भगवान्‌के अन्य द्वारपाल श्रीआञ्जनेयके समान नहीं देखे जाते। कहीं-कहीं तो स्वामीको सूचित किए बिना ही प्रतिहार अपनी उच्छृङ्खलताके कारण आगन्तुकको बहुत क्षुब्ध किया करते हैं। इस विषयकी स्पष्टताके लिए \textit{भागवत}के तृतीय स्कन्धका जय-विजय-उपाख्यान द्रष्टव्य है। सनकादिक एक बार भगवान् मधुसूदनके दर्शनार्थ श्रीवैकुण्ठ धाम पधारे। छः द्वारोंको सहजतया लाँघकर वे सप्तम द्वारको भी लाँघनेकी चेष्टा कर रहे थे कि उनका यह स्वतन्त्रतापूर्वक व्यवहार भगवान्‌के प्रिय द्वारपाल जय-विजयको नहीं भाया। सन्तोंका व्यक्तित्व चमत्कार-शून्य तथा नमस्कार-प्रधान हुआ करता है। उनकी रहनीमें संसारका दिखावा तथा आडम्बर नहीं होता। सनक, सनन्दन, सनातन, सनत्कुमारको यह आशा भी न थी कि उन जैसे विधि-निषेधसे बहिर्भूत परम अन्तरङ्गतम भगवत्प्रेमी महात्माओंके साथ भी द्वारपालसे आदेश-रूप निरर्थक सांसारिक औपचारिकताकी अपेक्षा की जाएगी। जय-विजयने उन पञ्चवर्षीय दिगम्बर मुनिकुमारोंको बिना अनुमतिके ही भगवन्मन्दिरमें प्रवेश करते देख ईषत् क्रोधजडीभूत होकर परिहासपूर्वक अपने बेंतका प्रहार कर नीचे गिरा दिया। जय-विजयका यह स्वभाव भगवान् तथा भगवान्‌के भक्त दोनोंके लिए प्रतिकूल था। भागवतकारने \textbf{वेत्रेण चास्खलयताम्}का प्रयोग किया है। \textit{स्खलन}का अर्थ होता है गिरना। \textit{अस्खलयताम्} प्रेरणार्थक णिजन्त रूप है, जिसका अर्थ होता है ‘उन दोनोंने गिरा दिया’। यथा—
\end{sloppypar}
{\bfseries
\setlength{\mylenone}{0pt}
\settowidth{\mylentwo}{तान् वीक्ष्य वातरशनांश्चतुरः कुमारान्}
\setlength{\mylenone}{\maxof{\mylenone}{\mylentwo}}
\settowidth{\mylentwo}{वृद्धान्दशार्धवयसो विदितात्मतत्त्वान्}
\setlength{\mylenone}{\maxof{\mylenone}{\mylentwo}}
\settowidth{\mylentwo}{वेत्रेण चास्खलयतामतदर्हणांस्तौ}
\setlength{\mylenone}{\maxof{\mylenone}{\mylentwo}}
\settowidth{\mylentwo}{तेजो विहस्य भगवत्प्रतिकूलशीलौ}
\setlength{\mylenone}{\maxof{\mylenone}{\mylentwo}}
\setlength{\mylentwo}{\baselineskip}
\setlength{\mylenone}{\mylenone + 1pt}
\setlength{\mylen}{(\textwidth - \mylenone)*\real{0.5}}
\begin{longtable}[l]{@{\hspace*{\mylen}}>{\setlength\parfillskip{0pt}}p{\mylenone}@{}@{}l@{}}
 & \\[-\the\mylentwo]
तान् वीक्ष्य वातरशनांश्चतुरः कुमारान् & \\ \nopagebreak
वृद्धान्दशार्धवयसो विदितात्मतत्त्वान् & ।\\
वेत्रेण चास्खलयतामतदर्हणांस्तौ & \\ \nopagebreak
तेजो विहस्य भगवत्प्रतिकूलशीलौ & ॥\\ \nopagebreak
\caption*{—भा.पु. ३-१५-३०}
\end{longtable}
}
\begin{sloppypar}\justifying\hyphenrules{nohyphenation}
\noindent इस उद्दण्डताकी पराकाष्ठासे वीतराग महर्षियोंका भी हृदयसागर क्रोधकी लहरसे कुछ क्षुब्ध-सा हो गया। तथा वे बोल पड़े, “तुम सर्वान्तर्यामी परम कृपालु प्रभु भगवान् विष्णुके पार्षद होनेके योग्य नहीं हो। आज भी तुम्हारा हृदय महत्त्वाकाङ्क्षाकी आगसे जल रहा है। अतः इस अपराधका उचित दण्ड ही तुम्हारे लिए उपयुक्त है। तुम काम, क्रोध, लोभ इन तीनोंसे पीड़ित हो। इसलिए तीन पापिष्ठ लोकोंमें जाओ।” अर्थात् प्रथम जन्ममें क्रोध-प्रधान दैत्य बनो (हिरण्यकशिपु और हिरण्याक्ष), द्वितीय जन्ममें काम-प्रधान राक्षस बनो (रावण और कुम्भकर्ण), एवं तृतीय जन्ममें लोभ-प्रधान दानव-रूप मानवताहीन मानव (शिशुपाल और दन्तवक्र) बनो। यहाँ तीन जन्म पर्यन्त पापिष्ठ लोकोंमें जानेका शाप भी साभिप्राय था। सनकादि महर्षियोंने जय-विजयको तीन जन्मके लिए इस कारण शाप दिया कि जय-विजयने उन्हें तीन-तीन बेंत लगाये थे। यही \textbf{त्रय इमे} शब्दके प्रयोगका कारण प्रतीत होता है, यथा—
\end{sloppypar}
{\bfseries
\setlength{\mylenone}{0pt}
\settowidth{\mylentwo}{तद्वाममुष्य परमस्य विकुण्ठभर्तुः}
\setlength{\mylenone}{\maxof{\mylenone}{\mylentwo}}
\settowidth{\mylentwo}{कर्तुं प्रकृष्टमिह धीमहि मन्दधीभ्याम्}
\setlength{\mylenone}{\maxof{\mylenone}{\mylentwo}}
\settowidth{\mylentwo}{लोकानितो व्रजतमन्तरभावदृष्ट्या}
\setlength{\mylenone}{\maxof{\mylenone}{\mylentwo}}
\settowidth{\mylentwo}{पापीयसस्त्रय इमे रिपवोऽस्य यत्र}
\setlength{\mylenone}{\maxof{\mylenone}{\mylentwo}}
\setlength{\mylentwo}{\baselineskip}
\setlength{\mylenone}{\mylenone + 1pt}
\setlength{\mylen}{(\textwidth - \mylenone)*\real{0.5}}
\begin{longtable}[l]{@{\hspace*{\mylen}}>{\setlength\parfillskip{0pt}}p{\mylenone}@{}@{}l@{}}
 & \\[-\the\mylentwo]
तद्वाममुष्य परमस्य विकुण्ठभर्तुः & \\ \nopagebreak
कर्तुं प्रकृष्टमिह धीमहि मन्दधीभ्याम् & ।\\
लोकानितो व्रजतमन्तरभावदृष्ट्या & \\ \nopagebreak
पापीयसस्त्रय इमे रिपवोऽस्य यत्र & ॥\\ \nopagebreak
\caption*{—भा.पु. ३-१५-३४}
\end{longtable}
}
\begin{sloppypar}\justifying\hyphenrules{nohyphenation}
\noindent गोस्वामीजी भी इस प्रसंगकी चर्चा मानसजीमें बड़े ही रोचक ढंगसे करते हैं—
\end{sloppypar}
{\bfseries
\setlength{\mylenone}{0pt}
\settowidth{\mylentwo}{द्वारपाल हरि के प्रिय दोऊ}
\setlength{\mylenone}{\maxof{\mylenone}{\mylentwo}}
\settowidth{\mylentwo}{जय अरु विजय जान सब कोऊ}
\setlength{\mylenone}{\maxof{\mylenone}{\mylentwo}}
\settowidth{\mylentwo}{बिप्र-शाप तें दूनउ भाई}
\setlength{\mylenone}{\maxof{\mylenone}{\mylentwo}}
\settowidth{\mylentwo}{तामस असुर-देह तिन पाई}
\setlength{\mylenone}{\maxof{\mylenone}{\mylentwo}}
\setlength{\mylentwo}{\baselineskip}
\setlength{\mylenone}{\mylenone + 1pt}
\setlength{\mylen}{(\textwidth - \mylenone)*\real{0.5}}
\begin{longtable}[l]{@{\hspace*{\mylen}}>{\setlength\parfillskip{0pt}}p{\mylenone}@{}@{}l@{}}
 & \\[-\the\mylentwo]
द्वारपाल हरि के प्रिय दोऊ & ।\\ \nopagebreak
जय अरु विजय जान सब कोऊ & ॥\\
बिप्र-शाप तें दूनउ भाई & ।\\ \nopagebreak
तामस असुर-देह तिन पाई & ॥\\ \nopagebreak
\caption*{—रा.च.मा. १-१२२-४,५}
\end{longtable}
}
{\bfseries
\setlength{\mylenone}{0pt}
\settowidth{\mylentwo}{मुक्त न भये हते भगवाना}
\setlength{\mylenone}{\maxof{\mylenone}{\mylentwo}}
\settowidth{\mylentwo}{तीनि जनम द्विज-बचन प्रमाना}
\setlength{\mylenone}{\maxof{\mylenone}{\mylentwo}}
\setlength{\mylentwo}{\baselineskip}
\setlength{\mylenone}{\mylenone + 1pt}
\setlength{\mylen}{(\textwidth - \mylenone)*\real{0.5}}
\begin{longtable}[l]{@{\hspace*{\mylen}}>{\setlength\parfillskip{0pt}}p{\mylenone}@{}@{}l@{}}
 & \\[-\the\mylentwo]
मुक्त न भये हते भगवाना & ।\\ \nopagebreak
तीनि जनम द्विज-बचन प्रमाना & ॥\\ \nopagebreak
\caption*{— रा.च.मा. १-१२३-१}
\end{longtable}
}
\begin{sloppypar}\justifying\hyphenrules{nohyphenation}
\noindent अर्थात् अन्य द्वारपाल अप्रत्याशित रूपमें स्वामीके यहाँ जानेवालोंको बेंतके प्रहारसे निरस्त करते हैं, किन्तु श्रीहनुमान्‌जी महाराज रावणके द्वारा कृत चरण-प्रहारसे पीड़ित विभीषणको भी भगवान्‌के श्रीचरणारविन्दका शरणागत बना देते हैं। श्रीराघवकी शरणमें समागत विभीषणके प्रति जब सुग्रीव नाना प्रकारके आक्षेप-प्रत्याक्षेप करने लगे, तब आञ्जनेयको बहुत दुःख हुआ, पर श्रीराघवने \textbf{मम पन शरनागत-भय\-हारी} (रा.च.मा. ५-४३-८) कहकर विभीषणको स्वीकारनेका निश्चय किया। तब हनुमान्‌जी अत्यन्त प्रसन्न हुए, यथा—
\end{sloppypar}
{\bfseries
\setlength{\mylenone}{0pt}
\settowidth{\mylentwo}{सुनि प्रभु बचन हरष हनुमाना}
\setlength{\mylenone}{\maxof{\mylenone}{\mylentwo}}
\settowidth{\mylentwo}{शरनागत-वत्सल भगवाना}
\setlength{\mylenone}{\maxof{\mylenone}{\mylentwo}}
\setlength{\mylentwo}{\baselineskip}
\setlength{\mylenone}{\mylenone + 1pt}
\setlength{\mylen}{(\textwidth - \mylenone)*\real{0.5}}
\begin{longtable}[l]{@{\hspace*{\mylen}}>{\setlength\parfillskip{0pt}}p{\mylenone}@{}@{}l@{}}
 & \\[-\the\mylentwo]
सुनि प्रभु बचन हरष हनुमाना & ।\\ \nopagebreak
शरनागत-वत्सल भगवाना & ॥\\ \nopagebreak
\caption*{—रा.च.मा. ५-४३-९}
\end{longtable}
}
\begin{sloppypar}\justifying\hyphenrules{nohyphenation}
\noindent इस प्रकार अन्य द्वारपालकी अपेक्षा आगन्तुकको प्रभुके चरणोंमें जोड़नेकी श्रीहनुमान्‌जीमें विलक्षण क्षमता है।
\end{sloppypar}
\paraseplotus
\pagebreak


\fancyhead[LE,RO]{{\textmd{\large चौ. २२: सब सुख लहहिं तुम्हारी शरना}}}
\phantomsection
\addcontentsline{toc}{section}{चौपाई २२: सब सुख लहहिं तुम्हारी शरना}
\centering{॥ श्रीराम ॥}
\begin{sloppypar}\justifying\hyphenrules{nohyphenation}
मूल (चौपाई)—
\end{sloppypar}

{\bfseries\relscale{1.2}
\setlength{\mylenone}{0pt}
\settowidth{\mylentwo}{सब सुख लहहिं तुम्हारी शरना}
\setlength{\mylenone}{\maxof{\mylenone}{\mylentwo}}
\settowidth{\mylentwo}{तुम रक्षक काहू को डर ना}
\setlength{\mylenone}{\maxof{\mylenone}{\mylentwo}}
\setlength{\mylentwo}{\baselineskip}
\setlength{\mylenone}{\mylenone + 1pt}
\setlength{\mylen}{(\textwidth - \mylenone)*\real{0.5}}
\begin{longtable}[l]{@{\hspace*{\mylen}}>{\setlength\parfillskip{0pt}}p{\mylenone}@{}@{}l@{}}
 & \\[-\the\mylentwo]
सब सुख लहहिं तुम्हारी शरना & ।\\ \nopagebreak[1mm]
तुम रक्षक काहू को डर ना & ॥ २२ ॥
\end{longtable}
}

\parasepone
\index[ardha]{सबसुखलहहिंतुम्हारीशरना(चौ.२२,पूर्वार्ध)@सब सुख लहहिं तुम्हारी शरना (चौ. २२, पूर्वार्ध)}
\index[ardha]{तुमरक्षककाहूकोडरना(चौ.२२,उत्तरार्ध)@तुम रक्षक काहू को डर ना (चौ. २२, उत्तरार्ध)}
\index[pada]{सब@सब}
\index[pada]{सुख@सुख}
\index[pada]{लहहिं@लहहिं}
\index[pada]{तुम्हारी@तुम्हारी}
\index[pada]{शरना@शरना}
\index[pada]{तुम@तुम}
\index[pada]{रक्षक@रक्षक}
\index[pada]{काहू@काहू}
\index[pada]{को@को}
\index[pada]{डर@डर}
\index[pada]{ना@ना}
\begin{sloppypar}\justifying\hyphenrules{nohyphenation}
\textbf{शब्दार्थ}—\textbf{\textit{शरना}} {\unifont{\relscale{0.7}▶}} शरण में। \textbf{\textit{लहहिं}} {\unifont{\relscale{0.7}▶}} प्राप्त करते हैं।
\end{sloppypar}
\begin{sloppypar}\justifying\hyphenrules{nohyphenation}
\textbf{अर्थ}—हे हनुमान्‌जी महाराज! आपकी शरणमें आकर साधक जन समस्त सुख प्राप्त कर लेते हैं। आप रक्षक हैं, अतः अब किसीका डर नहीं है।
\end{sloppypar}
\parasepone
\begin{sloppypar}\justifying\hyphenrules{nohyphenation}
\textbf{व्याख्या}—जय-विजयकी भाँति आप किसी आगन्तुकको भगवान्‌के दर्शनसे वञ्चित नहीं करते, अपितु उनसे पूछे बिना भी अपनी कृपालुतासे आप श्रीराघवका दर्शन करा देते हैं। अब हम निर्भीक हो गए हैं। यथा—
\end{sloppypar}
{\bfseries
\setlength{\mylenone}{0pt}
\settowidth{\mylentwo}{साहसी समत्थ तुलसी को नाह जाकी बाँह}
\setlength{\mylenone}{\maxof{\mylenone}{\mylentwo}}
\settowidth{\mylentwo}{लोकपाल-पालन को फिर थिर थल भो}
\setlength{\mylenone}{\maxof{\mylenone}{\mylentwo}}
\setlength{\mylentwo}{\baselineskip}
\setlength{\mylenone}{\mylenone + 1pt}
\setlength{\mylen}{(\textwidth - \mylenone)*\real{0.5}}
\begin{longtable}[l]{@{\hspace*{\mylen}}>{\setlength\parfillskip{0pt}}p{\mylenone}@{}@{}l@{}}
 & \\[-\the\mylentwo]
साहसी समत्थ तुलसी को नाह जाकी बाँह & \\ \nopagebreak
लोकपाल-पालन को फिर थिर थल भो & ।\\ \nopagebreak
\caption*{—ह.बा. ६}
\end{longtable}
}
\paraseplotus
\pagebreak


\fancyhead[LE,RO]{{\textmd{\large चौ. २३: आपन तेज सम्हारो आपे}}}
\phantomsection
\addcontentsline{toc}{section}{चौपाई २३: आपन तेज सम्हारो आपे}
\centering{॥ श्रीराम ॥}
\begin{sloppypar}\justifying\hyphenrules{nohyphenation}
मूल (चौपाई)—
\end{sloppypar}

{\bfseries\relscale{1.2}
\setlength{\mylenone}{0pt}
\settowidth{\mylentwo}{आपन तेज सम्हारो आपे}
\setlength{\mylenone}{\maxof{\mylenone}{\mylentwo}}
\settowidth{\mylentwo}{तीनौं लोक हाँक ते काँपे}
\setlength{\mylenone}{\maxof{\mylenone}{\mylentwo}}
\setlength{\mylentwo}{\baselineskip}
\setlength{\mylenone}{\mylenone + 1pt}
\setlength{\mylen}{(\textwidth - \mylenone)*\real{0.5}}
\begin{longtable}[l]{@{\hspace*{\mylen}}>{\setlength\parfillskip{0pt}}p{\mylenone}@{}@{}l@{}}
 & \\[-\the\mylentwo]
आपन तेज सम्हारो आपे & ।\\ \nopagebreak[1mm]
तीनौं लोक हाँक ते काँपे & ॥ २३ ॥
\end{longtable}
}

\parasepone
\index[ardha]{आपनतेजसम्हारोआपे(चौ.२३,पूर्वार्ध)@आपन तेज सम्हारो आपे (चौ. २३, पूर्वार्ध)}
\index[ardha]{तीनौंलोकहाँकतेकाँपे(चौ.२३,उत्तरार्ध)@तीनौं लोक हाँक ते काँपे (चौ. २३, उत्तरार्ध)}
\index[pada]{आपन@आपन}
\index[pada]{तेज@तेज}
\index[pada]{सम्हारो@सम्हारो}
\index[pada]{आपे@आपे}
\index[pada]{तीनौं@तीनौं}
\index[pada]{लोक@लोक}
\index[pada]{हाँक@हाँक}
\index[pada]{ते@ते}
\index[pada]{काँपे@काँपे}
\begin{sloppypar}\justifying\hyphenrules{nohyphenation}
\textbf{शब्दार्थ}—\textbf{\textit{सम्हारो}} {\unifont{\relscale{0.7}▶}} स्मरण करें।
\end{sloppypar}
\begin{sloppypar}\justifying\hyphenrules{nohyphenation}
\textbf{अर्थ}—हे प्रभो! जब आप अपने तेजको स्मरण कर लेते हैं, तब आपकी हाँकसे ही त्रैलोक्य कम्पित हो उठता है।
\end{sloppypar}
\parasepone
\begin{sloppypar}\justifying\hyphenrules{nohyphenation}
\textbf{व्याख्या}—\textbf{\textit{सम्हारो}} शब्दका अर्थ है स्मरण। यथा—\textbf{दीन दयाल बिरद संभारी} (रा.च.मा. ५-२७-४)। क्योंकि ऋषियोंके शापसे इन्हें अपना तेज विस्मृत रहता है, इसलिए जाम्बवान्‌को स्मरण दिलाना पड़ा। यथा—
\end{sloppypar}
{\bfseries
\setlength{\mylenone}{0pt}
\settowidth{\mylentwo}{कवन सो काज कठिन जग माहीं}
\setlength{\mylenone}{\maxof{\mylenone}{\mylentwo}}
\settowidth{\mylentwo}{जो नहिं होइ तात तुम पाहीं}
\setlength{\mylenone}{\maxof{\mylenone}{\mylentwo}}
\setlength{\mylentwo}{\baselineskip}
\setlength{\mylenone}{\mylenone + 1pt}
\setlength{\mylen}{(\textwidth - \mylenone)*\real{0.5}}
\begin{longtable}[l]{@{\hspace*{\mylen}}>{\setlength\parfillskip{0pt}}p{\mylenone}@{}@{}l@{}}
 & \\[-\the\mylentwo]
कवन सो काज कठिन जग माहीं & ।\\ \nopagebreak
जो नहिं होइ तात तुम पाहीं & ॥\\ \nopagebreak
\caption*{—रा.च.मा. ४-३०-५}
\end{longtable}
}
\begin{sloppypar}\justifying\hyphenrules{nohyphenation}
\noindent अतः आज भी भक्त लोग विरुदावली गाकर हनुमान्‌जीके तेजका उद्बोधन करते हैं।
\end{sloppypar}
\begin{sloppypar}\justifying\hyphenrules{nohyphenation}
\textbf{\textit{तीनौं लोक हाँक ते काँपे}}—इनकी हाँकका वर्णन \textit{कवितावली}के युद्धकाण्डमें इस प्रकार है—
\end{sloppypar}
{\bfseries
\setlength{\mylenone}{0pt}
\settowidth{\mylentwo}{मत्तभट-मुकुट-दशकंध-साहस-सइल-}
\setlength{\mylenone}{\maxof{\mylenone}{\mylentwo}}
\settowidth{\mylentwo}{सृंग-बिद्दरनि जनु बज्र टाँकी}
\setlength{\mylenone}{\maxof{\mylenone}{\mylentwo}}
\settowidth{\mylentwo}{दसन धरि धरनि चिक्करत दिग्गज कमठ}
\setlength{\mylenone}{\maxof{\mylenone}{\mylentwo}}
\settowidth{\mylentwo}{शेष संकुचित संकित पिनाकी}
\setlength{\mylenone}{\maxof{\mylenone}{\mylentwo}}
\settowidth{\mylentwo}{चलित महि मेरु उच्छलित सागर सकल}
\setlength{\mylenone}{\maxof{\mylenone}{\mylentwo}}
\settowidth{\mylentwo}{बिकल बिधि बधिर दिसि बिदिसि झाँकी}
\setlength{\mylenone}{\maxof{\mylenone}{\mylentwo}}
\settowidth{\mylentwo}{रजनिचर-घरनि घर गर्भ-अर्भक स्रवत}
\setlength{\mylenone}{\maxof{\mylenone}{\mylentwo}}
\settowidth{\mylentwo}{सुनत हनुमान की हाँक बाँकी}
\setlength{\mylenone}{\maxof{\mylenone}{\mylentwo}}
\setlength{\mylentwo}{\baselineskip}
\setlength{\mylenone}{\mylenone + 1pt}
\setlength{\mylen}{(\textwidth - \mylenone)*\real{0.5}}
\begin{longtable}[l]{@{\hspace*{\mylen}}>{\setlength\parfillskip{0pt}}p{\mylenone}@{}@{}l@{}}
 & \\[-\the\mylentwo]
मत्तभट-मुकुट-दशकंध-साहस-सइल- & \\ \nopagebreak
सृंग-बिद्दरनि जनु बज्र टाँकी & ।\\
दसन धरि धरनि चिक्करत दिग्गज कमठ & \\ \nopagebreak
शेष संकुचित संकित पिनाकी & ।\\
चलित महि मेरु उच्छलित सागर सकल & \\ \nopagebreak
बिकल बिधि बधिर दिसि बिदिसि झाँकी & ।\\
रजनिचर-घरनि घर गर्भ-अर्भक स्रवत & \\ \nopagebreak
सुनत हनुमान की हाँक बाँकी & ॥\\ \nopagebreak
\caption*{—क. ६-४४}
\end{longtable}
}
\paraseplotus
\pagebreak


\fancyhead[LE,RO]{{\textmd{\large चौ. २४: भूत पिशाच निकट नहिं आवै}}}
\phantomsection
\addcontentsline{toc}{section}{चौपाई २४: भूत पिशाच निकट नहिं आवै}
\centering{॥ श्रीराम ॥}
\begin{sloppypar}\justifying\hyphenrules{nohyphenation}
मूल (चौपाई)—
\end{sloppypar}

{\bfseries\relscale{1.2}
\setlength{\mylenone}{0pt}
\settowidth{\mylentwo}{भूत पिशाच निकट नहिं आवै}
\setlength{\mylenone}{\maxof{\mylenone}{\mylentwo}}
\settowidth{\mylentwo}{महाबीर जब नाम सुनावै}
\setlength{\mylenone}{\maxof{\mylenone}{\mylentwo}}
\setlength{\mylentwo}{\baselineskip}
\setlength{\mylenone}{\mylenone + 1pt}
\setlength{\mylen}{(\textwidth - \mylenone)*\real{0.5}}
\begin{longtable}[l]{@{\hspace*{\mylen}}>{\setlength\parfillskip{0pt}}p{\mylenone}@{}@{}l@{}}
 & \\[-\the\mylentwo]
भूत पिशाच निकट नहिं आवै & ।\\ \nopagebreak[1mm]
महाबीर जब नाम सुनावै & ॥ २४ ॥
\end{longtable}
}

\parasepone
\index[ardha]{भूतपिशाचनिकटनहिंआवै(चौ.२४,पूर्वार्ध)@भूत पिशाच निकट नहिं आवै (चौ. २४, पूर्वार्ध)}
\index[ardha]{महाबीरजबनामसुनावै(चौ.२४,उत्तरार्ध)@महाबीर जब नाम सुनावै (चौ. २४, उत्तरार्ध)}
\index[pada]{भूत@भूत}
\index[pada]{पिशाच@पिशाच}
\index[pada]{निकट@निकट}
\index[pada]{नहिं@नहिं}
\index[pada]{आवै@आवै}
\index[pada]{महाबीर@महाबीर}
\index[pada]{जब@जब}
\index[pada]{नाम@नाम}
\index[pada]{सुनावै@सुनावै}
\begin{sloppypar}\justifying\hyphenrules{nohyphenation}
\textbf{शब्दार्थ}—\textbf{\textit{भूत पिशाच}} {\unifont{\relscale{0.7}▶}} अकाल मृत्युको प्राप्त उग्र आत्मा अथवा निम्न देवयोनि।
\end{sloppypar}
\begin{sloppypar}\justifying\hyphenrules{nohyphenation}
\textbf{अर्थ}—जब भावुक भक्त \textit{महावीर} नाम सुना-सुना कर कीर्तन करते हैं, उस समय भूत पिशाच उनके निकट नहीं आते।
\end{sloppypar}
\parasepone
\begin{sloppypar}\justifying\hyphenrules{nohyphenation}
\textbf{व्याख्या}—अब \textit{महावीर} नाम विशेषकी महत्ता कहते हैं। यह भूत-पिशाचोंका त्रासक है। यथा—\textbf{पूतना पिशाची जातुधानी जातुधान वाम रामदूत~की रजाइ माथे मान लेत हैं} (ह.बा. ३२)।
\end{sloppypar}
\paraseplotus
\pagebreak


\fancyhead[LE,RO]{{\textmd{\large चौ. २५: नासै रोग हरै सब पीरा}}}
\phantomsection
\addcontentsline{toc}{section}{चौपाई २५: नासै रोग हरै सब पीरा}
\centering{॥ श्रीराम ॥}
\begin{sloppypar}\justifying\hyphenrules{nohyphenation}
मूल (चौपाई)—
\end{sloppypar}

{\bfseries\relscale{1.2}
\setlength{\mylenone}{0pt}
\settowidth{\mylentwo}{नासै रोग हरै सब पीरा}
\setlength{\mylenone}{\maxof{\mylenone}{\mylentwo}}
\settowidth{\mylentwo}{जपत निरंतर हनुमत बीरा}
\setlength{\mylenone}{\maxof{\mylenone}{\mylentwo}}
\setlength{\mylentwo}{\baselineskip}
\setlength{\mylenone}{\mylenone + 1pt}
\setlength{\mylen}{(\textwidth - \mylenone)*\real{0.5}}
\begin{longtable}[l]{@{\hspace*{\mylen}}>{\setlength\parfillskip{0pt}}p{\mylenone}@{}@{}l@{}}
 & \\[-\the\mylentwo]
नासै रोग हरै सब पीरा & ।\\ \nopagebreak[1mm]
जपत निरंतर हनुमत बीरा & ॥ २५ ॥
\end{longtable}
}

\parasepone
\index[ardha]{नासैरोगहरैसबपीरा(चौ.२५,पूर्वार्ध)@नासै रोग हरै सब पीरा (चौ. २५, पूर्वार्ध)}
\index[ardha]{जपतनिरंतरहनुमतबीरा(चौ.२५,उत्तरार्ध)@जपत निरंतर हनुमत बीरा (चौ. २५, उत्तरार्ध)}
\index[pada]{नासै@नासै}
\index[pada]{रोग@रोग}
\index[pada]{हरै@हरै}
\index[pada]{सब@सब}
\index[pada]{पीरा@पीरा}
\index[pada]{जपत@जपत}
\index[pada]{निरंतर@निरंतर}
\index[pada]{हनुमत@हनुमत}
\index[pada]{बीरा@बीरा}
\begin{sloppypar}\justifying\hyphenrules{nohyphenation}
\textbf{शब्दार्थ}—\textbf{\textit{निरंतर}} {\unifont{\relscale{0.7}▶}} नित्य, सदा।
\end{sloppypar}
\begin{sloppypar}\justifying\hyphenrules{nohyphenation}
\textbf{अर्थ}—सदैव भक्तोंके द्वारा जपके विषय-भूत होनेपर वीर हनुमान्‌जी रोगोंको नष्ट कर देते हैं एवं समस्त पीड़ाओंको हर लेते हैं।
\end{sloppypar}
\parasepone
\begin{sloppypar}\justifying\hyphenrules{nohyphenation}
\textbf{व्याख्या}—\textbf{\textit{रोग}} यहाँ शारीरिक रोगोंका वाचक है। \textbf{\textit{पीरा}} शब्दका कामादि आध्यात्मिक पीड़ाओंसे तात्पर्य है। भाव यह है कि गुरुदीक्षालब्ध श्रीहनुमन्मन्त्रका जप करनेसे साधकके शरीरके रोग तथा आध्यात्मिक ताप नष्ट हो जाते हैं।
\end{sloppypar}
\paraseplotus
\pagebreak


\fancyhead[LE,RO]{{\textmd{\large चौ. २६: संकट तें हनुमान छुड़ावै}}}
\phantomsection
\addcontentsline{toc}{section}{चौपाई २६: संकट तें हनुमान छुड़ावै}
\centering{॥ श्रीराम ॥}
\begin{sloppypar}\justifying\hyphenrules{nohyphenation}
मूल (चौपाई)—
\end{sloppypar}

{\bfseries\relscale{1.2}
\setlength{\mylenone}{0pt}
\settowidth{\mylentwo}{संकट तें हनुमान छुड़ावै}
\setlength{\mylenone}{\maxof{\mylenone}{\mylentwo}}
\settowidth{\mylentwo}{मन क्रम बचन ध्यान जो लावै}
\setlength{\mylenone}{\maxof{\mylenone}{\mylentwo}}
\setlength{\mylentwo}{\baselineskip}
\setlength{\mylenone}{\mylenone + 1pt}
\setlength{\mylen}{(\textwidth - \mylenone)*\real{0.5}}
\begin{longtable}[l]{@{\hspace*{\mylen}}>{\setlength\parfillskip{0pt}}p{\mylenone}@{}@{}l@{}}
 & \\[-\the\mylentwo]
संकट तें हनुमान छुड़ावै & ।\\ \nopagebreak[1mm]
मन क्रम बचन ध्यान जो लावै & ॥ २६ ॥
\end{longtable}
}

\parasepone
\index[ardha]{संकटतेंहनुमानछुड़ावै(चौ.२६,पूर्वार्ध)@संकट तें हनुमान छुड़ावै (चौ. २६, पूर्वार्ध)}
\index[ardha]{मनक्रमबचनध्यानजोलावै(चौ.२६,उत्तरार्ध)@मन क्रम बचन ध्यान जो लावै (चौ. २६, उत्तरार्ध)}
\index[pada]{संकट@संकट}
\index[pada]{तें@तें}
\index[pada]{हनुमान@हनुमान}
\index[pada]{छुडावै@छुड़ावै}
\index[pada]{मन@मन}
\index[pada]{क्रम@क्रम}
\index[pada]{बचन@बचन}
\index[pada]{ध्यान@ध्यान}
\index[pada]{जो@जो}
\index[pada]{लावै@लावै}
\begin{sloppypar}\justifying\hyphenrules{nohyphenation}
\textbf{शब्दार्थ}—\textbf{\textit{संकट}} {\unifont{\relscale{0.7}▶}} विपत्ति।
\end{sloppypar}
\begin{sloppypar}\justifying\hyphenrules{nohyphenation}
\textbf{अर्थ}—जो मन, कर्म, और वचनसे एकाग्र होकर हनुमान्‌जीको ध्यानमें ले आते हैं, उन्हें श्रीहनुमान्‌जी सभी संकटोंसे मुक्त कर देते हैं।
\end{sloppypar}
\parasepone
\begin{sloppypar}\justifying\hyphenrules{nohyphenation}
\textbf{व्याख्या}—इनका नाम ही \textit{संकटमोचन} है। यथा—\textbf{को नहिं जानत है जग में कपि संकटमोचन नाम तिहारो} (सं.ह.अ. १,२,३,४,५,६,७,८) और \textbf{गुण गनत नमत सुमिरत जपत समन सकल संकट बिकट} (ह.बा. १)।
\end{sloppypar}
\paraseplotus
\pagebreak


\fancyhead[LE,RO]{{\textmd{\large चौ. २७: सब-पर राम राय-सिरताजा}}}
\phantomsection
\addcontentsline{toc}{section}{चौपाई २७: सब-पर राम राय-सिरताजा}
\centering{॥ श्रीराम ॥}
\begin{sloppypar}\justifying\hyphenrules{nohyphenation}
मूल (चौपाई)—
\end{sloppypar}

{\bfseries\relscale{1.2}
\setlength{\mylenone}{0pt}
\settowidth{\mylentwo}{सब-पर राम राय-सिरताजा}
\setlength{\mylenone}{\maxof{\mylenone}{\mylentwo}}
\settowidth{\mylentwo}{तिन के काज सकल तुम साजा}
\setlength{\mylenone}{\maxof{\mylenone}{\mylentwo}}
\setlength{\mylentwo}{\baselineskip}
\setlength{\mylenone}{\mylenone + 1pt}
\setlength{\mylen}{(\textwidth - \mylenone)*\real{0.5}}
\begin{longtable}[l]{@{\hspace*{\mylen}}>{\setlength\parfillskip{0pt}}p{\mylenone}@{}@{}l@{}}
 & \\[-\the\mylentwo]
सब-पर राम राय-सिरताजा & ।\\ \nopagebreak[1mm]
तिन के काज सकल तुम साजा & ॥ २७ ॥
\end{longtable}
}

\parasepone
\index[ardha]{सबपररामरायसिरताजा(चौ.२७,पूर्वार्ध)@सब-पर राम राय-सिरताजा (चौ. २७, पूर्वार्ध)}
\index[ardha]{तिनकेकाजसकलतुमसाजा(चौ.२७,उत्तरार्ध)@तिन के काज सकल तुम साजा (चौ. २७, उत्तरार्ध)}
\index[pada]{सब@सब}
\index[pada]{पर@पर}
\index[pada]{राम@राम}
\index[pada]{राय@राय}
\index[pada]{सिरताजा@सिरताजा}
\index[pada]{तिन@तिन}
\index[pada]{के@के}
\index[pada]{काज@काज}
\index[pada]{सकल@सकल}
\index[pada]{तुम@तुम}
\index[pada]{साजा@साजा}
\begin{sloppypar}\justifying\hyphenrules{nohyphenation}
\textbf{शब्दार्थ}—\textbf{\textit{सब-पर}} {\unifont{\relscale{0.7}▶}} सर्वोपरि। \textit{सब-पर} शब्द \textit{सर्वपर}का तद्भव है। यहाँ \textbf{सर्वत्र लवराम्} (प्रा.प्र. ३-३) इस प्राकृत व्याकरण के सूत्रसे \textit{र}का लोप हुआ।
\end{sloppypar}
\begin{sloppypar}\justifying\hyphenrules{nohyphenation}
\textbf{अर्थ}—श्रीराम परब्रह्म और राजाओंके मुकुटमणि हैं। उनके भी संपूर्ण कार्योंको आपने ही संपन्न किया।
\end{sloppypar}
\parasepone
\begin{sloppypar}\justifying\hyphenrules{nohyphenation}
\textbf{व्याख्या}—श्रीरामजी सर्वोपरि हैं। यथा—
\end{sloppypar}
{\bfseries
\setlength{\mylenone}{0pt}
\settowidth{\mylentwo}{शंभु बिरंचि बिष्णु भगवाना}
\setlength{\mylenone}{\maxof{\mylenone}{\mylentwo}}
\settowidth{\mylentwo}{उपजहिं जासु अंश ते नाना}
\setlength{\mylenone}{\maxof{\mylenone}{\mylentwo}}
\setlength{\mylentwo}{\baselineskip}
\setlength{\mylenone}{\mylenone + 1pt}
\setlength{\mylen}{(\textwidth - \mylenone)*\real{0.5}}
\begin{longtable}[l]{@{\hspace*{\mylen}}>{\setlength\parfillskip{0pt}}p{\mylenone}@{}@{}l@{}}
 & \\[-\the\mylentwo]
शंभु बिरंचि बिष्णु भगवाना & ।\\ \nopagebreak
उपजहिं जासु अंश ते नाना & ॥\\ \nopagebreak
\caption*{—रा.च.मा. १-१४४-६}
\end{longtable}
}
\begin{sloppypar}\justifying\hyphenrules{nohyphenation}
\noindent अर्थात् श्रीराम राजाधिराज हैं फिर भी उनके संपूर्ण कार्योंको आपने ही संपन्न किया। भाव यह है कि सर्वोपरि परब्रह्म राजाधिराज भगवान् श्रीरामको भी जिनकी निरन्तर अपेक्षा रहती है, तो हम जैसे जीवोंकी उनके बिना कैसी स्थिति होगी? यथा—
\end{sloppypar}
{\bfseries
\setlength{\mylenone}{0pt}
\settowidth{\mylentwo}{संकट-समाज असमंजस भो रामराज}
\setlength{\mylenone}{\maxof{\mylenone}{\mylentwo}}
\settowidth{\mylentwo}{काज जुग पूगनि को करतल पल भो}
\setlength{\mylenone}{\maxof{\mylenone}{\mylentwo}}
\setlength{\mylentwo}{\baselineskip}
\setlength{\mylenone}{\mylenone + 1pt}
\setlength{\mylen}{(\textwidth - \mylenone)*\real{0.5}}
\begin{longtable}[l]{@{\hspace*{\mylen}}>{\setlength\parfillskip{0pt}}p{\mylenone}@{}@{}l@{}}
 & \\[-\the\mylentwo]
संकट-समाज असमंजस भो रामराज & \\ \nopagebreak
काज जुग पूगनि को करतल पल भो & ।\\ \nopagebreak
\caption*{—ह.बा. ६}
\end{longtable}
}
\paraseplotus
\pagebreak


\fancyhead[LE,RO]{{\textmd{\large चौ. २८: और मनोरथ जो कोइ लावै}}}
\phantomsection
\addcontentsline{toc}{section}{चौपाई २८: और मनोरथ जो कोइ लावै}
\centering{॥ श्रीराम ॥}
\begin{sloppypar}\justifying\hyphenrules{nohyphenation}
मूल (चौपाई)—
\end{sloppypar}

{\bfseries\relscale{1.2}
\setlength{\mylenone}{0pt}
\settowidth{\mylentwo}{और मनोरथ जो कोइ लावै}
\setlength{\mylenone}{\maxof{\mylenone}{\mylentwo}}
\settowidth{\mylentwo}{तासु अमित जीवन फल पावै}
\setlength{\mylenone}{\maxof{\mylenone}{\mylentwo}}
\setlength{\mylentwo}{\baselineskip}
\setlength{\mylenone}{\mylenone + 1pt}
\setlength{\mylen}{(\textwidth - \mylenone)*\real{0.5}}
\begin{longtable}[l]{@{\hspace*{\mylen}}>{\setlength\parfillskip{0pt}}p{\mylenone}@{}@{}l@{}}
 & \\[-\the\mylentwo]
और मनोरथ जो कोइ लावै & ।\\ \nopagebreak[1mm]
तासु अमित जीवन फल पावै & ॥ २८ ॥
\end{longtable}
}

\parasepone
\index[ardha]{औरमनोरथजोकोइलावै(चौ.२८,पूर्वार्ध)@और मनोरथ जो कोइ लावै (चौ. २८, पूर्वार्ध)}
\index[ardha]{तासुअमितजीवनफलपावै(चौ.२८,उत्तरार्ध)@तासु अमित जीवन फल पावै (चौ. २८, उत्तरार्ध)}
\index[pada]{और@और}
\index[pada]{मनोरथ@मनोरथ}
\index[pada]{जो@जो}
\index[pada]{कोइ@कोइ}
\index[pada]{लावै@लावै}
\index[pada]{तासु@तासु}
\index[pada]{अमित@अमित}
\index[pada]{जीवन@जीवन}
\index[pada]{फल@फल}
\index[pada]{पावै@पावै}
\begin{sloppypar}\justifying\hyphenrules{nohyphenation}
\textbf{शब्दार्थ}—\textbf{\textit{मनोरथ}} {\unifont{\relscale{0.7}▶}} अभिलाषा। \textbf{\textit{अमित}} {\unifont{\relscale{0.7}▶}} असीम।
\end{sloppypar}
\begin{sloppypar}\justifying\hyphenrules{nohyphenation}
\textbf{अर्थ}—हे प्रभो! और जो आपके समक्ष कोई भी मनोरथ लेकर आता है, उस मनोरथका अपने इसी जीवनमें असीम फल पाता है।
\end{sloppypar}
\parasepone
\begin{sloppypar}\justifying\hyphenrules{nohyphenation}
\textbf{व्याख्या}—\textbf{\textit{सोइ अमित जीवन फल पावै}} यह पाठ माननेपर “वह व्यक्ति इसी जीवनमें उस इच्छाका असीम फल पा लेता है” अर्थ होगा। भाव यह है कि मानवकी इच्छाएँ प्रायः पूर्ण नहीं होतीं, यदि होती भी हैं तो शरीरान्तके पश्चात्। परन्तु हनुमान्‌जीके समक्ष इसी जीवनमें समस्त इच्छाएँ पूर्ण हो जाती हैं। यथा—\textbf{नाम कलि-कामतरु केसरी-कुमार को} (ह.बा.~९)।
\end{sloppypar}
\paraseplotus
\pagebreak


\fancyhead[LE,RO]{{\textmd{\large चौ. २९: चारिउ जुग परताप तुम्हारा}}}
\phantomsection
\addcontentsline{toc}{section}{चौपाई २९: चारिउ जुग परताप तुम्हारा}
\centering{॥ श्रीराम ॥}
\begin{sloppypar}\justifying\hyphenrules{nohyphenation}
मूल (चौपाई)—
\end{sloppypar}

{\bfseries\relscale{1.2}
\setlength{\mylenone}{0pt}
\settowidth{\mylentwo}{चारिउ जुग परताप तुम्हारा}
\setlength{\mylenone}{\maxof{\mylenone}{\mylentwo}}
\settowidth{\mylentwo}{है परसिद्ध जगत-उजियारा}
\setlength{\mylenone}{\maxof{\mylenone}{\mylentwo}}
\setlength{\mylentwo}{\baselineskip}
\setlength{\mylenone}{\mylenone + 1pt}
\setlength{\mylen}{(\textwidth - \mylenone)*\real{0.5}}
\begin{longtable}[l]{@{\hspace*{\mylen}}>{\setlength\parfillskip{0pt}}p{\mylenone}@{}@{}l@{}}
 & \\[-\the\mylentwo]
चारिउ जुग परताप तुम्हारा & ।\\ \nopagebreak[1mm]
है परसिद्ध जगत-उजियारा & ॥ २९ ॥
\end{longtable}
}

\parasepone
\begin{sloppypar}\justifying\hyphenrules{nohyphenation}
\textbf{शब्दार्थ}—\textbf{\textit{उजियारा}} {\unifont{\relscale{0.7}▶}} उजाला।
\end{sloppypar} 
\begin{sloppypar}\justifying\hyphenrules{nohyphenation}
\textbf{अर्थ}—हे प्रभो! आपका प्रताप कृतयुग, त्रेतायुग, द्वापरयुग, एवं कलियुग—इन चारों युगोंमें प्रसिद्ध है। इससे जगत्‌में उजाला छाया हुआ है।
\end{sloppypar}
\parasepone
\index[ardha]{चारिउजुगपरतापतुम्हारा(चौ.२९,पूर्वार्ध)@चारिउ जुग परताप तुम्हारा (चौ. २९, पूर्वार्ध)}
\index[ardha]{हैपरसिद्धजगतउजियारा(चौ.२९,उत्तरार्ध)@है परसिद्ध जगत-उजियारा (चौ. २९, उत्तरार्ध)}
\index[pada]{चारिउ@चारिउ}
\index[pada]{जुग@जुग}
\index[pada]{परताप@परताप}
\index[pada]{तुम्हारा@तुम्हारा}
\index[pada]{है@है}
\index[pada]{परसिद्ध@परसिद्ध}
\index[pada]{जगत@जगत}
\index[pada]{उजियारा@उजियारा}
\begin{sloppypar}\justifying\hyphenrules{nohyphenation}
\textbf{व्याख्या}—श्रीराम सार्वकालिक हैं। अतः प्राकट्यके पहले स्वायम्भुव मन्वन्तरके कृतयुगमें मनु-शतरूपाको द्विभुज रूपमें ही दर्शन दिया। यथा—
\end{sloppypar}
{\bfseries
\setlength{\mylenone}{0pt}
\settowidth{\mylentwo}{भृकुटि बिलास जासु जग होई}
\setlength{\mylenone}{\maxof{\mylenone}{\mylentwo}}
\settowidth{\mylentwo}{राम बाम दिशि सीता सोई}
\setlength{\mylenone}{\maxof{\mylenone}{\mylentwo}}
\setlength{\mylentwo}{\baselineskip}
\setlength{\mylenone}{\mylenone + 1pt}
\setlength{\mylen}{(\textwidth - \mylenone)*\real{0.5}}
\begin{longtable}[l]{@{\hspace*{\mylen}}>{\setlength\parfillskip{0pt}}p{\mylenone}@{}@{}l@{}}
 & \\[-\the\mylentwo]
भृकुटि बिलास जासु जग होई & ।\\ \nopagebreak
राम बाम दिशि सीता सोई & ॥\\ \nopagebreak
\caption*{—रा.च.मा. १-१४८-४}
\end{longtable}
}
\begin{sloppypar}\justifying\hyphenrules{nohyphenation}
\noindent उसी प्रकार इनका नाम भी चारों युगोंमें प्रसिद्ध है, यथा—\textbf{चहुँ जुग चहुँ श्रुति नाम प्रभाऊ} (रा.च.मा.१-२२-८)। अतः कृतयुगमें प्रह्लाद भी रामनाम जपते थे। यथा—\textbf{राम कहाँ सब ठौर हैं खम्भ~में हाँ सुनि हाँक नृकेहरि जागे} (क. ७-१२८)। प्रह्लाद दैत्य-बालकोंसे स्वयं कहते हैं—
\end{sloppypar}
{\bfseries
\setlength{\mylenone}{0pt}
\settowidth{\mylentwo}{रामनामजपतां कुतो भयं सर्वतापशमनैकभेषजम्}
\setlength{\mylenone}{\maxof{\mylenone}{\mylentwo}}
\settowidth{\mylentwo}{पश्य तात मम गात्रसन्निधौ पावकोऽपि सलिलायतेऽधुना}
\setlength{\mylenone}{\maxof{\mylenone}{\mylentwo}}
\setlength{\mylentwo}{\baselineskip}
\setlength{\mylenone}{\mylenone + 1pt}
\setlength{\mylen}{(\textwidth - \mylenone)*\real{0.5}}
\begin{longtable}[l]{@{\hspace*{\mylen}}>{\setlength\parfillskip{0pt}}p{\mylenone}@{}@{}l@{}}
 & \\[-\the\mylentwo]
रामनामजपतां कुतो भयं सर्वतापशमनैकभेषजम् & ।\\ \nopagebreak
पश्य तात मम गात्रसन्निधौ पावकोऽपि सलिलायतेऽधुना & ॥\\ \nopagebreak
\end{longtable}
}

\begin{sloppypar}\justifying\hyphenrules{nohyphenation}
\noindent तद्वत् श्रीरामजीके परिकर श्रीहनुमान्‌जी भी चारों युगोंमें रहते हैं। वैवस्वत मन्वन्तरके २४वें त्रेतायुगके अन्तमें प्रभुका प्राकट्य हुआ। पश्चात् प्रभुने आञ्जनेयको तब तकके लिए अमरत्व प्रदान किया, जब तक श्रीराम-कथाका प्रवाह धराधामपर अक्षुण्ण रहे। अतः उस समयसे अद्यावधि श्रीराम-कथाके साथ आञ्जनेयका स्वस्थ रहना सुस्पष्ट है। यह २८वाँ कलियुग है। प्रभुके प्राकट्यसे आज तक चार चतुर्युगियाँ बीत ही गईं। अतः हनुमान्‌जीके अमरत्वमें किमपि संदेह नहीं है।
{\bfseries
\setlength{\mylenone}{0pt}
\settowidth{\mylentwo}{कृते साकेतलोके स्यात्त्रेतायामवधे तथा}
\setlength{\mylenone}{\maxof{\mylenone}{\mylentwo}}
\settowidth{\mylentwo}{द्वापरे पार्थकेतौ तु कलौ राज्यं करिष्यति}
\setlength{\mylenone}{\maxof{\mylenone}{\mylentwo}}
\setlength{\mylentwo}{\baselineskip}
\setlength{\mylenone}{\mylenone + 1pt}
\setlength{\mylen}{(\textwidth - \mylenone)*\real{0.5}}
\begin{longtable}[l]{@{\hspace*{\mylen}}>{\setlength\parfillskip{0pt}}p{\mylenone}@{}@{}l@{}}
 & \\[-\the\mylentwo]
कृते साकेतलोके स्यात्त्रेतायामवधे तथा & ।\\ \nopagebreak
द्वापरे पार्थकेतौ तु कलौ राज्यं करिष्यति & ॥\\ \nopagebreak
\end{longtable}
}

\end{sloppypar}
\begin{sloppypar}\justifying\hyphenrules{nohyphenation}
\textbf{\textit{जगत-उजियारा}}—प्रतापकी उपमा सूर्यसे दी जाती है। यथा—\textbf{प्रभु-प्रताप-रबि छबिहिं न हरिही} (रा.च.मा. २-२०९-३)। इसलिए इससे जगत्‌में उजाला कहना उचित है।
\end{sloppypar}
\paraseplotus
\pagebreak


\fancyhead[LE,RO]{{\textmd{\large चौ. ३०: साधु संत के तुम रखवारे}}}
\phantomsection
\addcontentsline{toc}{section}{चौपाई ३०: साधु संत के तुम रखवारे}
\centering{॥ श्रीराम ॥}
\begin{sloppypar}\justifying\hyphenrules{nohyphenation}
मूल (चौपाई)—
\end{sloppypar}

{\bfseries\relscale{1.2}
\setlength{\mylenone}{0pt}
\settowidth{\mylentwo}{साधु संत के तुम रखवारे}
\setlength{\mylenone}{\maxof{\mylenone}{\mylentwo}}
\settowidth{\mylentwo}{असुर-निकंदन राम-दुलारे}
\setlength{\mylenone}{\maxof{\mylenone}{\mylentwo}}
\setlength{\mylentwo}{\baselineskip}
\setlength{\mylenone}{\mylenone + 1pt}
\setlength{\mylen}{(\textwidth - \mylenone)*\real{0.5}}
\begin{longtable}[l]{@{\hspace*{\mylen}}>{\setlength\parfillskip{0pt}}p{\mylenone}@{}@{}l@{}}
 & \\[-\the\mylentwo]
साधु संत के तुम रखवारे & ।\\ \nopagebreak[1mm]
असुर-निकंदन राम-दुलारे & ॥ ३० ॥
\end{longtable}
}

\parasepone
\index[ardha]{साधुसंतकेतुमरखवारे(चौ.३०,पूर्वार्ध)@साधु संत के तुम रखवारे (चौ. ३०, पूर्वार्ध)}
\index[ardha]{असुरनिकंदनरामदुलारे(चौ.३०,उत्तरार्ध)@असुर-निकंदन राम-दुलारे (चौ. ३०, उत्तरार्ध)}
\index[pada]{साधु@साधु}
\index[pada]{संत@संत}
\index[pada]{के@के}
\index[pada]{तुम@तुम}
\index[pada]{रखवारे@रखवारे}
\index[pada]{असुर@असुर}
\index[pada]{निकंदन@निकंदन}
\index[pada]{राम@राम}
\index[pada]{दुलारे@दुलारे}
\begin{sloppypar}\justifying\hyphenrules{nohyphenation}
\textbf{शब्दार्थ}—\textbf{\textit{असुर-निकंदन}} {\unifont{\relscale{0.7}▶}} राक्षसोंको मारने वाले।
\end{sloppypar}
\begin{sloppypar}\justifying\hyphenrules{nohyphenation}
\textbf{अर्थ}—हे राक्षसोंको नष्ट करने वाले श्रीरामजीके दुलारे श्रीहनुमान्‌जी महाराज! आप साधु तथा सन्तोंके रक्षक हैं।
\end{sloppypar}
\parasepone
\begin{sloppypar}\justifying\hyphenrules{nohyphenation}
\textbf{व्याख्या}—यहाँ \textbf{\textit{साधु}}\ \ शब्द साधनारत भक्तोंकी ओर संकेत करता है एवं \textbf{\textit{संत}} शब्द साधनसंपन्न भक्तको सूचित करता है। दोनोंको हनुमान्‌जीकी अपेक्षा है। साधु सुग्रीव तथा सन्त विभीषण दोनों हनुमान्‌जीसे रक्षित हैं। यथा—\textbf{दुर्जन~को काल सो कराल पाल सज्जन~को} (ह.बा. १०)।
\end{sloppypar}
\begin{sloppypar}\justifying\hyphenrules{nohyphenation}
\textbf{\textit{राम-दुलारे}}—हनुमान्‌जी राघवजीके दुलारे हैं। यथा—\textbf{राम~को दुलारो दास} (ह.बा. ९)।
\end{sloppypar}
\paraseplotus
\pagebreak


\fancyhead[LE,RO]{{\textmd{\large चौ. ३१: अष्ट सिद्धि नव निधि के दाता}}}
\phantomsection
\addcontentsline{toc}{section}{चौपाई ३१: अष्ट सिद्धि नव निधि के दाता}
\centering{॥ श्रीराम ॥}
\begin{sloppypar}\justifying\hyphenrules{nohyphenation}
मूल (चौपाई)—
\end{sloppypar}

{\bfseries\relscale{1.2}
\setlength{\mylenone}{0pt}
\settowidth{\mylentwo}{अष्ट सिद्धि नव निधि के दाता}
\setlength{\mylenone}{\maxof{\mylenone}{\mylentwo}}
\settowidth{\mylentwo}{अस बर दीन्ह जानकी माता}
\setlength{\mylenone}{\maxof{\mylenone}{\mylentwo}}
\setlength{\mylentwo}{\baselineskip}
\setlength{\mylenone}{\mylenone + 1pt}
\setlength{\mylen}{(\textwidth - \mylenone)*\real{0.5}}
\begin{longtable}[l]{@{\hspace*{\mylen}}>{\setlength\parfillskip{0pt}}p{\mylenone}@{}@{}l@{}}
 & \\[-\the\mylentwo]
अष्ट सिद्धि नव निधि के दाता & ।\\ \nopagebreak[1mm]
अस बर दीन्ह जानकी माता & ॥ ३१ ॥
\end{longtable}
}

\parasepone
\index[ardha]{अष्टसिद्धिनवनिधिकेदाता(चौ.३१,पूर्वार्ध)@अष्ट सिद्धि नव निधि के दाता (चौ. ३१, पूर्वार्ध)}
\index[ardha]{असबरदीन्हजानकीमाता(चौ.३१,उत्तरार्ध)@अस बर दीन्ह जानकी माता (चौ. ३१, उत्तरार्ध)}
\index[pada]{अष्ट@अष्ट}
\index[pada]{सिद्धि@सिद्धि}
\index[pada]{नव@नव}
\index[pada]{निधि@निधि}
\index[pada]{के@के}
\index[pada]{दाता@दाता}
\index[pada]{अस@अस}
\index[pada]{बर@बर}
\index[pada]{दीन्ह@दीन्ह}
\index[pada]{जानकी@जानकी}
\index[pada]{माता@माता}
\begin{sloppypar}\justifying\hyphenrules{nohyphenation}
\textbf{शब्दार्थ}—\textbf{\textit{दाता}} {\unifont{\relscale{0.7}▶}} देने वाले।
\end{sloppypar}
\begin{sloppypar}\justifying\hyphenrules{nohyphenation}
\textbf{अर्थ}—आप अष्ट सिद्धियों एवं नव निधियोंके देने वाले हैं। जनकनन्दिनी श्रीसीता माताने आपको ऐसा वरदान दिया है।
\end{sloppypar}
\parasepone
\begin{sloppypar}\justifying\hyphenrules{nohyphenation}
\textbf{व्याख्या}—अशोकवाटिकामें जब श्रीआञ्जनेयके वाक्चातुर्यसे माँ मैथिली भली-भाँति संतुष्ट हो गईं, तब इन्होंने आञ्जनेयको आशीर्वाद-पुष्पोंसे विभूषित कर दिया। यहाँ मायाकी सीता अविद्या अथवा विद्या रूपमें नहीं हैं, अपितु श्रीराघवकी लीलाशक्ति ही माया सीताके रूपमें वर्तमान हैं, अन्यथा मायाके मिथ्यात्वसे आञ्जनेयको दिए हुए आशीर्वाद भी प्रामाणिक न हो पाएँगे। इस प्रसंगका संकेत भावी किसी ग्रन्थमें किया जाएगा। अथ प्रकृतमनुसरामः। यथा—
\end{sloppypar}
{\bfseries
\setlength{\mylenone}{0pt}
\settowidth{\mylentwo}{आशिष दीन्ह राम प्रिय जाना}
\setlength{\mylenone}{\maxof{\mylenone}{\mylentwo}}
\settowidth{\mylentwo}{होहु तात बल शील निधाना}
\setlength{\mylenone}{\maxof{\mylenone}{\mylentwo}}
\settowidth{\mylentwo}{अजर अमर गुणनिधि सुत होहू}
\setlength{\mylenone}{\maxof{\mylenone}{\mylentwo}}
\settowidth{\mylentwo}{करहु बहुत रघुनायक छोहू}
\setlength{\mylenone}{\maxof{\mylenone}{\mylentwo}}
\settowidth{\mylentwo}{करिहिं कृपा प्रभु अस सुनि काना}
\setlength{\mylenone}{\maxof{\mylenone}{\mylentwo}}
\settowidth{\mylentwo}{निर्भर प्रेम मगन हनुमाना}
\setlength{\mylenone}{\maxof{\mylenone}{\mylentwo}}
\setlength{\mylentwo}{\baselineskip}
\setlength{\mylenone}{\mylenone + 1pt}
\setlength{\mylen}{(\textwidth - \mylenone)*\real{0.5}}
\begin{longtable}[l]{@{\hspace*{\mylen}}>{\setlength\parfillskip{0pt}}p{\mylenone}@{}@{}l@{}}
 & \\[-\the\mylentwo]
आशिष दीन्ह राम प्रिय जाना & ।\\ \nopagebreak
होहु तात बल शील निधाना & ॥\\
अजर अमर गुणनिधि सुत होहू & ।\\ \nopagebreak
करहु बहुत रघुनायक छोहू & ॥\\
करिहिं कृपा प्रभु अस सुनि काना & ।\\ \nopagebreak
निर्भर प्रेम मगन हनुमाना & ॥\\ \nopagebreak
\caption*{—रा.च.मा. ५-१७-२,३,४}
\end{longtable}
}
\begin{sloppypar}\justifying\hyphenrules{nohyphenation}
सिद्धियाँ आठ हैं—
\end{sloppypar}
{\bfseries
\setlength{\mylenone}{0pt}
\settowidth{\mylentwo}{अणिमा गरिमा चैव महिमा लघिमा तथा}
\setlength{\mylenone}{\maxof{\mylenone}{\mylentwo}}
\settowidth{\mylentwo}{प्राप्तिः प्राकाम्यमीशित्वं वशित्वं चाष्ट सिद्धयः}
\setlength{\mylenone}{\maxof{\mylenone}{\mylentwo}}
\setlength{\mylentwo}{\baselineskip}
\setlength{\mylenone}{\mylenone + 1pt}
\setlength{\mylen}{(\textwidth - \mylenone)*\real{0.5}}
\begin{longtable}[l]{@{\hspace*{\mylen}}>{\setlength\parfillskip{0pt}}p{\mylenone}@{}@{}l@{}}
 & \\[-\the\mylentwo]
अणिमा गरिमा चैव महिमा लघिमा तथा & ।\\ \nopagebreak
प्राप्तिः प्राकाम्यमीशित्वं वशित्वं चाष्ट सिद्धयः & ॥\\ \nopagebreak
\caption*{—अ.को. १-१-३५क}
\end{longtable}
}
\begin{sloppypar}\justifying\hyphenrules{nohyphenation}
\noindent अर्थात् अणिमा, गरिमा, महिमा, लघिमा, प्राप्ति, प्राकाम्य, ईशित्व, और वशित्व आठ सिद्धियाँ हैं। निधियाँ नौ हैं—महापद्म, पद्म, शङ्ख, मकर, कच्छप, मुकुन्द, कुन्द, नील, और खर्व। यथा—
\end{sloppypar}
{\bfseries
\setlength{\mylenone}{0pt}
\settowidth{\mylentwo}{महापद्मञ्च पद्मञ्च शङ्खो मकरकच्छपौ}
\setlength{\mylenone}{\maxof{\mylenone}{\mylentwo}}
\settowidth{\mylentwo}{मुकुन्दः कुन्दनीलौ च खर्वश्च निधयो नव}
\setlength{\mylenone}{\maxof{\mylenone}{\mylentwo}}
\setlength{\mylentwo}{\baselineskip}
\setlength{\mylenone}{\mylenone + 1pt}
\setlength{\mylen}{(\textwidth - \mylenone)*\real{0.5}}
\begin{longtable}[l]{@{\hspace*{\mylen}}>{\setlength\parfillskip{0pt}}p{\mylenone}@{}@{}l@{}}
 & \\[-\the\mylentwo]
महापद्मञ्च पद्मञ्च शङ्खो मकरकच्छपौ & ।\\ \nopagebreak
मुकुन्दः कुन्दनीलौ च खर्वश्च निधयो नव & ॥\\ \nopagebreak
\caption*{— अ.को. १-१-७१क}
\end{longtable}
}
\paraseplotus
\pagebreak


\fancyhead[LE,RO]{{\textmd{\large चौ. ३२: राम-रसायन तुम्हरे पासा}}}
\phantomsection
\addcontentsline{toc}{section}{चौपाई ३२: राम-रसायन तुम्हरे पासा}
\centering{॥ श्रीराम ॥}
\begin{sloppypar}\justifying\hyphenrules{nohyphenation}
मूल (चौपाई)—
\end{sloppypar}

{\bfseries\relscale{1.2}
\setlength{\mylenone}{0pt}
\settowidth{\mylentwo}{राम-रसायन तुम्हरे पासा}
\setlength{\mylenone}{\maxof{\mylenone}{\mylentwo}}
\settowidth{\mylentwo}{सादर हौ रघुपति के दासा}
\setlength{\mylenone}{\maxof{\mylenone}{\mylentwo}}
\setlength{\mylentwo}{\baselineskip}
\setlength{\mylenone}{\mylenone + 1pt}
\setlength{\mylen}{(\textwidth - \mylenone)*\real{0.5}}
\begin{longtable}[l]{@{\hspace*{\mylen}}>{\setlength\parfillskip{0pt}}p{\mylenone}@{}@{}l@{}}
 & \\[-\the\mylentwo]
राम-रसायन तुम्हरे पासा & ।\\ \nopagebreak[1mm]
सादर हौ रघुपति के दासा & ॥ ३२ ॥
\end{longtable}
}

\parasepone
\index[ardha]{रामरसायनतुम्हरेपासा(चौ.३२,पूर्वार्ध)@राम-रसायन तुम्हरे पासा (चौ. ३२, पूर्वार्ध)}
\index[ardha]{सादरहौरघुपतिकेदासा(चौ.३२,उत्तरार्ध)@सादर हौ रघुपति के दासा (चौ. ३२, उत्तरार्ध)}
\index[pada]{राम@राम}
\index[pada]{रसायन@रसायन}
\index[pada]{तुम्हरे@तुम्हरे}
\index[pada]{पासा@पासा}
\index[pada]{सादर@सादर}
\index[pada]{हौ@हौ}
\index[pada]{रघुपति@रघुपति}
\index[pada]{के@के}
\index[pada]{दासा@दासा}
\begin{sloppypar}\justifying\hyphenrules{nohyphenation}
\textbf{शब्दार्थ}—\textbf{\textit{राम-रसायन}} {\unifont{\relscale{0.7}▶}} श्रीरामरसका भाण्डागार (भण्डार)।
\end{sloppypar}
\begin{sloppypar}\justifying\hyphenrules{nohyphenation}
\textbf{अर्थ}—हे पवननन्दन! रामप्रेमरसका भण्डार आपके ही पास है एवं आप निरन्तर आदरपूर्वक रघुपति श्रीरामभद्रजूके दास्य भावमें रहते हैं। यद्वा, हे आदरणीय रघुपति श्रीरामभद्रजूके दास हनुमान्‌जी! आपके पास रामप्रेमरसका भवन अर्थात् राघवचरित्र निरन्तर निवास करता है।
\end{sloppypar}
\parasepone
\begin{sloppypar}\justifying\hyphenrules{nohyphenation}
\textbf{व्याख्या}—श्रीहनुमान्‌जी रामभक्ति-रसके आचार्य ही हैं। अत एव इन्होंने श्रीराघवसे अनपायनी भक्ति माँगी—
\end{sloppypar}
{\bfseries
\setlength{\mylenone}{0pt}
\settowidth{\mylentwo}{नाथ भगति तव अति सुखदायिनि}
\setlength{\mylenone}{\maxof{\mylenone}{\mylentwo}}
\settowidth{\mylentwo}{देहु कृपा करि सो अनपायनि}
\setlength{\mylenone}{\maxof{\mylenone}{\mylentwo}}
\setlength{\mylentwo}{\baselineskip}
\setlength{\mylenone}{\mylenone + 1pt}
\setlength{\mylen}{(\textwidth - \mylenone)*\real{0.5}}
\begin{longtable}[l]{@{\hspace*{\mylen}}>{\setlength\parfillskip{0pt}}p{\mylenone}@{}@{}l@{}}
 & \\[-\the\mylentwo]
नाथ भगति तव अति सुखदायिनि & ।\\ \nopagebreak
देहु कृपा करि सो अनपायनि & ॥\\ \nopagebreak
\caption*{—रा.च.मा. ५-३४-३}
\end{longtable}
}
\begin{sloppypar}\justifying\hyphenrules{nohyphenation}
\noindent इन्हें गोस्वामीजीने \textit{रसाइनी} (संस्कृत: \textit{रसायनी}) शब्दसे संबोधित किया है। यथा—\textbf{राम~की रजाइ~तें रसाइनी समीरसूनु} (क. ५-२५)।
\end{sloppypar}
\paraseplotus
\pagebreak


\fancyhead[LE,RO]{{\textmd{\large चौ. ३३: तुम्हरे भजन राम को पावै}}}
\phantomsection
\addcontentsline{toc}{section}{चौपाई ३३: तुम्हरे भजन राम को पावै}
\centering{॥ श्रीराम ॥}
\begin{sloppypar}\justifying\hyphenrules{nohyphenation}
मूल (चौपाई)—
\end{sloppypar}

{\bfseries\relscale{1.2}
\setlength{\mylenone}{0pt}
\settowidth{\mylentwo}{तुम्हरे भजन राम को पावै}
\setlength{\mylenone}{\maxof{\mylenone}{\mylentwo}}
\settowidth{\mylentwo}{जनम जनम के दुख बिसरावै}
\setlength{\mylenone}{\maxof{\mylenone}{\mylentwo}}
\setlength{\mylentwo}{\baselineskip}
\setlength{\mylenone}{\mylenone + 1pt}
\setlength{\mylen}{(\textwidth - \mylenone)*\real{0.5}}
\begin{longtable}[l]{@{\hspace*{\mylen}}>{\setlength\parfillskip{0pt}}p{\mylenone}@{}@{}l@{}}
 & \\[-\the\mylentwo]
तुम्हरे भजन राम को पावै & ।\\ \nopagebreak[1mm]
जनम जनम के दुख बिसरावै & ॥ ३३ ॥
\end{longtable}
}

\parasepone
\index[ardha]{तुम्हरेभजनरामकोपावै(चौ.३३,पूर्वार्ध)@तुम्हरे भजन राम को पावै (चौ. ३३, पूर्वार्ध)}
\index[ardha]{जनमजनमकेदुखबिसरावै(चौ.३३,उत्तरार्ध)@जनम जनम के दुख बिसरावै (चौ. ३३, उत्तरार्ध)}
\index[pada]{तुम्हरे@तुम्हरे}
\index[pada]{भजन@भजन}
\index[pada]{राम@राम}
\index[pada]{को@को}
\index[pada]{पावै@पावै}
\index[pada]{जनम@जनम}
\index[pada]{जनम@जनम}
\index[pada]{के@के}
\index[pada]{दुख@दुख}
\index[pada]{बिसरावै@बिसरावै}
\begin{sloppypar}\justifying\hyphenrules{nohyphenation}
\textbf{शब्दार्थ}—\textbf{\textit{भजन}} {\unifont{\relscale{0.7}▶}} सेवा, शरणागति।
\end{sloppypar}
\begin{sloppypar}\justifying\hyphenrules{nohyphenation}
\textbf{अर्थ}—हे कपिकुल-तिलक! आपके भजनसे साधक श्रीरामभद्रजूको पा जाता है और प्रभुको पाकर वह अनेक जन्मोंके दुःखोंको भूल जाता है।
\end{sloppypar}
\parasepone
\begin{sloppypar}\justifying\hyphenrules{nohyphenation}
\textbf{व्याख्या}—आपका भजन भगवत्प्राप्तिमें साधन है क्योंकि भगवद्भक्ति दर्शनके लिए ज्ञान और वैराग्य इन दो नेत्रोंकी आवश्यकता होती है। यथा—\textbf{ज्ञान बिराग नयन उरगारी} (रा.च.मा. ७-१२०-१४)। और आप स्वयं ज्ञान-वैराग्य-रूप हैं। यथा—\textbf{ज्ञानिनामग्रगण्यम्} (रा.च.मा. ५-मङ्गलाचरण श्लोक~३) और \textbf{बिरागी पवनकुमार सो} (क. ५-१)। अतः आञ्जनेयके भजनसे निश्चित श्रीरामरूपकी प्राप्ति हो जाती है। राघव सुखसिन्धु हैं, अतः उन्हें प्राप्त कर व्यक्ति अनेक जन्मोंके दुःखोंको उन्हीं आनन्द-सिन्धुकी एक लहरमें विलीन कर देता है, जैसे जटायु—\textbf{निरखि राम छबि-धाम मुख बिगत भई सब पीर} (रा.च.मा. ३-३२)।
\end{sloppypar}
\paraseplotus
\pagebreak


\fancyhead[LE,RO]{{\textmd{\large चौ. ३४: अंत-काल रघुबर-पुर जाई}}}
\phantomsection
\addcontentsline{toc}{section}{चौपाई ३४: अंत-काल रघुबर-पुर जाई}
\centering{॥ श्रीराम ॥}
\begin{sloppypar}\justifying\hyphenrules{nohyphenation}
मूल (चौपाई)—
\end{sloppypar}

{\bfseries\relscale{1.2}
\setlength{\mylenone}{0pt}
\settowidth{\mylentwo}{अंत-काल रघुबर-पुर जाई}
\setlength{\mylenone}{\maxof{\mylenone}{\mylentwo}}
\settowidth{\mylentwo}{जहाँ जन्म हरि-भगत  कहाई}
\setlength{\mylenone}{\maxof{\mylenone}{\mylentwo}}
\setlength{\mylentwo}{\baselineskip}
\setlength{\mylenone}{\mylenone + 1pt}
\setlength{\mylen}{(\textwidth - \mylenone)*\real{0.5}}
\begin{longtable}[l]{@{\hspace*{\mylen}}>{\setlength\parfillskip{0pt}}p{\mylenone}@{}@{}l@{}}
 & \\[-\the\mylentwo]
अंत-काल रघुबर-पुर जाई & ।\\ \nopagebreak[1mm]
जहाँ जन्म हरि-भगत कहाई & ॥ ३४ ॥
\end{longtable}
}

\parasepone
\index[ardha]{अंतकालरघुबरपुरजाई(चौ.३४,पूर्वार्ध)@अंत-काल रघुबर-पुर जाई (चौ. ३४, पूर्वार्ध)}
\index[ardha]{जहाँजन्महरिभगतकहाई(चौ.३४,उत्तरार्ध)@जहाँ जन्म हरि-भगत  कहाई (चौ. ३४, उत्तरार्ध)}
\index[pada]{अंत@अंत}
\index[pada]{काल@काल}
\index[pada]{रघुबर-पुर@रघुबर-पुर}
\index[pada]{जाई@जाई}
\index[pada]{जहाँ@जहाँ}
\index[pada]{जन्म@जन्म}
\index[pada]{हरि@हरि}
\index[pada]{भगत@भगत}
\index[pada]{कहाई@कहाई}
\begin{sloppypar}\justifying\hyphenrules{nohyphenation}
\textbf{शब्दार्थ}—\textbf{\textit{रघुबर-पुर}} {\unifont{\relscale{0.7}▶}} श्रीसाकेत\-लोक।
\end{sloppypar}
\begin{sloppypar}\justifying\hyphenrules{nohyphenation}
\textbf{अर्थ}—हे राम-दूत! आपके भजनके प्रतापसे वह साधक इसी भौतिक शरीरसे भगवान् श्रीरामभद्रका दर्शन करके शरीरावसान (अन्तकाल)के समय श्रीसाकेत\-लोक जाकर पुनः मर्त्यलोकमें जहाँ भी जन्म लेता है, वहाँ श्रीहरिका भक्त ही कहलाता है। अर्थात् पुनर्जन्ममें भी उसके भक्तिके संस्कार धूमिल नहीं होते।
\end{sloppypar}
\parasepone
\begin{sloppypar}\justifying\hyphenrules{nohyphenation}
\textbf{व्याख्या}—इसी शरीरसे साधक इसी लोकमें इन्हीं चक्षुओंसे श्रीरामजूके कोटि-कोटि-कन्दर्प-दर्प-दलन सगुण साकार नीलनीरधरश्याम लोकाभिराम श्रीविग्रहका दर्शन करता है और अन्तकालमें वह मोक्ष नहीं चाहता। क्योंकि—\textbf{सगुनोपासक मोक्ष न लेहीं} (रा.च.मा. ६-११२-७)। फिर प्रारब्धका क्षय करके शरीरको विसर्जित कर अपनी इच्छानुसार श्रीसाकेत\-लोकमें विश्राम कर पुनः भगवल्लीलारसकी अनुभूतिकी लालसासे भगवान् श्रीराघवके अवतार-कालमें संसारमें आकर किसी भाग्यशालिनी माँकी कोखको पवित्र कर रघुनाथजीकी प्रतीक्षामें निरत रहता है। यथा—
\end{sloppypar}
{\bfseries
\setlength{\mylenone}{0pt}
\settowidth{\mylentwo}{निज इच्छा प्रभु अवतरइ सुर महि गो द्विज लागि}
\setlength{\mylenone}{\maxof{\mylenone}{\mylentwo}}
\settowidth{\mylentwo}{सगुन उपासक संग तहँ रहहिं मोक्ष सुख त्यागि}
\setlength{\mylenone}{\maxof{\mylenone}{\mylentwo}}
\setlength{\mylentwo}{\baselineskip}
\setlength{\mylenone}{\mylenone + 1pt}
\setlength{\mylen}{(\textwidth - \mylenone)*\real{0.5}}
\begin{longtable}[l]{@{\hspace*{\mylen}}>{\setlength\parfillskip{0pt}}p{\mylenone}@{}@{}l@{}}
 & \\[-\the\mylentwo]
निज इच्छा प्रभु अवतरइ सुर महि गो द्विज लागि & ।\\ \nopagebreak
सगुन उपासक संग तहँ रहहिं मोक्ष सुख त्यागि & ॥\\ \nopagebreak
\caption*{—रा.च.मा. ४-२६}
\end{longtable}
}
\paraseplotus
\pagebreak


\fancyhead[LE,RO]{{\textmd{\large चौ. ३५: और देवता चित्त न धरई}}}
\phantomsection
\addcontentsline{toc}{section}{चौपाई ३५: और देवता चित्त न धरई}
\centering{॥ श्रीराम ॥}
\begin{sloppypar}\justifying\hyphenrules{nohyphenation}
मूल (चौपाई)—
\end{sloppypar}

{\bfseries\relscale{1.2}
\setlength{\mylenone}{0pt}
\settowidth{\mylentwo}{और देवता चित्त न धरई}
\setlength{\mylenone}{\maxof{\mylenone}{\mylentwo}}
\settowidth{\mylentwo}{हनुमत सेइ सर्ब सुख करई}
\setlength{\mylenone}{\maxof{\mylenone}{\mylentwo}}
\setlength{\mylentwo}{\baselineskip}
\setlength{\mylenone}{\mylenone + 1pt}
\setlength{\mylen}{(\textwidth - \mylenone)*\real{0.5}}
\begin{longtable}[l]{@{\hspace*{\mylen}}>{\setlength\parfillskip{0pt}}p{\mylenone}@{}@{}l@{}}
 & \\[-\the\mylentwo]
और देवता चित्त न धरई & ।\\ \nopagebreak[1mm]
हनुमत सेइ सर्ब सुख करई & ॥ ३५ ॥
\end{longtable}
}

\parasepone
\index[ardha]{औरदेवताचित्तनधरई(चौ.३५,पूर्वार्ध)@और देवता चित्त न धरई (चौ. ३५, पूर्वार्ध)}
\index[ardha]{हनुमतसेइसर्बसुखकरई(चौ.३५,उत्तरार्ध)@हनुमत सेइ सर्ब सुख करई (चौ. ३५, उत्तरार्ध)}
\index[pada]{और@और}
\index[pada]{देवता@देवता}
\index[pada]{चित्त@चित्त}
\index[pada]{न@न}
\index[pada]{धरई@धरई}
\index[pada]{हनुमत@हनुमत}
\index[pada]{सेइ@सेइ}
\index[pada]{सर्ब@सर्ब}
\index[pada]{सुख@सुख}
\index[pada]{करई@करई}
\begin{sloppypar}\justifying\hyphenrules{nohyphenation}
\textbf{शब्दार्थ}—\textbf{\textit{सर्ब सुख}} {\unifont{\relscale{0.7}▶}} लौकिक एवं पारलौकिक अनुकूलता।
\end{sloppypar}
\begin{sloppypar}\justifying\hyphenrules{nohyphenation}
\textbf{अर्थ}—जो भक्त किसी अन्य देवताको अपने चित्तमें न धारण कर केवल हनुमान्‌जीकी सेवा करता है, वह समस्त सुखोंको प्राप्त कर लेता है। यद्वा, जो अन्य किसी देवताको अपने चित्तमें नहीं धारण करता, वह भी हनुमान्‌जीकी सेवा करके समस्त सुखोंकी प्राप्ति कर लेता है।
\end{sloppypar}
\parasepone
\begin{sloppypar}\justifying\hyphenrules{nohyphenation}
\textbf{व्याख्या}—देवता-विमुखको नास्तिक कहा जाता है, तथा उसे सुखकी प्राप्ति नहीं होती क्योंकि इष्ट भोगोंको देने वाले देवता ही हैं। यथा—\textbf{इष्टान् भोगान् हि वो देवा दास्यन्ते यज्ञभाविताः} (भ.गी. ३-१२)। पर वह भी हनुमत्कृपासे सर्वसुखका अधिकारी हो सकता है क्योंकि समस्त देव उन्हींकी कृपाकी अपेक्षा रखते हैं। यथा—\textbf{देवी देव दानव दयावने ह्वै जोरैं हाथ बापुरे बराक कहा और राजा राँक~को} (ह.बा. १२)। अथवा, अनन्य निष्ठासे सेवा करनेपर हनुमान्‌जी समस्त सुख प्रदान करते हैं।
\end{sloppypar}
\paraseplotus
\pagebreak


\fancyhead[LE,RO]{{\textmd{\large चौ. ३६: संकट कटै मिटै सब पीरा}}}
\phantomsection
\addcontentsline{toc}{section}{चौपाई ३६: संकट कटै मिटै सब पीरा}
\centering{॥ श्रीराम ॥}
\begin{sloppypar}\justifying\hyphenrules{nohyphenation}
मूल (चौपाई)—
\end{sloppypar}

{\bfseries\relscale{1.2}
\setlength{\mylenone}{0pt}
\settowidth{\mylentwo}{संकट कटै मिटै सब पीरा}
\setlength{\mylenone}{\maxof{\mylenone}{\mylentwo}}
\settowidth{\mylentwo}{जो सुमिरै हनुमत बलबीरा}
\setlength{\mylenone}{\maxof{\mylenone}{\mylentwo}}
\setlength{\mylentwo}{\baselineskip}
\setlength{\mylenone}{\mylenone + 1pt}
\setlength{\mylen}{(\textwidth - \mylenone)*\real{0.5}}
\begin{longtable}[l]{@{\hspace*{\mylen}}>{\setlength\parfillskip{0pt}}p{\mylenone}@{}@{}l@{}}
 & \\[-\the\mylentwo]
संकट कटै मिटै सब पीरा & ।\\ \nopagebreak[1mm]
जो सुमिरै हनुमत बलबीरा & ॥ ३६ ॥
\end{longtable}
}

\parasepone
\index[ardha]{संकटकटैमिटैसबपीरा(चौ.३६,पूर्वार्ध)@संकट कटै मिटै सब पीरा (चौ. ३६, पूर्वार्ध)}
\index[ardha]{जोसुमिरैहनुमतबलबीरा(चौ.३६,उत्तरार्ध)@जो सुमिरै हनुमत बलबीरा (चौ. ३६, उत्तरार्ध)}
\index[pada]{संकट@संकट}
\index[pada]{कटै@कटै}
\index[pada]{मिटै@मिटै}
\index[pada]{सब@सब}
\index[pada]{पीरा@पीरा}
\index[pada]{जो@जो}
\index[pada]{सुमिरै@सुमिरै}
\index[pada]{हनुमत@हनुमत}
\index[pada]{बलबीरा@बलबीरा}
\begin{sloppypar}\justifying\hyphenrules{nohyphenation}
\textbf{शब्दार्थ}—\textbf{\textit{बलबीरा}} {\unifont{\relscale{0.7}▶}} बलसे युक्त वीर।
\end{sloppypar}
\begin{sloppypar}\justifying\hyphenrules{nohyphenation}
\textbf{अर्थ}—जो अतुलनीय-बलयुक्त वीर श्रीहनुमान्‌जी महाराजका स्मरण करता है, उसके समस्त संकट कट जाते हैं तथा सभी पीड़ाएँ मिट जाती हैं।
\end{sloppypar}
\parasepone
\begin{sloppypar}\justifying\hyphenrules{nohyphenation}
\textbf{व्याख्या}—हनुमान्‌जीके स्मरणसे श्रीशिव-पार्वती एवं श्रीसीता-राम-लक्ष्मण प्रसन्न होकर साधकके पञ्चक्लेश (अविद्या, अस्मिता, राग, द्वेष, और अभिनिवेश) को हर लेते हैं। यथा—
\end{sloppypar}
{\bfseries
\setlength{\mylenone}{0pt}
\settowidth{\mylentwo}{सानुग सगौरि सानुकूल सूलपानि ताहि}
\setlength{\mylenone}{\maxof{\mylenone}{\mylentwo}}
\settowidth{\mylentwo}{लोकपाल सकल लखन राम जानकी}
\setlength{\mylenone}{\maxof{\mylenone}{\mylentwo}}
\settowidth{\mylentwo}{लोक परलोक को बिसोक सो तिलोक ताहि}
\setlength{\mylenone}{\maxof{\mylenone}{\mylentwo}}
\settowidth{\mylentwo}{तुलसी तमाहि ताहि काहू बीर आन की}
\setlength{\mylenone}{\maxof{\mylenone}{\mylentwo}}
\settowidth{\mylentwo}{केसरी-किसोर बंदीछोर के नेवाजे सब}
\setlength{\mylenone}{\maxof{\mylenone}{\mylentwo}}
\settowidth{\mylentwo}{कीरति बिमल कपि करुनानिधान की}
\setlength{\mylenone}{\maxof{\mylenone}{\mylentwo}}
\settowidth{\mylentwo}{बालक ज्यों पालिहैं कृपालु मुनि सिद्ध ताको}
\setlength{\mylenone}{\maxof{\mylenone}{\mylentwo}}
\settowidth{\mylentwo}{जाके हिये हुलसति हाँक हनुमान की}
\setlength{\mylenone}{\maxof{\mylenone}{\mylentwo}}
\setlength{\mylentwo}{\baselineskip}
\setlength{\mylenone}{\mylenone + 1pt}
\setlength{\mylen}{(\textwidth - \mylenone)*\real{0.5}}
\begin{longtable}[l]{@{\hspace*{\mylen}}>{\setlength\parfillskip{0pt}}p{\mylenone}@{}@{}l@{}}
 & \\[-\the\mylentwo]
सानुग सगौरि सानुकूल सूलपानि ताहि & \\ \nopagebreak
लोकपाल सकल लखन राम जानकी & ।\\
लोक परलोक को बिसोक सो तिलोक ताहि & \\ \nopagebreak
तुलसी तमाहि ताहि काहू बीर आन की & ।\\
केसरी-किसोर बंदीछोर के नेवाजे सब & \\ \nopagebreak
कीरति बिमल कपि करुनानिधान की & ।\\
बालक ज्यों पालिहैं कृपालु मुनि सिद्ध ताको & \\ \nopagebreak
जाके हिये हुलसति हाँक हनुमान की & ॥\\ \nopagebreak
\caption*{—ह.बा. १३}
\end{longtable}
}
\paraseplotus
\pagebreak


\fancyhead[LE,RO]{{\textmd{\large चौ. ३७: जय जय जय हनुमान गोसाईं}}}
\phantomsection
\addcontentsline{toc}{section}{चौपाई ३७: जय जय जय हनुमान गोसाईं}
\centering{॥ श्रीराम ॥}
\begin{sloppypar}\justifying\hyphenrules{nohyphenation}
मूल (चौपाई)—
\end{sloppypar}

{\bfseries\relscale{1.2}
\setlength{\mylenone}{0pt}
\settowidth{\mylentwo}{जय जय जय हनुमान गोसाईं}
\setlength{\mylenone}{\maxof{\mylenone}{\mylentwo}}
\settowidth{\mylentwo}{कृपा करहु गुरुदेव की नाईं}
\setlength{\mylenone}{\maxof{\mylenone}{\mylentwo}}
\setlength{\mylentwo}{\baselineskip}
\setlength{\mylenone}{\mylenone + 1pt}
\setlength{\mylen}{(\textwidth - \mylenone)*\real{0.5}}
\begin{longtable}[l]{@{\hspace*{\mylen}}>{\setlength\parfillskip{0pt}}p{\mylenone}@{}@{}l@{}}
 & \\[-\the\mylentwo]
जय जय जय हनुमान गोसाईं & ।\\ \nopagebreak[1mm]
कृपा करहु गुरुदेव की नाईं & ॥ ३७ ॥
\end{longtable}
}

\parasepone
\index[ardha]{जयजयजयहनुमानगोसाईं(चौ.३७,पूर्वार्ध)@जय जय जय हनुमान गोसाईं (चौ. ३७, पूर्वार्ध)}
\index[ardha]{कृपाकरहुगुरुदेवकीनाईं(चौ.३७,उत्तरार्ध)@कृपा करहु गुरुदेव की नाईं (चौ. ३७, उत्तरार्ध)}
\index[pada]{जय@जय}
\index[pada]{जय@जय}
\index[pada]{जय@जय}
\index[pada]{हनुमान@हनुमान}
\index[pada]{गोसाईं@गोसाईं}
\index[pada]{कृपा@कृपा}
\index[pada]{करहु@करहु}
\index[pada]{गुरुदेव@गुरुदेव}
\index[pada]{की@की}
\index[pada]{नाईं@नाईं}
\begin{sloppypar}\justifying\hyphenrules{nohyphenation}
\textbf{शब्दार्थ}—\textbf{\textit{गोसाईं}} {\unifont{\relscale{0.7}▶}} इन्द्रियोंके स्वामी।
\end{sloppypar}
\begin{sloppypar}\justifying\hyphenrules{nohyphenation}
\textbf{अर्थ}—हे गोसाईं हनुमान्‌जी महाराज! आपकी जय हो! जय हो!! जय हो!!! आप गुरुदेवकी भाँति वात्सल्यपूर्ण कृपा करें।
\end{sloppypar}
\parasepone
\begin{sloppypar}\justifying\hyphenrules{nohyphenation}
\textbf{व्याख्या}—हनुमान्‌जी सच्चिदानन्द हैं, इसलिए गोस्वामीजी तीन बार \textbf{\textit{जय}} शब्दका प्रयोग करते हैं। प्रथम \textbf{\textit{जय}}से \textit{हनुमान्‌-चालीसा} ग्रन्थका उपक्रम किया था। पुनः \textbf{\textit{जय}}से उपसंहार करके उनसे कृपाकी याचना कर रहे हैं।
\end{sloppypar}
\begin{sloppypar}\justifying\hyphenrules{nohyphenation}
\textbf{\textit{कृपा करहु गुरुदेव~की नाईं}} अर्थात् कठोर कृपा न करें। जीवपर दो ही कृपा कर सकते हैं—गुरु एवं गोविन्द। गोविन्दकी कृपामें कठोरताके साथ कोमलता होती है, जैसे असुरोंके निग्रह में। किन्तु गुरुकृपा निरन्तर कोमलतासे ओत-प्रोत रहती है। यथा—
\end{sloppypar}
{\bfseries
\setlength{\mylenone}{0pt}
\settowidth{\mylentwo}{एक शूल मोहि बिसर न काऊ}
\setlength{\mylenone}{\maxof{\mylenone}{\mylentwo}}
\settowidth{\mylentwo}{गुरु कर कोमल शील सुभाऊ}
\setlength{\mylenone}{\maxof{\mylenone}{\mylentwo}}
\setlength{\mylentwo}{\baselineskip}
\setlength{\mylenone}{\mylenone + 1pt}
\setlength{\mylen}{(\textwidth - \mylenone)*\real{0.5}}
\begin{longtable}[l]{@{\hspace*{\mylen}}>{\setlength\parfillskip{0pt}}p{\mylenone}@{}@{}l@{}}
 & \\[-\the\mylentwo]
एक शूल मोहि बिसर न काऊ & ।\\ \nopagebreak
गुरु कर कोमल शील सुभाऊ & ॥\\ \nopagebreak
\caption*{—रा.च.मा. ७-११०-२}
\end{longtable}
}
\begin{sloppypar}\justifying\hyphenrules{nohyphenation}
अतः प्रभो! गुरुदेवकी भाँति कृपा करके आप मुझे श्रीरामप्रेमका ज्ञान कराएँ।
\end{sloppypar}
\paraseplotus
\pagebreak


\fancyhead[LE,RO]{{\textmd{\large चौ. ३८: जो शत बार पाठ कर कोई}}}
\phantomsection
\addcontentsline{toc}{section}{चौपाई ३८: जो शत बार पाठ कर कोई}
\centering{॥ श्रीराम ॥}
\begin{sloppypar}\justifying\hyphenrules{nohyphenation}
मूल (चौपाई)—
\end{sloppypar}

{\bfseries\relscale{1.2}
\setlength{\mylenone}{0pt}
\settowidth{\mylentwo}{जो शत बार पाठ कर कोई}
\setlength{\mylenone}{\maxof{\mylenone}{\mylentwo}}
\settowidth{\mylentwo}{छूटहिं बंदि महा सुख होई}
\setlength{\mylenone}{\maxof{\mylenone}{\mylentwo}}
\setlength{\mylentwo}{\baselineskip}
\setlength{\mylenone}{\mylenone + 1pt}
\setlength{\mylen}{(\textwidth - \mylenone)*\real{0.5}}
\begin{longtable}[l]{@{\hspace*{\mylen}}>{\setlength\parfillskip{0pt}}p{\mylenone}@{}@{}l@{}}
 & \\[-\the\mylentwo]
जो शत बार पाठ कर कोई & ।\\ \nopagebreak[1mm]
छूटहिं बंदि महा सुख होई & ॥ ३८ ॥
\end{longtable}
}

\parasepone
\index[ardha]{जोशतबारपाठकरकोई(चौ.३८,पूर्वार्ध)@जो शत बार पाठ कर कोई (चौ. ३८, पूर्वार्ध)}
\index[ardha]{छूटहिंबंदिमहासुखहोई(चौ.३८,उत्तरार्ध)@छूटहिं बंदि महा सुख होई (चौ. ३८, उत्तरार्ध)}
\index[pada]{जो@जो}
\index[pada]{शत@शत}
\index[pada]{बार@बार}
\index[pada]{पाठ@पाठ}
\index[pada]{कर@कर}
\index[pada]{कोई@कोई}
\index[pada]{छूटहिं@छूटहिं}
\index[pada]{बंदि@बंदि}
\index[pada]{महा@महा}
\index[pada]{सुख@सुख}
\index[pada]{होई@होई}
\begin{sloppypar}\justifying\hyphenrules{nohyphenation}
\textbf{शब्दार्थ}—\textbf{\textit{बंदि}} {\unifont{\relscale{0.7}▶}} लौकिक तथा पारलौकिक बन्धन।
\end{sloppypar}
\begin{sloppypar}\justifying\hyphenrules{nohyphenation}
\textbf{अर्थ}—यदि कोई इसका १०० बार पाठ श्रद्धा एवं भक्तिसे करेगा, उसके लौकिक एवं पारलौकिक बन्धन छूट जाएँगे एवं उसे महासुखकी प्राप्ति होगी।
\end{sloppypar}
\parasepone
\begin{sloppypar}\justifying\hyphenrules{nohyphenation}
\textbf{व्याख्या}—यहाँ \textbf{\textit{शत}} शब्द १०८का बोधक है तथा \textbf{\textit{बार}} शब्द दिन एवं संख्याका वाचक है। \textbf{\textit{कोई}} शब्द सर्वसाधारणके लिए अधिकार-समर्पक है, अर्थात् कोई भी मनुष्य (ब्राह्मण, क्षत्रिय, वैश्य, शूद्र, स्त्री, इतरधर्मी कोई भी) यदि \textit{हनुमान्‌-चालीसा}का १०८ बार प्रतिदिनके क्रमसे १०८ दिन तक पाठ करे तो निश्चित उसका अनिष्ट बन्धन छूट जाएगा। इस \textit{हनुमान्‌-चालीसा}को गोस्वामी तुलसीदासजी महाराजने समस्त प्राणिमात्रको तुलसीदल-प्रसाद-रूपमें वितरित किया है। यह फलश्रुति है।
\end{sloppypar}
\paraseplotus
\pagebreak


\fancyhead[LE,RO]{{\textmd{\large चौ. ३९: जो यह पढ़ै हनुमान-चलीसा}}}
\phantomsection
\addcontentsline{toc}{section}{चौपाई ३९: जो यह पढ़ै हनुमान-चलीसा}
\centering{॥ श्रीराम ॥}
\begin{sloppypar}\justifying\hyphenrules{nohyphenation}
मूल (चौपाई)—
\end{sloppypar}

{\bfseries\relscale{1.2}
\setlength{\mylenone}{0pt}
\settowidth{\mylentwo}{जो यह पढ़ै हनुमान-चलीसा}
\setlength{\mylenone}{\maxof{\mylenone}{\mylentwo}}
\settowidth{\mylentwo}{होय सिद्धि साखी गौरीसा}
\setlength{\mylenone}{\maxof{\mylenone}{\mylentwo}}
\setlength{\mylentwo}{\baselineskip}
\setlength{\mylenone}{\mylenone + 1pt}
\setlength{\mylen}{(\textwidth - \mylenone)*\real{0.5}}
\begin{longtable}[l]{@{\hspace*{\mylen}}>{\setlength\parfillskip{0pt}}p{\mylenone}@{}@{}l@{}}
 & \\[-\the\mylentwo]
जो यह पढ़ै हनुमान-चलीसा & ।\\ \nopagebreak[1mm]
होय सिद्धि साखी गौरीसा & ॥ ३९ ॥
\end{longtable}
}

\parasepone
\index[ardha]{जोयहपढ़ैहनुमानचलीसा(चौ.३९,पूर्वार्ध)@जो यह पढ़ै हनुमान-चलीसा (चौ. ३९, पूर्वार्ध)}
\index[ardha]{होयसिद्धिसाखीगौरीसा(चौ.३९,उत्तरार्ध)@होय सिद्धि साखी गौरीसा (चौ. ३९, उत्तरार्ध)}
\index[pada]{जो@जो}
\index[pada]{यह@यह}
\index[pada]{पढै@पढ़ै}
\index[pada]{हनुमान@हनुमान}
\index[pada]{चलीसा@चलीसा}
\index[pada]{होय@होय}
\index[pada]{सिद्धि@सिद्धि}
\index[pada]{साखी@साखी}
\index[pada]{गौरीसा@गौरीसा}
\begin{sloppypar}\justifying\hyphenrules{nohyphenation}
\textbf{शब्दार्थ}—\textbf{\textit{गौरीसा}} {\unifont{\relscale{0.7}▶}} भगवान् शिव।
\end{sloppypar}
\begin{sloppypar}\justifying\hyphenrules{nohyphenation}
\textbf{अर्थ}—अब गोस्वामीजी भविष्यत्कालमें वर्तमानकालका दर्शन कर बोल रहे हैं कि जो इस \textit{हनुमान्‌-चालीसा}को पढ़ेगा, उसको अवश्य लौकिक तथा पारलौकिक सिद्धि प्राप्त होगी। इस विषय-प्रतिज्ञाके साक्षी भगवान् शिव~हैं।
\end{sloppypar}
\parasepone
\begin{sloppypar}\justifying\hyphenrules{nohyphenation}
\textbf{व्याख्या}—\textit{हनुमान्‌-चालीसा} लौकिक तथा पारलौकिक सिद्धियोंकी प्रदाता है, इस प्रतिज्ञामें शिवजीका साक्ष्य देते हैं। जैसे \textit{विनयपत्रिका}में श्रीराम-नाम प्रतीत करते समय शिवजीको साक्षी-रूपमें उपस्थित किया है। यथा—\textbf{शंकर साखि जो राखि कहौं कछु तौ जरि जीह गरो} (वि.प. २२६-६)।
\end{sloppypar}
\paraseplotus
\pagebreak


\fancyhead[LE,RO]{{\textmd{\large चौ. ४०: तुलसीदास सदा हरि-चेरा}}}
\phantomsection
\addcontentsline{toc}{section}{चौपाई ४०: तुलसीदास सदा हरि-चेरा}
\centering{॥ श्रीराम ॥}
\begin{sloppypar}\justifying\hyphenrules{nohyphenation}
मूल (चौपाई)—
\end{sloppypar}

{\bfseries\relscale{1.2}
\setlength{\mylenone}{0pt}
\settowidth{\mylentwo}{तुलसीदास सदा हरि-चेरा}
\setlength{\mylenone}{\maxof{\mylenone}{\mylentwo}}
\settowidth{\mylentwo}{कीजै नाथ हृदय महँ डेरा}
\setlength{\mylenone}{\maxof{\mylenone}{\mylentwo}}
\setlength{\mylentwo}{\baselineskip}
\setlength{\mylenone}{\mylenone + 1pt}
\setlength{\mylen}{(\textwidth - \mylenone)*\real{0.5}}
\begin{longtable}[l]{@{\hspace*{\mylen}}>{\setlength\parfillskip{0pt}}p{\mylenone}@{}@{}l@{}}
 & \\[-\the\mylentwo]
तुलसीदास सदा हरि-चेरा & ।\\ \nopagebreak[1mm]
कीजै नाथ हृदय महँ डेरा & ॥ ४० ॥
\end{longtable}
}

\parasepone
\begin{sloppypar}\justifying\hyphenrules{nohyphenation}
\textbf{शब्दार्थ}—\textbf{\textit{हरि-चेरा}} {\unifont{\relscale{0.7}▶}} श्रीरामजीके सेवक।
\end{sloppypar}
\begin{sloppypar}\justifying\hyphenrules{nohyphenation}
\textbf{अर्थ}—हे कीशनाथ श्रीहनुमान्‌जी महाराज! आप निरन्तर भगवान् श्रीरामके कैङ्कर्यमें निरत रहते हैं, अतः उसी कृपालुतावश तुलसीदासके हृदयमें निवास कीजिए। यद्वा गोस्वामीजी कहते हैं कि आञ्जनेय! आप निरन्तर भगवद्भक्तके हृदयमें निवास कीजिए। अथवा, मैं तुलसीदास निरन्तर \textit{हरि} अर्थात् वानरश्रेष्ठ आपश्रीका सदैव दास हूँ; हे नाथ! आप मेरे हृदयमें श्रीराम-लक्ष्मण-सीता सहित निवास कीजिए।
\end{sloppypar}
\parasepone
\index[ardha]{तुलसीदाससदाहरिचेरा(चौ.४०,पूर्वार्ध)@तुलसीदास सदा हरि-चेरा (चौ. ४०, पूर्वार्ध)}
\index[ardha]{कीजैनाथहृदयमहँडेरा(चौ.४०,उत्तरार्ध)@कीजै नाथ हृदय महँ डेरा (चौ. ४०, उत्तरार्ध)}
\index[pada]{तुलसीदास@तुलसीदास}
\index[pada]{सदा@सदा}
\index[pada]{हरि@हरि}
\index[pada]{चेरा@चेरा}
\index[pada]{कीजै@कीजै}
\index[pada]{नाथ@नाथ}
\index[pada]{हृदय@हृदय}
\index[pada]{महँ@महँ}
\index[pada]{डेरा@डेरा}
\begin{sloppypar}\justifying\hyphenrules{nohyphenation}
\textbf{व्याख्या}—अन्तमें गोस्वामीजी अपने हृदयमें निवास करनेके लिए किंवा वैष्णवजनोंके लिए हनुमान्‌जीसे प्रार्थना करते हैं। \textbf{\textit{हरि-चेरा}} शब्दका \textbf{\textit{नाथ}}, \textbf{\textit{हृदय}}, तथा \textbf{\textit{तुलसीदास}} शब्दसे अन्वय करनेपर उपर्युक्त तीनों अर्थ संगत हो जाते हैं। अर्थात् \textbf{\textit{नाथ सदा हरि-चेरा तुलसीदास हृदय महँ डेरा कीजै}}—इस अन्वयसे प्रथम अर्थ; तथा \textbf{\textit{तुलसीदास नाथ हरि-चेरा हृदय महँ सदा डेरा कीजै}}—इस अन्वयसे द्वितीय अर्थकी पुष्टि हो जाती है। यहाँ (दूसरे अर्थ में) \textbf{\textit{तुलसीदास}} शब्दके साथ \textit{कहत} इस बाहरी क्रियाको जोड़ना पड़ता है, जैसे अन्यत्र। यथा—\textbf{शरद सरोरुह नैन तुलसी भरे सनेह जल} (रा.च.मा. २-२२६)। \textbf{\textit{तुलसीदास सदा हरि-चेरा नाथ हृदय महँ डेरा कीजै}}—इस अन्वयसे तृतीय अर्थकी पुष्टि होती है।
\end{sloppypar}
\paraseplotus
\pagebreak


\fancyhead[LE,RO]{{\textmd{\large दो.: पवनतनय संकट-हरन}}}
\phantomsection
\addcontentsline{toc}{section}{उपसंहार दोहा: पवनतनय संकट-हरन}
\centering{॥ श्रीराम ॥}
\begin{sloppypar}\justifying\hyphenrules{nohyphenation}
मूल (दोहा)—
\end{sloppypar}

{\bfseries\relscale{1.2}
\setlength{\mylenone}{0pt}
\settowidth{\mylentwo}{पवनतनय संकट-हरन मंगल-मूरति-रूप}
\setlength{\mylenone}{\maxof{\mylenone}{\mylentwo}}
\settowidth{\mylentwo}{राम लखन सीता सहित हृदय बसहु सुर-भूप}
\setlength{\mylenone}{\maxof{\mylenone}{\mylentwo}}
\setlength{\mylentwo}{\baselineskip}
\setlength{\mylenone}{\mylenone + 1pt}
\setlength{\mylen}{(\textwidth - \mylenone)*\real{0.5}}
\begin{longtable}[l]{@{\hspace*{\mylen}}>{\setlength\parfillskip{0pt}}p{\mylenone}@{}@{}l@{}}
 & \\[-\the\mylentwo]
पवनतनय संकट-हरन मंगल-मूरति-रूप & ।\\ \nopagebreak[1mm]
राम लखन सीता सहित हृदय बसहु सुर-भूप & ॥
\end{longtable}
}

\parasepone
\index[ardha]{पवनतनयसंकटहरनमंगलमूरतिरूप(उ.दो.,पूर्वार्ध)@पवनतनय संकट-हरन मंगल-मूरति-रूप (उ.दो., पूर्वार्ध)}
\index[ardha]{रामलखनसीतासहितहृदयबसहुसुरभूप(उ.दो.,उत्तरार्ध)@राम लखन सीता सहित हृदय बसहु सुर-भूप (उ.दो., उत्तरार्ध)}
\index[pada]{पवनतनय@पवनतनय}
\index[pada]{संकट@संकट}
\index[pada]{हरन@हरन}
\index[pada]{मंगल@मंगल}
\index[pada]{मूरति@मूरति}
\index[pada]{रूप@रूप}
\index[pada]{राम@राम}
\index[pada]{लखन@लखन}
\index[pada]{सीता@सीता}
\index[pada]{सहित@सहित}
\index[pada]{हृदय@हृदय}
\index[pada]{बसहु@बसहु}
\index[pada]{सुर@सुर}
\index[pada]{भूप@भूप}
\begin{sloppypar}\justifying\hyphenrules{nohyphenation}
\textbf{शब्दार्थ}—\textbf{\textit{पवनतनय}} {\unifont{\relscale{0.7}▶}} पवनके पुत्र। \textbf{\textit{सुर-भूप}} {\unifont{\relscale{0.7}▶}} देवताओंके राजा।
\end{sloppypar}
\begin{sloppypar}\justifying\hyphenrules{nohyphenation}
\textbf{अर्थ}—हे पवनके पुत्र! समस्त संकटोंको हरनेवाले मङ्गलमूर्ति-रूप!! समस्त देवताओंके अधिष्ठान-स्वरूप श्रीहनुमान्‌जी महाराज!!! आप श्रीराम, श्रीलक्ष्मण, एवं माँ मैथिलीके साथ हमारे हृदयमें निवास करें।
\end{sloppypar}
\parasepone
\begin{sloppypar}\justifying\hyphenrules{nohyphenation}
\textbf{व्याख्या}—चार विशेषण देकर श्रीहनुमान्‌जीको ही मन, बुद्धि, अहंकार, एवं चित्तको शुद्ध करनेमें सहायक सिद्ध करते हैं और पश्चात् राम, लक्ष्मण, एवं सीताजीके सहित हृदयमें विश्राम करनेकी प्रार्थना करके सर्वतोभावेन श्रीमन्मारुतिके ही श्रीचरणकमलकी शरणागतिको ही परम पुरुषार्थ बताकर ग्रन्थको विश्राम दे रहे हैं।
\end{sloppypar}
\vspace{0.5mm}

\phantomsection
\addcontentsline{toc}{section}{व्याख्याकारका उपसंहार}
\centering{॥ उपसंहारः ॥\\}
{\bfseries
\setlength{\mylenone}{0pt}
\settowidth{\mylentwo}{सुमिरि राम-सिय-चरन-कमल गुरु-पद-रज शिर धरि}
\setlength{\mylenone}{\maxof{\mylenone}{\mylentwo}}
\settowidth{\mylentwo}{चऊद्वार उत्कल-थल मारुतसुतहि ध्यान करि}
\setlength{\mylenone}{\maxof{\mylenone}{\mylentwo}}
\settowidth{\mylentwo}{संबत नभ-फल-ख-दृग सुमाधव शिव शनिवारा}
\setlength{\mylenone}{\maxof{\mylenone}{\mylentwo}}
\settowidth{\mylentwo}{शुक्ल दूज हनुमान-चलीसा मति अनुसारा}
\setlength{\mylenone}{\maxof{\mylenone}{\mylentwo}}
\settowidth{\mylentwo}{जुगुति-शास्त्र-सिद्धान्तमय वैष्णव-रीति-भगति-भरी}
\setlength{\mylenone}{\maxof{\mylenone}{\mylentwo}}
\settowidth{\mylentwo}{नाम महावीरी ललित लघु व्याख्या गिरिधर करी}
\setlength{\mylenone}{\maxof{\mylenone}{\mylentwo}}
\setlength{\mylentwo}{\baselineskip}
\setlength{\mylenone}{\mylenone + 1pt}
\setlength{\mylen}{(\textwidth - \mylenone)*\real{0.5}}
\begin{longtable}[l]{@{\hspace*{\mylen}}>{\setlength\parfillskip{0pt}}p{\mylenone}@{}@{}l@{}}
 & \\[-\the\mylentwo]
सुमिरि राम-सिय-चरन-कमल गुरु-पद-रज शिर धरि & ।\\ \nopagebreak
चऊद्वार उत्कल-थल मारुतसुतहि ध्यान करि & ॥\\
संबत नभ-फल-ख-दृग सुमाधव शिव शनिवारा & ।\\ \nopagebreak
शुक्ल दूज हनुमान-चलीसा मति अनुसारा & ॥\\
जुगुति-शास्त्र-सिद्धान्तमय वैष्णव-रीति-भगति-भरी & ।\\ \nopagebreak
नाम महावीरी ललित लघु व्याख्या गिरिधर करी & ॥\\ \nopagebreak
\end{longtable}
}
\centering{॥ श्रीहनुमते नमः ॥\\}
\paraseplotus
\cleardoublepage
