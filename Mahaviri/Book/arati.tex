% !TeX TS-program = lualatex
% This file is part of Mahāvīrī LaTeX Source Code
% Mahāvīrī LaTeX Source Code is free software: you can redistribute it and/or modify it under the terms of the GNU General Public License as published by the Free Software Foundation, either version 3 of the License, or (at your option) any later version.
% Mahāvīrī LaTeX Source Code is distributed in the hope that it will be useful, but WITHOUT ANY WARRANTY; without even the implied warranty of MERCHANTABILITY or FITNESS FOR A PARTICULAR PURPOSE. See the GNU General Public License for more details.
% You should have received a copy of the GNU General Public License along with Mahāvīrī LaTeX Source Code. If not, see <https://www.gnu.org/licenses/>.
\centering{रचयिता–हिन्दूधर्मोद्धारक जगद्गुरु आद्य रामानन्दाचार्य\footnote{\ इस पदको आचार्य रामचन्द्र शुक्ल, डॉ.~श्यामसुन्दर दास, सर~जॉर्ज ग्रियर्सन, और रामकुमार वर्मा सदृश अनेक विद्वानोंने रामानन्दाचार्यजीकी रचना माना है।}\\}
\vspace{0.25\baselineskip}
{\bfseries
\setlength{\mylenone}{0pt}
\settowidth{\mylentwo}{आरति कीजै हनुमान लला की}
\setlength{\mylenone}{\maxof{\mylenone}{\mylentwo}}
\settowidth{\mylentwo}{दुष्ट-दलन रघुनाथ-कला की}
\setlength{\mylenone}{\maxof{\mylenone}{\mylentwo}}
\settowidth{\mylentwo}{जाके बल गरजे महि काँपे}
\setlength{\mylenone}{\maxof{\mylenone}{\mylentwo}}
\settowidth{\mylentwo}{रोग सोग जाके सिमाँ न चाँपे}
\setlength{\mylenone}{\maxof{\mylenone}{\mylentwo}}
\settowidth{\mylentwo}{अंजनी-सुत महाबल-दायक}
\setlength{\mylenone}{\maxof{\mylenone}{\mylentwo}}
\settowidth{\mylentwo}{साधु संत पर सदा सहायक}
\setlength{\mylenone}{\maxof{\mylenone}{\mylentwo}}
\settowidth{\mylentwo}{बाएँ भुजा सब असुर सँघारी}
\setlength{\mylenone}{\maxof{\mylenone}{\mylentwo}}
\settowidth{\mylentwo}{दहिन भुजा सब संत उबारी}
\setlength{\mylenone}{\maxof{\mylenone}{\mylentwo}}
\settowidth{\mylentwo}{लछिमन धरनि में मूर्छि पड़्यो}
\setlength{\mylenone}{\maxof{\mylenone}{\mylentwo}}
\settowidth{\mylentwo}{पैठि पताल जमकातर तोड़्यो}
\setlength{\mylenone}{\maxof{\mylenone}{\mylentwo}}
\settowidth{\mylentwo}{आनि सजीवन प्रान उबार्यो}
\setlength{\mylenone}{\maxof{\mylenone}{\mylentwo}}
\settowidth{\mylentwo}{मही सबन कै भुजा उपार्यो}
\setlength{\mylenone}{\maxof{\mylenone}{\mylentwo}}
\settowidth{\mylentwo}{गाढ़ परे कपि सुमिरौं तोहीं}
\setlength{\mylenone}{\maxof{\mylenone}{\mylentwo}}
\settowidth{\mylentwo}{होहु दयाल देहु जस मोहीं}
\setlength{\mylenone}{\maxof{\mylenone}{\mylentwo}}
\settowidth{\mylentwo}{लंका कोट समुंदर खाई}
\setlength{\mylenone}{\maxof{\mylenone}{\mylentwo}}
\settowidth{\mylentwo}{जात पवनसुत बार न लाई}
\setlength{\mylenone}{\maxof{\mylenone}{\mylentwo}}
\settowidth{\mylentwo}{लंक प्रजारि असुर सब मार्यो}
\setlength{\mylenone}{\maxof{\mylenone}{\mylentwo}}
\settowidth{\mylentwo}{राजा राम कै काज सँवार्यो}
\setlength{\mylenone}{\maxof{\mylenone}{\mylentwo}}
\settowidth{\mylentwo}{घंटा ताल झालरी बाजै}
\setlength{\mylenone}{\maxof{\mylenone}{\mylentwo}}
\settowidth{\mylentwo}{जगमग जोति अवधपुर छाजै}
\setlength{\mylenone}{\maxof{\mylenone}{\mylentwo}}
\settowidth{\mylentwo}{जो हनुमान की आरति गावै}
\setlength{\mylenone}{\maxof{\mylenone}{\mylentwo}}
\settowidth{\mylentwo}{बसि बैकुंठ परम पद पावै}
\setlength{\mylenone}{\maxof{\mylenone}{\mylentwo}}
\settowidth{\mylentwo}{लंक बिधंस कियो रघुराई}
\setlength{\mylenone}{\maxof{\mylenone}{\mylentwo}}
\settowidth{\mylentwo}{रामानन्द आरती गाई}
\setlength{\mylenone}{\maxof{\mylenone}{\mylentwo}}
\settowidth{\mylentwo}{सुर नर मुनि सब करहिं आरती}
\setlength{\mylenone}{\maxof{\mylenone}{\mylentwo}}
\settowidth{\mylentwo}{जै जै जै हनुमान लाल की}
\setlength{\mylenone}{\maxof{\mylenone}{\mylentwo}}
\setlength{\mylentwo}{\baselineskip}
\setlength{\mylenone}{\mylenone + 1pt}
\setlength{\mylen}{(\textwidth - \mylenone)*\real{0.5}}
\begin{longtable}[l]{@{\hspace*{\mylen}}>{\setlength\parfillskip{0pt}}p{\mylenone}@{}@{}l@{}}
 & \\[-\the\mylentwo]
आरति कीजै हनुमान लला की & ।\\ \nopagebreak
दुष्ट-दलन रघुनाथ-कला की & ॥ १ ॥\\
जाके बल गरजे महि काँपे & ।\\ \nopagebreak
रोग सोग जाके सिमाँ न चाँपे & ॥ २ ॥\\
अंजनी-सुत महाबल-दायक & ।\\ \nopagebreak
साधु संत पर सदा सहायक & ॥ ३ ॥\\
बाएँ भुजा सब असुर सँघारी & ।\\ \nopagebreak
दहिन भुजा सब संत उबारी & ॥ ४ ॥\\
लछिमन धरनि में मूर्छि पड़्यो & ।\\ \nopagebreak
पैठि पताल जमकातर तोड़्यो & ॥ ५ ॥\\
आनि सजीवन प्रान उबार्यो & ।\\ \nopagebreak
मही सबन कै भुजा उपार्यो & ॥ ६ ॥\\
गाढ़ परे कपि सुमिरौं तोहीं & ।\\ \nopagebreak
होहु दयाल देहु जस मोहीं & ॥ ७ ॥\\
लंका कोट समुंदर खाई & ।\\ \nopagebreak
जात पवनसुत बार न लाई & ॥ ८ ॥\\
लंक प्रजारि असुर सब मार्यो & ।\\ \nopagebreak
राजा राम कै काज सँवार्यो & ॥ ९ ॥\\
घंटा ताल झालरी बाजै & ।\\ \nopagebreak
जगमग जोति अवधपुर छाजै & ॥ १० ॥\\
जो हनुमान की आरति गावै & ।\\ \nopagebreak
बसि बैकुंठ परम पद पावै & ॥ ११ ॥\\
लंक बिधंस कियो रघुराई & ।\\ \nopagebreak
रामानन्द आरती गाई & ॥ १२ ॥\\
सुर नर मुनि सब करहिं आरती & ।\\ \nopagebreak
जै जै जै हनुमान लाल की & ॥ १३ ॥\\
\end{longtable}
}
\parasepone
\fontsize{13.5}{17}\selectfont
\begin{sloppypar}\justifying\hyphenrules{nohyphenation}
\noindent डॉ.~रामाधार शर्मा द्वारा संपादित प्रस्तुत पाठके स्रोत हैं—(१)~डॉ.~पीताम्बर दत्त बड़थ्वाल (संपादित) (१९५५ ई.), \textit{रामानन्द की हिन्दी रचनाएँ} (प्रथम संस्करण), काशी: नागरी प्रचारिणी सभा,  पृष्ठ~७; और (२)~डॉ.~बदरीनारायण श्रीवास्तव (१९५७ ई.), \textit{रामानन्द साहित्य तथा हिन्दी साहित्य पर उसका प्रभाव}, प्रयाग: हिन्दी परिषद् प्रयाग विश्वविद्यालय, पृष्ठ~१३९। दोनों स्रोतोंमें कईं पाठभेद हैं, यथामति समीचीन पाठ ही यहाँ डॉ.~रामाधार शर्मा द्वारा प्रस्तुत किया गया है—संपादक।
\end{sloppypar}
