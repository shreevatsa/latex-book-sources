% !TeX TS-program = lualatex
% This file is part of Mahāvīrī LaTeX Source Code
% Mahāvīrī LaTeX Source Code is free software: you can redistribute it and/or modify it under the terms of the GNU General Public License as published by the Free Software Foundation, either version 3 of the License, or (at your option) any later version.
% Mahāvīrī LaTeX Source Code is distributed in the hope that it will be useful, but WITHOUT ANY WARRANTY; without even the implied warranty of MERCHANTABILITY or FITNESS FOR A PARTICULAR PURPOSE. See the GNU General Public License for more details.
% You should have received a copy of the GNU General Public License along with Mahāvīrī LaTeX Source Code. If not, see <https://www.gnu.org/licenses/>.
\begin{sloppypar}\justifying
\textit{महावीरी} व्याख्याकी अनवरत बढ़ती माँगको देखते हुए इसका चतुर्थ संस्करण प्रकाशित किया गया है। गुरुदेवके निर्देशानुसार चौपाई~२९की व्याख्यामें एक श्लोक जोड़ा गया है और एक-दो स्थानोंपर मूलपाठमें साधारण सुधार (यथा \textbf{चारों} के स्थान पर \textbf{चारिउ}) किया गया है। शेष सब तृतीय संस्करणका ही नया रूप है। मात्र एक दिनमें (१३–१४ मई, १९८३ ई.) प्रणीत इस व्याख्याके विषयमें डॉ.~रामचन्द्र प्रसादने लिखा था–\footnote{\ {\englishfont{\relscale{0.75} \foreignlanguage{english}{Ram Chandra Prasad (2008) [1990]. \textit{Shri Ramacharitamanasa: The Holy Lake of the Acts of Rama} (2nd ed.). Delhi: Motilal Banarsidass, ISBN 978-81-208-0443-2, p.~849, footnote 1.}}}}
\end{sloppypar}
\vspace{-1ex}
\begin{quote}
\begin{sloppypar}\justifying
“श्रीहनुमानचालीसा की सर्वश्रेष्ठ व्याख्या के लिए देखें महावीरी व्याख्या, जिसके लेखक हैं प्रज्ञाचक्षु आचार्य श्रीरामभद्रदासजी। श्रीहनुमानचालीसा के प्रस्तुत भाष्य का आधार श्रीरामभद्रदासजी की ही वैदुष्यमंडित टीका है। इसके लिए मैं आचार्य-प्रवर का ऋणी हूँ।”
\end{sloppypar}
\end{quote}
\vspace{-1ex}
\begin{sloppypar}\justifying
\textit{महावीरी} व्याख्याका चतुर्थ संस्करण सभी हनुमद्भक्तोंको समर्पित है।
\end{sloppypar}
\vspace{0.5\baselineskip}
%\begin{longtable}{|@{}C{0.5\textwidth}@{}|@{}C{0.5\textwidth}@{}|}
\begin{longtable}{@{}C{0.5\textwidth}@{}@{}C{0.5\textwidth}@{}}
\textbf{मनीषकुमार शुक्ल} & \textbf{नित्यानन्द मिश्र}\\
% बेङ्गलूरु & मुम्बई\\
\end{longtable}
\vspace{0.5\baselineskip}
\raggedleft{अपरा एकादशी, वि.सं. २०७८\\
(६ जून २०२१ ई.)\\}
% \paraseplotus

