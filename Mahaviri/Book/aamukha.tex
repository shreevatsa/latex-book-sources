% !TeX TS-program = lualatex
% This file is part of Mahāvīrī LaTeX Source Code
% Mahāvīrī LaTeX Source Code is free software: you can redistribute it and/or modify it under the terms of the GNU General Public License as published by the Free Software Foundation, either version 3 of the License, or (at your option) any later version.
% Mahāvīrī LaTeX Source Code is distributed in the hope that it will be useful, but WITHOUT ANY WARRANTY; without even the implied warranty of MERCHANTABILITY or FITNESS FOR A PARTICULAR PURPOSE. See the GNU General Public License for more details.
% You should have received a copy of the GNU General Public License along with Mahāvīrī LaTeX Source Code. If not, see <https://www.gnu.org/licenses/>.
\fancyhead[RE,LO]{{\textmd{\large \textit{महावीरी} व्याख्या}}}
\fancyhead[LE,RO]{{\textmd{\large आमुख}}}
{\bfseries
\setlength{\mylenone}{0pt}
\settowidth{\mylentwo}{उद्यच्चण्डकराभभव्यभुवनाभ्यर्चाप्रदीप्तं वपु-}
\setlength{\mylenone}{\maxof{\mylenone}{\mylentwo}}
\settowidth{\mylentwo}{र्बिभ्रन्मञ्जुलमौञ्जसूत्रमनघं घर्मघ्नकान्तस्मितम्}
\setlength{\mylenone}{\maxof{\mylenone}{\mylentwo}}
\settowidth{\mylentwo}{सीतारामपदारविन्दमधुपः प्रावृट्पयोदद्विषां}
\setlength{\mylenone}{\maxof{\mylenone}{\mylentwo}}
\settowidth{\mylentwo}{झञ्झावातनिभो भवाय भवतां भूयान्मुहुर्मारुतिः}
\setlength{\mylenone}{\maxof{\mylenone}{\mylentwo}}
\setlength{\mylentwo}{\baselineskip}
\setlength{\mylenone}{\mylenone + 1pt}
\setlength{\mylen}{(\textwidth - \mylenone)*\real{0.5}}
\begin{longtable}[l]{@{\hspace*{\mylen}}>{\setlength\parfillskip{0pt}}p{\mylenone}@{}@{}l@{}}
 & \\[-\the\mylentwo]
उद्यच्चण्डकराभभव्यभुवनाभ्यर्चाप्रदीप्तं वपु- & \\ \nopagebreak
र्बिभ्रन्मञ्जुलमौञ्जसूत्रमनघं घर्मघ्नकान्तस्मितम् & ।\\ \nopagebreak
सीतारामपदारविन्दमधुपः प्रावृट्पयोदद्विषां & \\ \nopagebreak
झञ्झावातनिभो भवाय भवतां भूयान्मुहुर्मारुतिः & ॥\\
\end{longtable}
}
\begin{sloppypar}\justifying
साहित्य-गगनके मरीचिमाली एवं कविता-कामिनी-यामिनीके शारद निष्कलङ्क शशाङ्क, रामभक्ति-भागीरथी-सनाथित-हृदय-धरातल, सकल-कविकुल-शेखर, वैष्णव-वृन्द-वृन्दारकेश, सीतारमण-पदपद्म-पराग-परिमल-मकरन्द-मधुकर, कलिपावनावतार, प्रातःस्मरणीय, परम आदरणीय श्रीमद्गोस्वामी तुलसीदासजी महाराजकी कृतियोंमें \textit{श्रीहनुमान्‌-चालीसा}को भी बहुचर्चित रूपमें स्थान प्राप्त है। \textbf{गुणगृह्या वचने विपश्चितः} (कि. २-५) की दृष्टिसे यह विषय विशेष आलोचनीय नहीं है, तथापि कुछ विचार करना अनुपयुक्त भी नहीं होगा। गोस्वामीजीके ही द्वारा श्रीकाशीमें प्रतिष्ठित श्रीसंकटमोचन-हनुमान्‌जीके मन्दिरमें भी यह \textit{हनुमान्‌-चालीसा} स्तोत्ररत्न भित्तिपर लिखा हुआ लेख रूपमें आज भी दृष्टिगोचर है। मानसजीके तथा गोस्वामीजीके अन्य सर्वमान्य ग्रन्थरत्नोंकी प्रसंग-संगति भी इस ग्रन्थके प्रसंगोंसे एकवाक्यतापन्न हो जाती है। यथा—\textbf{\textit{लाय सँजीवनि लखन जियाये}} (ह.चा. ११), \textbf{\textit{तुम उपकार सुग्रीवहिं कीन्हा}} (ह.चा. १६), \textbf{\textit{तुम्हरो मंत्र बिभीषन माना}} (ह.चा. १७) इत्यादि प्रसंग मानससे पूर्णतया मिलते हैं। श्रीहनुमान्‌-विभीषण-संवाद श्रीमानसजीके अतिरिक्त गोस्वामीजीके अन्य किसी ग्रन्थमें शब्दतः चर्चित नहीं है। पर मानसके इस गोपनीयतम प्रसंगरत्नकी चर्चा \textit{श्रीहनुमान्‌चालीसा}में \textbf{\textit{तुम्हरो मंत्र बिभीषन माना}} (ह.चा. १७) कह कर सूत्ररूपमें कर दी गई है। \textit{श्रीरामचरितमानस}में कथित प्रसंगोंकी गोस्वामीजीके अन्य ग्रन्थोंसे संगति लगाई जाती है। इसीलिए द्वादश ग्रन्थ मानसजीके पूरक माने जाते हैं। जैसे द्रोणाचलको लेकर श्रीअवधके ऊपर आते हुए हनुमान्‌जीको भरतजीने बिना फरके बाणसे विद्धकर नीचे गिराया, यथा—\textbf{परेउ मुरछि महि लागत सायक} (रा.च.मा. ६-५९-१)। पर पर्वतकी क्या दशा हुई, इसका स्पष्टीकरण मानसमें न करके गोस्वामीजीने इसके पोषक \textit{गीतावली} ग्रन्थमें किया है–\linebreak[0]\textbf{पर्यो कहि राम पवन राख्यो गिरि} (गी. ६-१०-२) अर्थात् हनुमान्‌जीने अपनेको गिरता हुआ जानकर द्रोणाचल पर्वतको पवनके हाथ सौंप दिया। एवमन्यत्रापि। जैसे गोस्वामीजीके अन्य ग्रन्थ मानसके प्रसंगोंके पूरक हैं, वैसे ही \textit{श्रीहनुमान्‌-चालीसा} भी है। यथा राघवने हनुमान्‌जीको मुद्रिका दी—
\end{sloppypar}
{\bfseries
\setlength{\mylenone}{0pt}
\settowidth{\mylentwo}{परसा शीष सरोरुह पानी}
\setlength{\mylenone}{\maxof{\mylenone}{\mylentwo}}
\settowidth{\mylentwo}{करमुद्रिका दीन्ह जन जानी}
\setlength{\mylenone}{\maxof{\mylenone}{\mylentwo}}
\setlength{\mylentwo}{\baselineskip}
\setlength{\mylenone}{\mylenone + 1pt}
\setlength{\mylen}{(\textwidth - \mylenone)*\real{0.5}}
\begin{longtable}[l]{@{\hspace*{\mylen}}>{\setlength\parfillskip{0pt}}p{\mylenone}@{}@{}l@{}}
 & \\[-\the\mylentwo]
परसा शीष सरोरुह पानी & ।\\ \nopagebreak
करमुद्रिका दीन्ह जन जानी & ॥\\ \nopagebreak
\caption*{—रा.च.मा. ४-२३-१०}
\end{longtable}
}
\begin{sloppypar}\justifying
\noindent पर इस मुद्रिकाको हनुमान्‌जी महाराजने कैसे एवं कहाँ सम्भाला, मानसके इस निगूढ प्रसंगका स्पष्टीकरण \textit{श्रीहनुमान्‌-चालीसा}में ही होता है। यथा—
\end{sloppypar}
{\bfseries
\setlength{\mylenone}{0pt}
\settowidth{\mylentwo}{प्रभु-मुद्रिका मेलि मुख माहीं}
\setlength{\mylenone}{\maxof{\mylenone}{\mylentwo}}
\settowidth{\mylentwo}{जलधि लाँघि गये अचरज नाहीं}
\setlength{\mylenone}{\maxof{\mylenone}{\mylentwo}}
\setlength{\mylentwo}{\baselineskip}
\setlength{\mylenone}{\mylenone + 1pt}
\setlength{\mylen}{(\textwidth - \mylenone)*\real{0.5}}
\begin{longtable}[l]{@{\hspace*{\mylen}}>{\setlength\parfillskip{0pt}}p{\mylenone}@{}@{}l@{}}
 & \\[-\the\mylentwo]
प्रभु-मुद्रिका मेलि मुख माहीं & ।\\ \nopagebreak
जलधि लाँघि गये अचरज नाहीं & ॥\\ \nopagebreak
\caption*{—ह.चा. १९}
\end{longtable}
}
\begin{sloppypar}\justifying
\noindent अतः इस परीक्षणमें भी यह ग्रन्थ खरा उतरा।
\end{sloppypar}
\begin{sloppypar}\justifying
भाषा एवं शैलीकी दृष्टिसे भी यह निन्द्य नहीं कहा जा सकता। सामान्य लोगोंके कल्याणार्थ गोस्वामीजीने \textbf{सुरसरि सम सब कहँ हित होई} (रा.च.मा. १-१४-९) की मान्यताके अनुसार अति सरल ग्राम्य भाषामें रचना करके मधुरतम शिष्ट एवं सुबोध ग्रामीण शब्दोंमें इसे सजाया है। यही कारण है कि यह विद्वानोंकी भी हृदयतन्त्रीको झङ्कृत करता है एवं अति गँवार निरक्षर महिलाओंके भी हृदय-श्रद्धा-सुमनका परम पावन मकरन्द होकर ग्रामीण भारती मधुकरीको भी गुनगुनवाता रहता है। आज यह \textit{हनुमान्‌-चालीसा} हिमाचलसे कन्या\-कुमारीतक प्रत्येक भारतवासीके मनोमन्दिरका देवता बना हुआ है, चाहे वह व्यक्ति किसी भी धर्म या संप्रदायका हो। विदेशोंके भी ७५ प्रतिशत भागोंमें \textbf{\textit{जय हनुमान ज्ञान-गुण-सागर}} (ह.चा. १) का नारा बुलन्द हो रहा है। गोस्वामीजीके अतिरिक्त और किसी मनीषीकी लेखनीमें ऐसी उत्कृष्ट लोकप्रियताका प्रवाह नहीं दृष्टिगोचर होता। अन्य ग्रन्थों जैसी लोकप्रियता तुलसीकृत \textit{हनुमान्‌-चालीसा}में विद्यमान है। प्रत्येक सनातन-धर्मी श्रीमानसजीके पाठ-प्रारम्भ तथा पाठ-विश्राममें \textit{हनुमान्‌-चालीसा}का संपुट लगाता है।
\end{sloppypar}
\begin{sloppypar}\justifying
यदि भाषापर विचार करें तो गोस्वामीजीके \textit{श्रीरामललानहछू}से सरल \textit{हनुमान्‌-चालीसा}की भाषा नहीं है। गोस्वामीजीके अन्य ग्रन्थोंके समान इसमें भी स्वभावतः अलंकार आए हैं। यथा—
\end{sloppypar}
{\bfseries
\setlength{\mylenone}{0pt}
\settowidth{\mylentwo}{कंचन-बरन बिराज सुबेसा}
\setlength{\mylenone}{\maxof{\mylenone}{\mylentwo}}
\settowidth{\mylentwo}{कानन कुंडल कुंचित केसा}
\setlength{\mylenone}{\maxof{\mylenone}{\mylentwo}}
\setlength{\mylentwo}{\baselineskip}
\setlength{\mylenone}{\mylenone + 1pt}
\setlength{\mylen}{(\textwidth - \mylenone)*\real{0.5}}
\begin{longtable}[l]{@{\hspace*{\mylen}}>{\setlength\parfillskip{0pt}}p{\mylenone}@{}@{}l@{}}
 & \\[-\the\mylentwo]
कंचन-बरन बिराज सुबेसा & ।\\ \nopagebreak
कानन कुंडल कुंचित केसा & ॥\\ \nopagebreak
\caption*{—ह.चा. ४}
\end{longtable}
}
\begin{sloppypar}\justifying
\noindent अतः भले ही यह \textit{हनुमान्‌-चालीसा} \textit{तुलसीग्रन्थावली}में न मुद्रित हो, पर गोस्वामीजीकी रचना होनेमें किसी भी सहृदयको संदेह नहीं होगा। इसका श्रद्धासे पाठ करनेपर बहुत-से लोगोंको सफलमनोरथ होते देखा एवं सुना गया है। प्रायः भक्त महात्मा जन \textit{हनुमान्‌-चालीसा}का उनचास(४९)-दिवसीय तथा अष्टोत्तरशत(१०८)-दिवसीय अनुष्ठान किया करते हैं। भीषण रोगसे आक्रान्त व्यक्ति भी इसका अनुष्ठान-विधिसे पाठ कर अति शीघ्र लाभ पाते हैं। परम श्रद्धेय, ब्रह्मलीन, अनन्तश्रीविभूषित स्वामी हरिहरानन्दजी सरस्वती (श्रीकरपात्रीजी महाराज) तो यहाँ तक कहते थे कि \textit{श्रीहनुमान्‌-चालीसा} आर्ष मन्त्रोंकी भाँति ही परमप्रमाण, सर्वशक्तिमान्, तथा सर्ववाञ्छाकल्पतरु है। यह अवधी भाषामें उपनिबद्ध तैंतालीस छन्दोंमें लिखा हुआ एक स्तोत्रकाव्य है, जिसे हम गोस्वामीजीकी सिद्ध रचना मानते हैं। \textit{श्रीहनुमान्‌-चालीसा}की भाषा-शैली गोस्वामीजीके अन्य ग्रन्थोंसे मिलती-जुलती है। \textit{श्रीहनुमान्‌-चालीसा}की सार्वभौमता एवं सर्वजन-सुलभताको देखते हुए कोई भी सहृदय सन्त इसे अनार्ष नहीं मान सकता। गोस्वामीजीकी द्वादश-ग्रन्थावलीके अन्तर्गत इस ग्रन्थका संग्रह न होना कोई विशेष महत्त्वका नहीं है क्योंकि बहुत-से ऐसे पद श्रीगोस्वामीजीके नामसे मिलते हैं जिनका संग्रह ग्रन्थावलीमें नहीं है, जबकि उनकी रचना-शैली क्वचित्-क्वचित् गोस्वामीजीके संगृहीत पदोंसे भी अधिक रुचिकर लगती है। यथा—
\end{sloppypar}
{\bfseries
\setlength{\mylenone}{0pt}
\settowidth{\mylentwo}{ठुमुकि चलत रामचन्द्र बाजत पैजनियाँ}
\setlength{\mylenone}{\maxof{\mylenone}{\mylentwo}}
\settowidth{\mylentwo}{किलकि किलकि उठत धाय परत भूमि लटपटाय}
\setlength{\mylenone}{\maxof{\mylenone}{\mylentwo}}
\settowidth{\mylentwo}{धाय मातु गोद लेत दशरथ की रनियाँ}
\setlength{\mylenone}{\maxof{\mylenone}{\mylentwo}}
\settowidth{\mylentwo}{अंचल रज अंग झारि बिबिध भाँति सौं दुलारि}
\setlength{\mylenone}{\maxof{\mylenone}{\mylentwo}}
\settowidth{\mylentwo}{तन मन धन वारि वारि कहत मृदु बचनियाँ}
\setlength{\mylenone}{\maxof{\mylenone}{\mylentwo}}
\settowidth{\mylentwo}{विद्रुम से अरुण अधर बोलत मुख मधुर मधुर}
\setlength{\mylenone}{\maxof{\mylenone}{\mylentwo}}
\settowidth{\mylentwo}{सुभग नासिका में चारु लटकत लटकनियाँ}
\setlength{\mylenone}{\maxof{\mylenone}{\mylentwo}}
\settowidth{\mylentwo}{तुलसिदास अति अनंद देख के मुखारबिंद}
\setlength{\mylenone}{\maxof{\mylenone}{\mylentwo}}
\settowidth{\mylentwo}{रघुबर-छबि के समान रघुबर-छबि बनियाँ}
\setlength{\mylenone}{\maxof{\mylenone}{\mylentwo}}
\setlength{\mylentwo}{\baselineskip}
\setlength{\mylenone}{\mylenone + 1pt}
\setlength{\mylen}{(\textwidth - \mylenone)*\real{0.5}}
\begin{longtable}[l]{@{\hspace*{\mylen}}>{\setlength\parfillskip{0pt}}p{\mylenone}@{}@{}l@{}}
 & \\[-\the\mylentwo]
ठुमुकि चलत रामचन्द्र बाजत पैजनियाँ & ॥\\ \nopagebreak
किलकि किलकि उठत धाय परत भूमि लटपटाय & ।\\ \nopagebreak
धाय मातु गोद लेत दशरथ की रनियाँ & ॥\\
अंचल रज अंग झारि बिबिध भाँति सौं दुलारि & ।\\ \nopagebreak
तन मन धन वारि वारि कहत मृदु बचनियाँ & ॥\\
विद्रुम से अरुण अधर बोलत मुख मधुर मधुर & ।\\ \nopagebreak
सुभग नासिका में चारु लटकत लटकनियाँ & ॥\\
तुलसिदास अति अनंद देख के मुखारबिंद & ।\\ \nopagebreak
रघुबर-छबि के समान रघुबर-छबि बनियाँ & ॥
\end{longtable}
}
\begin{sloppypar}\justifying
\noindent अहो! इस पदमें उपस्थित की हुई राघव सरकारकी यह भुवनमोहन झाँकी किस सहृदय मनको बालरूप श्रीरामभद्रकी ओर झटिति नहीं खींच लेती! यह पद अलंकार, रस, भक्ति, तथा संगीतकी दृष्टिसे अनुपम होता हुआ भी गोस्वामीजीके किसी भी ग्रन्थमें संगृहीत नहीं हो सका, पर ४०० वर्षोंसे चली आ रही अविच्छिन्न परम्परामें अद्यावधि यह गोस्वामीजीकी गेय रचनाओंका चूडामणि माना जाता है। ठीक यही तथ्य \textit{श्रीहनुमान्‌-चालीसा}के विषयमें भी जानना चाहिए।
\end{sloppypar} \begin{sloppypar}\justifying
\textit{श्रीहनुमान्‌-चालीसा}की श्रीतुलसीदासजीकी रचना होनेके पक्षमें एक और सशक्त प्रमाण उद्धृत किया जा रहा है। प्रायः गोस्वामीजीके अन्य ग्रन्थोंमें उनके द्वारा रचित एकमें दूसरे ग्रन्थके कतिपय पद्य उद्धृत देखे जाते हैं। यथा \textit{दोहावली}का प्रथम दोहा (\textbf{राम बाम दिसि जानकी}, दो. १) \textit{रामाज्ञा\-प्रश्न} (रा.प्र. ७-३-७) तथा \textit{वैराग्य-संदीपनी} (वै.सं. १)में ज्यों-का-त्यों उद्धृत है। इसी प्रकार श्रीमानसजीके लगभग १०० दोहे और सोरठे यथानुपूर्वी \textit{श्रीदोहावली}में संगृहीत हैं।
उदाहरणके लिए दो-एक देखे जाएँ–
\end{sloppypar}
{\bfseries
\setlength{\mylenone}{0pt}
\settowidth{\mylentwo}{एक छत्र एक मुकुटमनि सब बरनन पर जोउ}
\setlength{\mylenone}{\maxof{\mylenone}{\mylentwo}}
\settowidth{\mylentwo}{तुलसी रघुबर-नाम के बरन बिराजत दोउ}
\setlength{\mylenone}{\maxof{\mylenone}{\mylentwo}}
\setlength{\mylentwo}{\baselineskip}
\setlength{\mylenone}{\mylenone + 1pt}
\setlength{\mylen}{(\textwidth - \mylenone)*\real{0.5}}
\begin{longtable}[l]{@{\hspace*{\mylen}}>{\setlength\parfillskip{0pt}}p{\mylenone}@{}@{}l@{}}
 & \\[-\the\mylentwo]
एक छत्र एक मुकुटमनि सब बरनन पर जोउ & ।\\ \nopagebreak
तुलसी रघुबर-नाम के बरन बिराजत दोउ & ॥\\ \nopagebreak
\caption*{—रा.च.मा. १-२०}
\end{longtable}
}
\begin{sloppypar}\justifying
\noindent यही दोहा \textit{दोहावली} ग्रन्थका ९वाँ है। बालकाण्डका २७वाँ  दोहा (\textbf{राम-नाम नरकेसरी}) \textit{दोहावली}का २६वाँ है। ठीक इसी पद्धतिका अनुसरण \textit{श्रीहनुमान्‌-चालीसा}के प्रारम्भमें किया गया। अयोध्याकाण्डके प्रथम दोहेका \textit{श्रीहनुमान्‌-चालीसा}के मङ्गलाचरणमें प्रस्तुतीकरण ही \textit{हनुमान्‌-चालीसा}को निःसंदिग्ध कर देता है। अयोध्याकाण्डका प्रथम दोहा \textbf{श्रीगुरु-चरन-सरोज-रज निज-मन-मुकुर सुधारि} इत्यादि ज्यों-का-त्यों \textit{हनुमान्‌-चालीसा}के मङ्गलाचरणके रूपमें सनातन-धर्मावलम्बी आबालवृद्ध जन-जनके मुखमण्डलपर विराजमान है। अतः–
\end{sloppypar}
{\bfseries
\setlength{\mylenone}{0pt}
\settowidth{\mylentwo}{एतेहु पर करिहैं जे शंका}
\setlength{\mylenone}{\maxof{\mylenone}{\mylentwo}}
\settowidth{\mylentwo}{मोहि ते अधिक ते जड़ मति-रंका}
\setlength{\mylenone}{\maxof{\mylenone}{\mylentwo}}
\setlength{\mylentwo}{\baselineskip}
\setlength{\mylenone}{\mylenone + 1pt}
\setlength{\mylen}{(\textwidth - \mylenone)*\real{0.5}}
\begin{longtable}[l]{@{\hspace*{\mylen}}>{\setlength\parfillskip{0pt}}p{\mylenone}@{}@{}l@{}}
 & \\[-\the\mylentwo]
एतेहु पर करिहैं जे शंका & ।\\ \nopagebreak
मोहि ते अधिक ते जड़ मति-रंका & ॥\\ \nopagebreak
\caption*{—रा.च.मा. १-१२-८}
\end{longtable}
}
\begin{sloppypar}\justifying
\noindent इत्यलमतिपल्लवितेन।
\end{sloppypar}
\pagebreak

\begin{sloppypar}\justifying
\textit{श्रीहनुमान्‌बाहुक}की भाँति यह छोटे-छोटे मात्र ४३ छन्दोंमें उपनिबद्ध है। यह परम स्वस्त्ययन स्तोत्ररत्न समस्त ऐहलौकिक एवं पारलौकिक कामनाओंकी पूर्ति करता है। मैंने भी इसके विधिवत् प्रयोगका सद्यः फल देखा है। गोस्वामीजीकी ग्रन्थावलीमें संगृहीत न होनेके कारण आज तक पाश्चात्त्य-वासना-वासित-मनस्क साहित्यिक टीकाकार महानुभावों द्वारा उपेक्षया इसकी कोई टीका न लिखी जा सकी। कुछ वर्षों पूर्व श्रीइन्दुभूषण रामायणी द्वारा इसपर एक संक्षिप्त व्याख्या प्रस्तुत की गई। उसमें भी विषयका यथेष्ट व्यवस्थित प्रस्तुतीकरण नहीं हो पाया। अत एव गतवर्ष चौद्वार (उड़ीसा)में समायोजित श्रीसंकटमोचन हनुमान्‌जीके प्रतिष्ठा-महोत्सवके शुभ-अवसरपर अपने सद्गुरुदेव अनन्तश्रीविभूषित श्री~श्री~१०८ श्रीरामचरणदासजी महाराज (फलाहारी बाबा सरकार, अरैल, प्रयाग) के आदेशानुसार मैंने \textit{श्रीहनुमान्‌-चालीसा}पर लघु व्याख्या प्रस्तुत करनेका बाल-सुलभ प्रयास किया है। यह कितने अंशोंमें सफल हो पाया है, इसका आकलन सन्त महानुभाव ही कर सकते हैं; क्योंकि \textbf{हेम्नः संलक्ष्यते ह्यग्नौ विशुद्धिः श्यामिकाऽपि वा} (र.वं. १-१०)। शास्त्र-स्वाध्यायमें असमर्थता तथा मानव-स्वभावजन्य-प्रमाद-वशात् यदि कुत्रचित् त्रुटि हो गई हो तो भगवद्भक्तजन उसे क्षमा करेंगे।
\end{sloppypar}
\begin{sloppypar}\justifying
\textit{श्रीहनुमान्‌-चालीसा}में कुल ४३ पद हैं, जो \textit{दोहा} तथा \textit{चौपाई} छन्दमें निबद्ध हैं। इसके प्रारम्भमें दो दोहे तथा उपसंहारमें एक दोहा है, शेष ४० चौपाइयाँ हैं। ३२ मात्राओंकी एक पङ्क्तिको एक चौपाई मानकर प्रत्येक पङ्क्तिको पूर्ण छन्द स्वीकार करके ही उनकी ४० संख्याके आधारपर ग्रन्थका नाम \textit{श्रीहनुमान्‌-चालीसा} रखा गया। ६४ मात्राओंकी दो-दो पङ्क्तियोंको एक चौपाई मानना भ्रमपूर्ण और अशास्त्रीय है। यदि दो-दो पङ्क्तियोंको मिलाकर चौपाई होगी तो \textit{हनुमान्‌-चालीसा} सिद्ध न होगा क्योंकि चालीस पङ्क्तियोंमें बीस ही चौपाइयाँ होंगी। इस दृष्टिसे \textit{हनुमान्‌-बीसा} कहना उचित होगा; जबकि तुलसीदासजीने स्वयं \textit{हनुमान्‌-चालीसा} कहा है, यथा—\textbf{\textit{जो यह पढ़ै हनुमान-चलीसा}} (ह.चा. ३९)। श्रीमानसजीमें भी जहाँ-जहाँ विषम संख्यापर पङ्क्ति आई है, उस प्रत्येक पङ्क्तिको प्रत्येक टीकाकारने स्वतन्त्र चौपाईके रूपमें मानकर उसकी टीका की है। उदाहरणार्थ–
\end{sloppypar}
\begin{sloppypar}\justifying
(१) बालकाण्ड २-१३\nopagebreak
\end{sloppypar}
{\bfseries
\setlength{\mylenone}{0pt}
\settowidth{\mylentwo}{अकथ अलौकिक तीरथराऊ}
\setlength{\mylenone}{\maxof{\mylenone}{\mylentwo}}
\settowidth{\mylentwo}{देइ सद्य फल प्रगट प्रभाऊ}
\setlength{\mylenone}{\maxof{\mylenone}{\mylentwo}}
\setlength{\mylentwo}{\baselineskip}
\setlength{\mylenone}{\mylenone + 1pt}
\setlength{\mylen}{(\textwidth - \mylenone)*\real{0.5}}
\begin{longtable}[l]{@{\hspace*{\mylen}}>{\setlength\parfillskip{0pt}}p{\mylenone}@{}@{}l@{}}
 & \\[-\the\mylentwo]
अकथ अलौकिक तीरथराऊ & ।\\ \nopagebreak
देइ सद्य फल प्रगट प्रभाऊ & ॥
\end{longtable}
}
\begin{sloppypar}\justifying
(२) अयोध्याकाण्ड ८-७\nopagebreak
\end{sloppypar}
{\bfseries
\setlength{\mylenone}{0pt}
\settowidth{\mylentwo}{गावहिं मंगल कोकिलबयनी}
\setlength{\mylenone}{\maxof{\mylenone}{\mylentwo}}
\settowidth{\mylentwo}{बिधुबदनी मृगशावक-नयनी}
\setlength{\mylenone}{\maxof{\mylenone}{\mylentwo}}
\setlength{\mylentwo}{\baselineskip}
\setlength{\mylenone}{\mylenone + 1pt}
\setlength{\mylen}{(\textwidth - \mylenone)*\real{0.5}}
\begin{longtable}[l]{@{\hspace*{\mylen}}>{\setlength\parfillskip{0pt}}p{\mylenone}@{}@{}l@{}}
 & \\[-\the\mylentwo]
गावहिं मंगल कोकिलबयनी & ।\\ \nopagebreak
बिधुबदनी मृगशावक-नयनी & ॥
\end{longtable}
}
\begin{sloppypar}\justifying
(३) अरण्यकाण्ड १२-१४ (कुछ प्रतियोंमें १२-१३)\nopagebreak
\end{sloppypar}
{\bfseries
\setlength{\mylenone}{0pt}
\settowidth{\mylentwo}{जहँ लगि रहे अपर मुनि-बृंदा}
\setlength{\mylenone}{\maxof{\mylenone}{\mylentwo}}
\settowidth{\mylentwo}{हरषे सब बिलोकि सुखकंदा}
\setlength{\mylenone}{\maxof{\mylenone}{\mylentwo}}
\setlength{\mylentwo}{\baselineskip}
\setlength{\mylenone}{\mylenone + 1pt}
\setlength{\mylen}{(\textwidth - \mylenone)*\real{0.5}}
\begin{longtable}[l]{@{\hspace*{\mylen}}>{\setlength\parfillskip{0pt}}p{\mylenone}@{}@{}l@{}}
 & \\[-\the\mylentwo]
जहँ लगि रहे अपर मुनि-बृंदा & ।\\ \nopagebreak
हरषे सब बिलोकि सुखकंदा & ॥
\end{longtable}
}
\begin{sloppypar}\justifying
(४) किष्किन्धाकाण्ड १०-५\nopagebreak
\end{sloppypar}
{\bfseries
\setlength{\mylenone}{0pt}
\settowidth{\mylentwo}{मम लोचन-गोचर सोइ आवा}
\setlength{\mylenone}{\maxof{\mylenone}{\mylentwo}}
\settowidth{\mylentwo}{बहुरि कि प्रभु अस बनिहि बनावा}
\setlength{\mylenone}{\maxof{\mylenone}{\mylentwo}}
\setlength{\mylentwo}{\baselineskip}
\setlength{\mylenone}{\mylenone + 1pt}
\setlength{\mylen}{(\textwidth - \mylenone)*\real{0.5}}
\begin{longtable}[l]{@{\hspace*{\mylen}}>{\setlength\parfillskip{0pt}}p{\mylenone}@{}@{}l@{}}
 & \\[-\the\mylentwo]
मम लोचन-गोचर सोइ आवा & ।\\ \nopagebreak
बहुरि कि प्रभु अस बनिहि बनावा & ॥
\end{longtable}
}
\begin{sloppypar}\justifying
(५) सुन्दरकाण्ड १-९\nopagebreak
\end{sloppypar}
{\bfseries
\setlength{\mylenone}{0pt}
\settowidth{\mylentwo}{जलनिधि रघुपति दूत बिचारी}
\setlength{\mylenone}{\maxof{\mylenone}{\mylentwo}}
\settowidth{\mylentwo}{कह मैनाक होहु श्रमहारी}
\setlength{\mylenone}{\maxof{\mylenone}{\mylentwo}}
\setlength{\mylentwo}{\baselineskip}
\setlength{\mylenone}{\mylenone + 1pt}
\setlength{\mylen}{(\textwidth - \mylenone)*\real{0.5}}
\begin{longtable}[l]{@{\hspace*{\mylen}}>{\setlength\parfillskip{0pt}}p{\mylenone}@{}@{}l@{}}
 & \\[-\the\mylentwo]
जलनिधि रघुपति दूत बिचारी & ।\\ \nopagebreak
कह मैनाक होहु श्रमहारी & ॥
\end{longtable}
}
\begin{sloppypar}\justifying
(६) युद्धकाण्ड ८०-११\nopagebreak
\end{sloppypar}
{\bfseries
\setlength{\mylenone}{0pt}
\settowidth{\mylentwo}{सखा धर्ममय अस रथ जाके}
\setlength{\mylenone}{\maxof{\mylenone}{\mylentwo}}
\settowidth{\mylentwo}{जीतन कहँ न कतहुँ रिपु ताके}
\setlength{\mylenone}{\maxof{\mylenone}{\mylentwo}}
\setlength{\mylentwo}{\baselineskip}
\setlength{\mylenone}{\mylenone + 1pt}
\setlength{\mylen}{(\textwidth - \mylenone)*\real{0.5}}
\begin{longtable}[l]{@{\hspace*{\mylen}}>{\setlength\parfillskip{0pt}}p{\mylenone}@{}@{}l@{}}
 & \\[-\the\mylentwo]
सखा धर्ममय अस रथ जाके & ।\\ \nopagebreak
जीतन कहँ न कतहुँ रिपु ताके & ॥
\end{longtable}
}
\begin{sloppypar}\justifying
(७) उत्तरकाण्ड ६४-९\nopagebreak
\end{sloppypar}
{\bfseries
\setlength{\mylenone}{0pt}
\settowidth{\mylentwo}{प्रभु अवतार कथा पुनि गाई}
\setlength{\mylenone}{\maxof{\mylenone}{\mylentwo}}
\settowidth{\mylentwo}{तब शिशुचरित कहेसि मन लाई}
\setlength{\mylenone}{\maxof{\mylenone}{\mylentwo}}
\setlength{\mylentwo}{\baselineskip}
\setlength{\mylenone}{\mylenone + 1pt}
\setlength{\mylen}{(\textwidth - \mylenone)*\real{0.5}}
\begin{longtable}[l]{@{\hspace*{\mylen}}>{\setlength\parfillskip{0pt}}p{\mylenone}@{}@{}l@{}}
 & \\[-\the\mylentwo]
प्रभु अवतार कथा पुनि गाई & ।\\ \nopagebreak
तब शिशुचरित कहेसि मन लाई & ॥
\end{longtable}
}
\begin{sloppypar}\justifying
\noindent यह दिग्दर्शन मात्र प्रस्तुत किया गया। \textit{पद्मावत}की समीक्षामें आचार्य रामचन्द्र शुक्लने भी कहा है कि जायसीने सात-सात चौपाइयों अर्थात् सात-सात पङ्क्तियोंके बाद दोहा रचा है। चौपाईका तात्पर्य चार यतियों वाले बत्तीस (३२) मात्राओंके मात्रिक वृत्तसे है। महर्षि वाल्मीकिजीको जिस प्रकार बत्तीस अक्षरों वाला अनुष्टुप् सिद्ध है, उसी प्रकार वाल्मीकिजीके अवतार गोस्वामी तुलसीदासजीको बत्तीस मात्राओं वाली चौपाई सिद्ध है।
\end{sloppypar}
\begin{sloppypar}\justifying
यह व्याख्या लगभग एक वर्ष पहले पूर्वदेशमें ही उपनिबद्ध की गई थी। मुझे लगता है कि इसी \textit{हनुमान्‌-चालीसा}की व्याख्याके फलने दासको \textit{रामभद्रदास} कहलानेका सौभाग्य दे दिया। मुझे आशा ही नहीं, अपितु पूर्ण विश्वास है कि इस ग्रन्थके अनुशीलनसे आस्तिक सनातन-धर्मी हनुमत्परायण तथा श्रीमानसके कथावाचक महानुभाव परम संतोषका अनुभव करेंगे। मैं समस्त वैष्णव सन्तोंके ही कर-कमलोंमें इस ग्रन्थोपहारको समर्पित कर उनके पादपद्मोंमें साष्टाङ्ग प्रणत हो रहा हूँ।
\end{sloppypar}
\begin{sloppypar}\justifying
श्रीवैष्णवेभ्यो नमो नमः।
\end{sloppypar}
\begin{sloppypar}\justifying
इति निवेदयति राघवीयः
\end{sloppypar}
\raggedleft{\textbf{रामभद्रदास:}\\
फलाहारी आश्रम, अरैल\\
प्रयाग (उत्तरप्रदेश)\\
गङ्गा दशहरा, वि. सं. २०४१ (जून ८, १९८४ ई.)\\
परिशोधित–मार्गशीर्ष पूर्णिमा, वि. सं. २०७२ (दिसम्बर २५, २०१५ ई.)\\}
\paraseplotus
